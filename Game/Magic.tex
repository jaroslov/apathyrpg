<section>
  <title>Introduction</title>

  <text>
Magic exists in many forms.  It is a supernatural energy that
permeates everything in the world.  Characters with the right skills
can bend this energy to their will<mdash />and perform amazing feats in the
process.  Magic is capable of healing the sick, raising the dead,
creating a great fireball, protecting the caster from harm<mdash />and
nearly anything else you can imagine.
  </text>

Magic, whatever type it might be, is all about manipulating MANA.
Mana, as mentioned above, is a supernatural energy that is present in
everything.  It is in the air, the ground, in your food, even in your
body.  Mana is what is responsible for <dquo>enforcing</dquo> physical laws
such as gravity, etc.  When a person uses magic, they are manipulating
mana to do what they want.  Normally, this is the mana that is in
their own body.  It is possible to work with mana outside one<squo />s body
as well.  Nobody really knows exactly how mana works, but special
gestures and body movements handed down throughout time are believed
to initiate complex quantum-mechanical events that result in seemingly
impossible things taking place<mdash />magic.

Mana is found in all things.  However, more mana is found in some
things than others.  Gases (air, for example) have very little mana.
Denser substances such as water and stone have more mana.  Complex
things such as man-made objects (pottery, woodcarvings, machinery,
etc.) contain quite a lot of mana.  Living things, from mice to humans
to whales, contain even more mana.  Mana is released back into the
environment when living things die, or when mana-rich objects decay,
are burned, or broken.  Other objects tend to <dquo>soak up</dquo> mana that is
released near them.  Thus, older items tend to contain more mana than
do <dquo>new</dquo> ones.  Whenever an item is broken it is possible to extract
some of its mana as it dissipates back into the environment.  This is
also possible when a living creature dies.  In game terms, we call
mana <dquo>MP</dquo> (Mana Points / Magic Points).

Examples of locations/things that are very rich in mana:
\begin{itemize}
\item Graveyards, grave markers
\item Large Churches, Cathedrals, holy symbols, idols, and relics
\item Museums, famous artwork or historical objects
\item Antiques
\item Ancient ruins, crypts, etc.
\end{itemize}

Examples of locations/things that have a lot of mana:
\begin{itemize}
\item Old houses, family heirlooms
\item Smaller churches and shrines.
\item Local landmarks such as wishing wells, old trees, popular meeting places
\item Very old people, especially those that are revered by others.
\item Important people with a lot of public exposure or reknown
  (royalty, politicians, popular musicians, etc)
\item Slaughterhouses
\item Fossils, natural gemstones
\end{itemize}

Examples of locations/things that have very little mana:
\begin{itemize}
\item Barren landscapes such as desert or arctic regions
\item Sky
\item Some underground caves or dungeons
\item <dquo>Ghost towns</dquo> / looted or abandoned buildings
\item Newly cleared land
\end{itemize}

\section{Types of Magic}

\subsection{Invocation Magic (spellcasting)}

Invocation spellcasting is the most common and basic form of magic.
This is the art of the wizard and sorceress, where characters use
innate mana to cast spells.  Invocation spells are based on
methodical, practiced, gestures and chanting of magical incantations.

Incantation magic is learned.  The invocation magician will have a
list of spells that she has studied and memorized.  As long as she has
mana (MP) available, she may cast any spell that she knows.  Because
invocation magicians use innate mana and well defined, repeatable,
forms it is the easiest type of magic to learn.

Spells are learned from other magicians, studious research, or a
combination of the two.  Research requires access to various books on
the subject of magic, including book(s) that are specifically about
the spell to be learned.  True books on magical subjects are few and
far between, and books about high level spells are extremely valuable.
Invocation magicians have invented a special language to record and
describe magic spells and the procedures required to cast them.  All
\emph{true} magical books will be written, at least in part, in this
special language.  Wizards and Sorceresses might hail from different
countries, races, and moral standpoints<mdash />but they all share common
ground in the ability to read magical script.  Magical script is
specially designed to describe and transcribe magical spells,
therefore it is rarely used for other purposes.  It is difficult to
carry on a conversation, written or spoken, in the language of magic.

More often than not a wizard will belong to a particular clan or
school, where many wizards pool their knowledge and references for
members to share.  In such situations there will usually be more
experienced spellcasters that may teach their skills to junior
members.  Wizards may be children of existing wizards or they might be
apprentices or students at an established wizard<squo />s school or place of
business.  Wizard secret societies are common as well.  Some are
devoted to particular schools of magic.  Others are devoted to a
certain cause and happen to share arcane secrets to persue common
goals.  Wizard secret societies and wizard schools are places where
other wizards can learn new spells and find magical items.  However,
most magic secret societies have a strict hierarchy and code of
conduct.  Special oaths, tithes, and obedience to the order are often
required, and favors always entail repayment.

Spells are purchased (learned) similar to skills.  However, there are
no levels for spells.  A character either knows a given spell, or they
don<squo />t.  If you like you can think of this as a <dquo>level 1 skill</dquo>.

There exist different schools (aka categories) of magic.  Some Wizards
specialize in certain schools, but that is purely optional.  Any
character may learn spells from any category<mdash />but with the following
restrictions.  A character may only learn a given spell if she knows
at least one spell of lower rank from the same school.  As well, the
rank of the spell must be no higher than your \emph{spellcasting}
skill level.  For example, you cannot learn a rank 4 <dquo>Air</dquo> magic
spell unless you already know a rank 3 <dquo>Air</dquo> spell, AND you must
have Level 4 or higher \emph{spellcasting}.

You can learn ANY spell, regardless of type, if you have a
spellcasting skill at least three levels higher than the rank of the
spell.  For example, if you have a spellcasting skill level of 6, then
you could learn ANY spell of rank 3 or lower, regardless of school or
prerequisite lower-rank spell.

A character with level 1 or higher spellcasting can always learn Rank
1 magical spells from any school without restriction.

Note that a Wizard or Sorceress character must have at least one level
in Magical Studies in order to read spell books and to learn the
intonations required for magical incantations.  Beyond that Magical
Studies is a purely optional skill.

Characters that cast Invocation magic must have at least two free
points of Encumberance available.  If not, they are not able to cast
magic.


\subsection{Ritual Magic}

Ritual magic is performed in elaborate <dquo>performances</dquo> involving a
much human involvement.  Magic rituals involve material components
(ingredients) and a great deal of chanting, mumbling, gesturing, and
the like.  The traditional image of a witch, bent over a bubbling
cauldron containing newt eyes and salamander toenails is a good
example of ritual magic.  Whereas traditional spellcasting is an
almost scientific pursuit, ritual magic is more of an art. Rituals are
conducted with a trained eye and experience.  Because ritual
ingredients vary in quality and potency, precision has little
use. Ritual magicians are shamans, witches and warlocks.

Ritual magic is not religious.  In fact, it is no more religious than
making a bowl of oatmeal or mixing some cement.  However, in many
cultures ritual magic is heavily associated with religion, especially
in primitive, animistic, or tribal societies.  In these cultures
ritual magic is the domain of the <dquo>witch doctor</dquo> or <dquo>medicine
man</dquo><mdash />people who hold important positions in the tribe, often
including that of a kind of priest or spiritual leader.  Therefore,
there is a strong association between ritual magic and primitive,
pagan, religions.  Many ritual magicians are religious figures,
members of mysterious cults, etc.  Some ritual magicians have been
known to practice animal or even human sacrifice (an excellent, though
horrific, source of MP for rituals).  In these societies ritual magic
is often believed to be religious in nature despite the fact that it
is decidedly not.

Ritual magic users do not use innate MP to perform magic.  Rather,
they use material components (ingredients) to create magical effects.
Performing rituals requires a large amount of space and various
supplies and components. Ritual preparation and casting is a time
consuming and often messy proposition.  Therefore, most ritualists
have a set area where they can store their materials and perform
rituals: usually a quiet location away from prying eyes.  This might
be a purpose-built room in the home or a secret <dquo>lair</dquo> in a basement
or abandoned building.  It might even be a cave or hidden woodland
grotto.  Such an area is a very personal space for the ritualist, and
while it might be neatly organized or highly messy, it will be arrayed
exactly how its owner prefers.

Most ritual supplies are re-useable, and a ritualist<squo />s <dquo>kitchen</dquo>
will be well stocked with such items.  These include such things as
wands, knives, scales, measuring devices, a fireplace, cauldrons,
bottles, pots, strainers, mirrors, precious gems, candles, assorted
tools, and so on.  Approximately \$1000 * Rank worth of supplies is
required in a ritualist<squo />s <dquo>kitchen</dquo>.  (In other words, in order to
perform the Ritual Stoneform, which is a rank 6 spell, \$6000 worth of
supplies are required to be on hand) These supplies are reuseable and
are used for many different rituals.

Specific spells require different ingredients.  Ingredients could be
nearly anything, including herbs and roots, body parts (animal and
human), minerals, chemicals, plants, and so forth.  Some of these are
common and will be used for many different spells.  Others are quite
specific and rare, and have very specific applications.  A ritualist
will usually have a personal stockpile of ingredients and components
stored away.  This stockpile will usually include all but the most
expensive of components.  Of course, some components can be difficult
to find and some must be fresh, so any ritualist will definately have
the means to locate components that she needs.  Some components might
be gathered personally, through the use of skills such as herbalism,
alchemy, hunting, or scrounging.  Other components are purchased by
the ritualist.  Some components may be purchased at common merchants
while others must be purchased from special (and sometimes illicit)
vendors.  Most ritualists will have known connections through which
different components and supplies might be shared, purchased, or
loaned.

The cost of ingredients for a particular spell is approximately 1/10
of the <dquo>scroll or potion</dquo> cost from the spell chart. This assumes
fair prices in an open market.  However, this cost might vary greatly
depending on various circumstances.  For example, in a certain town
components for Fire spells might be common and therefore very
inexpensive, while components for Earth spells might be rare and
therefore much more expensive.

In addition to its specific <dquo>recipe</dquo>, any ritual spell will also
require the infusion of mana.  As mentioned above, in ritual magic the
MP does not come from the caster.  The caster must gather objects
containing sufficient MP to conduct the spell.  As mentioned above,
mana is present in nearly everything, but generally not in sufficient
quantity to do any good.  Therefore, the ritualist must seek out items
that are rich in MP to use when conducting a ritual.  This process
results in the destruction of the items used!  This might even entail
the sacrifice of living things<mdash />if such an amount of MP is required.
Generally, items yielding MP for ritual use will cost about \$1 per MP
required, though this will vary.  Much like individual spell
components, a ritualist will tend to hoard items that may be
sacrificed for MP.  Common items are living plants and small animals,
antiques, old artwork, fossils, old holy items, old clothing
(especially from special occasions) and well-used household goods.
MP-source items are not spell-dependant.  MP items can be used for ANY
spell.

\begin{itemize}
\item NO MP from the character are required for ritual magic
\item Each spell<squo />s components cost 1/10th of the <dquo>scroll/potion</dquo>
  cost from the spell table, PLUS \$1 for each MP required.
  Components are consumed and can only be used once.  (note that costs
  are approximate and should be determined by the GM)
\item The time required to prepare for a given ritual is equal to
  Rank-1 hours.  Once preparations are made, the ritual must be
  conducted within 12 hours or the preparations are wasted and must be
  begun again.  Preparation uses up one half of the required ritual
  ingredients.
\item The time required to perform a given ritual is [Time] minutes.
  Conducting the ritual itself must be done within 12 hours of
  preparations being made, and uses up the remaining half of the
  ritual ingredients.
\item	Ritual spells cost 1 less Rank to learn than normal
\item Ritual magicians learn free rituals as they gain levels in
  Ritual Magic, according to the chart below; note that they are free
  to purchase other spells as they desire.
\end{itemize}

\begin{table}[!htb]
\begin{center}
\begin{tabular}{c|r|r|r|r|r|r|r|r|r|r|}
\multicolumn{11}{c}{Rank of Ritual Spell} \\
\hline
\hline
Level of Ritual Magic & 1 & 2 & 3 & 4 & 5 & 6 & 7 & 8 & 9 & 10 \\
\hline
1 & 3 \\ \cline{1-3}
2 & 5 & 2 \\ \cline{1-4}
3 & 7 & 4 & 2 \\ \cline{1-5}
4 & 8 & 5 & 3 & 2 \\ \cline{1-6}
5 & 9 & 6 & 4 & 3 & 2 \\ \cline{1-7}
6 & 10 & 7 & 5 & 4 & 3 & 2 \\ \cline{1-8}
7 & 11 & 8 & 6 & 5 & 4 & 3 & 2 \\ \cline{1-9}
8 & 12 & 8 & 7 & 6 & 5 & 4 & 3 & 2 \\ \cline{1-10}
9 & 12 & 8 & 7 & 7 & 6 & 5 & 4 & 3 & 2 \\ \cline{1-11}
10 & 12 & 8 & 7 & 7 & 6 & 5 & 4 & 3 & 3 & 1 \\ \hline
\end{tabular}
\caption{Rank of Ritual Spells per Ritual Magic Level}
\end{center}
\end{table}

Ritual magicians may also perform a ritual in a special manner,
infusing the spell into an object, such as a weapon or magical totem.
Infusing a spell takes the same amount of time as performing the
ritual in normal fashion.  Once the spell is infused in an item, it
may be released very quickly.  Releasing a charge (the energy and
magical effects of a stored ritual) requires 1 action, regardless of
the original casting time of the spell.  Infused ritual items are only
good for twenty-four hours, at which point the charge is lost and the
item returns to normal.  The ritualist may create charges of extended
duration when he first performs the infusion, but each 24 hours of
time requires another batch of the requisite spell ingredients (not
Mana).  Therefore, it can become very expensive to make long duration
infused items.  An infused item may be used by anyone, not just the
ritualist.

\subsection{Religious Magic / Faith Magic}

Religious magic is magical effects that are brought into reality by
the combined will of the caster and the caster<squo />s god(s).  Whereas
primitive religions and cults tend towards the use of ritual magic,
Religious or Faith magic is practiced by priests of clearly defined
and well organized religions.  Most religions that practice Faith
magic have a clearly organized hierarchy within the church.  Religious
dogma and/or scripture is well defined and is extensively documented.
Faith-magic using religions are not necessarily <dquo>good</dquo>, but they are
usually known to the general public and often actively recruit new
members.

Religious magic is taught by a particular Church. Each faith has one
or more magic schools associated with it.  These nearly always include
\emph{Spiritual} and \emph{Life} magic, and may include others.  Most
faiths have 3 or 4 schools of spells.  A faith caster has access to
all spells in the school(s) associated with his \emph{faith}, except
those whose rank is higher than his level of faith, and those which
are specifically prohibited by his Church.  For example, a <dquo>good</dquo>
tending religion would typically deny harmful \emph{life}-magic spells
to their adherents, for example curse.

\begin{itemize}
\item	MP are consumed as normal spellcasting
\item	A holy symbol is required for religious magic casting
\item	A successful Faith check must be made for casting faith magic.
\end{itemize}

Priests, monks, and warriors of faith practice religious magic.  All
religious spells are channeled through a holy symbol (see below).

Once a day the caster must engage in prayer.  The exact means of
praying is left up to the players \& GM, but during this time the
priest will choose which spells he as access to for the remainder of
the day.  The caster may choose any combination of spells he has
access to, provided that he has enough MP (at the time of prayer) for
that particular combination.  The caster must allocate MP and spells
during this prayer session.  Until the next prayer session those are
the spells the caster may use.  A holy symbol of at least +0 quality
is required for this prayer, otherwise the caster cannot allocate
spells for that day.

A faith magic user always uses the \emph{faith} skill for any and all
magic-related skill rolls, including those that would normally require
\emph{spellcasting} or \emph{magical control}, etc.

A \emph{Holy Symbol} is always required to cast \emph{Faith} magic.
If a holy symbol is not available, then Faith magic simply cannot be
cast.  Thankfully, a simple gesture or trinket often suffices.  There
are different types of holy symbols.  <dquo>Better</dquo> and more significant
holy symbols offer a better bonus to Faith magic casting:

\begin{table}[htb]
\begin{center}
\begin{tabular}{p{2em}p{4.5in}}
  \textbf{Bonus} & \textbf{Description} \\
  -1 & A gesture made by the Faithful, or a crude marking (such as a cross hastily drawn on a parchment) \\
  0 & An inexpensive <dquo>trinket</dquo> type holy symbol, or a common copy of a holy book / scripture.  Generally mass-produced.  Might be given away at a particular church, or might be available for purchase.  Also a holy symbol made by a marginally skilled craftsman or apprentice.  Example: prayer beads, a simple carved wooden cross, a mass-printed holy book, etc.  Cost:  \$5 or more. \\
  1 & An unusually high quality trinket made by a good craftsman (such as a fine gold crucifix necklace).  Or, a good quality weapon bearing religious inscription and symbols. Or, an item from a particular church which is not available to the general public (such as the robes of an ordained priest), or a fine quality holy book.  Cost:  At least \$500 plus the cost of the <dquo>base item</dquo> \\
  2 & A trinket or jewelry made by a master craftsman and of fine pedigree and history.  Or, a weapon or other object of superior quality with appropriate engravings and blessings.  Cost: at least \$5000 plus the base cost of the item. \\
  3 & An artifact or relic of religious significance, generally has a famous history within the church, or a Legendary quality weapon blessed by a member of the church with Faith: 7 or higher.  Cost: \$1,000,000 or more. \\
  4 & An artifact or relic of tremendous religious significance.  Items like this are literally worshipped in the particular religion.  There will only be a handful of items like this in existence for any one religion at any given time (if any at all).  Examples:  Spear of Destiny, Holy Grail, piece of the true cross, skull of a major saint. Cost: priceless. \\
\end{tabular}
\caption{Bonuses provided by a \emph{Holy Symbol}}
\end{center}
\end{table}

Anytime a Faith magic caster has to make a magic-related roll
(<dquo>Magical Control</dquo> roll, or a strike roll associated with a magical
spell, etc), use the caster<squo />s Faith skill.  Whatever holy symbol that
the caster has is then applied as a bonus to that roll.

\subsection{Martial Magic / Super Power Combat}

Martial magic is the ability to perform supernatural <dquo>attacks</dquo> in
conjunction with normal combat, such as fist fighting, martial arts,
or combat with weapons.

Martial Magic always works in conjunction with some kind of verbal or
somatic component.  This means that the fighter must perform a
particular combat form (kata), stance, or must shout a particular
phrase in order to use the ability.

Martial magic is always taught from master to student.  It is a
closely guarded secret of warrior secret societies, secluded martial
arts temples, or families with great fighting traditions.  Once a
warrior masters the traditional (non-supernatural) combat arts, he may
be taught martial magic.  Since martial magic is highly dependent on
particular combat forms and stances, it is nearly impossible to learn
on one<squo />s own or from a book.

Magical effects that ordinarily require touch do so for the martial
magician as well.  This is normally done by striking the target with
some kind of melee attack.  The character does not have to deal damage
per se (if the target is wearing armor) but he must connect with a
solid blow.  The magical energy (effect) is transferred through the
blow.  Martial magic users may choose to use a normal touch instead.

Magical effects that are projectile in nature (such as
\emph{fireball}) and normally require a strike roll must be used in
conjunction with a missile or projectile weapon of some kind (thrown
weapon, arrow, dart, or even a bullet).

Magical effects that are autotargeting when performed by a spellcaster
(such as \emph{lightning bolt}) are no longer autotargeting.  The
Super Power combatant must strike with any attack of his choice, which
may be a ranged attack or a melee attack.

The effects of a martial magic spell are in addition to those normally
associated with a strike in combat.  If a martial artist punches a
target and also performs the spell \emph{Vampiric Drain}, then the
target will suffer normal damage from the Punch, as well as the damage
and effects from \emph{Vampiric Drain}.  Likewise, if a martial
magician shoots a target with an arrow in conjunction with
\emph{Energy Blast}, then both the arrow and \emph{Energy Blast} deal
damage.

Martial magic works similar to spellcasting, with the following notes:

\begin{itemize}
\item MP are consumed just like spellcasting
\item Touch spells require a successful strike roll in combat
\item Ranged spells require a missile weapon of some kind
\item Auto-target spells are now <dquo>cast</dquo> by a WIL check of the user
\item Martial Magic spells are learned/purchased just like normal Spells
\item Martial Magic users must have at least 1 level in \emph{Mysticism}
\end{itemize}

Rather than purchasing the \emph{spellcasting} skill, a Super Power
Fighter adds together his skill level of \emph{Martial Arts} and
\emph{Mysticism}.  The total is considered the skill level of
<dquo>Martial Magic</dquo>

\subsection{Mysticism / Psionics}

Psionics is the ability to perform magic or supernatural acts by pure
force of will.  No incantations, materials, or special preparation is
required.  A character that can perform these feats is called a
Mystic.

Whereas most magic users have trained to develop their abilities,
Mystic characters are more often <dquo>gifted</dquo> with magical abilities.
Quite often these powers manifest themselves as a result of a
traumatic event, such as a near-fatal accident, that is experienced by
the character.  They may also be abilities that are passed down
through a particular family<mdash />in which case the character is simply
born with his unique powers.  Mystic characters have an innate ability
to perform their abilities<mdash />which are completely independent of any
kind of structured <dquo>method</dquo> or <dquo>technique</dquo>.  While mystic
characters might join magical societies they cannot <dquo>learn</dquo> or
<dquo>teach</dquo> mystic techniques to others.  Psionic abilities are
something that a mystic discovers herself.

Mystic rules:

\begin{itemize}
\item MP are consumed just like spellcasting
\item No verbal, somatic, or material components are required.  A
  mystic can be totally still and silent and can sill use his
  abilities, though the time required to perform (use) a mystic
  ability is exactly the same as standard spellcasting.
\item Spells are purchased from the spell list at will.  There is no
  need to follow a particular category.  The only restriction is that
  learned spells must be no higher level than the character<squo />s
  \emph{Mysticism} skill.
\item Mystics always use the WIL attribute for magic-related rolls,
  such as \emph{Magical Control}, regardless of the normal attribute
  that would be used for that roll.
\end{itemize}

\subsection{Beast Magic (summoning)}

Beast Magic is a unique magic school.  It is the art of summoning
creatures to aid their master.  Any magic user (spellcasting, ritual,
etc.) may purchase summoning spells as normal.  However, a spellcaster
that specializes in Beast Magic (and has no other magical aptitudes)
may purchase summoning spells as if they are one rank lower than
normal.

There are five basic summoning spells:

\begin{description}
\item[Gate] allows the summoner to call forth a creature that appears
  somewhere near the caster.  The location of appearance must be
  within direct line of sight of the caster, and no more than 30 feet
  away.  Once the creature appears, it is free to do whatever it wants
  based on its own free will<mdash />the caster simply transports it to a
  new location.  It will return to its original location after 3D6
  rounds.

\item[Summon] allows the summoner to call forth a creature near the
  caster (like Gate).  This creature is bound to follow SIMPLE
  directives given by the caster.  The creature will not act in a
  hostile manner towards the caster.  If a summoner casts another
  Summon spell while another one is currently active, he must make a
  Magical Control roll.  Failure means that the new spell fails and
  the first expires immediately.

\item[Pact] allows the caster to form a complex bond with a creature,
  making that creature his Familiar.  The caster must first find a
  creature to use pact with<mdash />such as summoning the creature by gate
  or summon) The caster may <dquo>control</dquo> the creature in a manner
  similar to summon.  This is permanent and lasts until either the
  caster or the Familiar is slain.  If the caster casts Pact a while
  he already has a Familiar, the original spell expires and the caster
  suffers 1D6+3 U.  While active, the Familar is aware of what happens
  to the caster, and vice versa.  A basic empathic link exists between
  the creature and the caster.  The caster may make a Magical Control
  check in order to <dquo>listen in</dquo> on any of the familiar<squo />s senses at
  any point in time.

\item[Control] is very similar to summon except that the caster has
  exacting and specific control over the creature(s) summoned.  The
  caster can literally operate the creature(s) as though by <dquo>remote
  control</dquo>.  While Control is active, the summoner is responsible for
  literally every action made by the summoned creature<mdash />it has no
  will of its own.

\item[Transform] is a powerful technique in which the caster takes the
  form of the creature summoned.  The caster literally becomes the
  creature in question, but retains his own mental state and skills in
  addition to the skills of the summoned creature.  All physical
  attributes (STR, DEX, BLD) of the summoned creature replace those of
  the character, but the character retains his own mental attributes
  (INT, WIS, WIL, CHA).  This effect lasts 4x 1D4 turns.  It cannot be
  ended at will.  The transformation requires three actions to
  complete.  When the spell expires, the caster will be very tired,
  and will suffer the effects of slow, loosing 1 action per turn for
  1D4+1 turns.

\item[These] summon spells are used in conjunction with whatever
  creature <dquo>spells</dquo> a summoner knows.  In order to learn a creature
  spell, the Summoner must have encountered such a creature.  The rank
  of example creatures is shown below.  For others, see the Beastiary
  section.
\end{description}

These summon spells are used in conjunction with whatever creature
<dquo>spells</dquo> a summoner knows.  In order to learn a creature spell, the
Summoner must have encountered such a creature.  The rank of example
creatures is shown below.  For others, see the Beastiary section.

\begin{table}[htb]
\begin{center}
\begin{tabular}{c|p{4.5in}}
\textbf{Rank} & \textbf{Spell (choose specifically)} \\
\hline
\hline
1 & Insects, songbirds, rats, mice, other small rodents, frogs/toads, salamanders, misc. other small reptiles or amphibians. \\ \hline
2 & Dogs, cats, exotic small animals, larger birds (crow, raven, owl, etc.), common snakes.  Raccoon, possum, etc.  Bats, small flying/gliding mammals.  Wallaby. \\ \hline
3 & Very large dogs, Cow, goat, sheep, deer, antelope.  Large predatory birds (hawk, eagle), large reptiles (small crocodile, monitor lizard, large snake).  Lesser monkeys.  Lynx, Kangaroo \\ \hline
4 & Crocodile, Horse, Bull.  Exotic 4-legged animals.  Large constricting snakes.  Apes. \\ \hline
5 & Hippo, Cape buffalo, Lion, Black Bear \\ \hline
6 & Elephant, Rhinocerous, Grizzly or Polar bear.  Mammoth \\
\end{tabular}
\caption{Summonable animals by Rank}
\end{center}
\end{table}

For generic animals follow the table above.  For very specific,
exotic, or rare animals jump to the next higher rank.  (Example:
Summon Frog: rank 1.  Summon Brazilian Poison-dart frog: rank 2.)

Whenever a summoning spell is cast the Summoner must make a Magical
Control skill check.  The negative to the roll is the total summoning
rank(s) of the creatures to be summoned.  A summoner can attempt to
summon multiple of the same creature simultaneously, but not a mixture
of different creatures.  The caster has a bonus to this check based on
what <dquo>Beast Lore</dquo> skills he knows.

\subsection{Magic-Related Skill Summary}

\begin{itemize}
\item Spellcasting gives additional $(LVL^2* WIS)$ MP.  This is the
  governing skill for learning Invocation spells.
\item Magical Studies is the academic study of magic and how it works.
  It also gives the player the ability to identify magical items or
  effects (a simple skill test), and to determine what a particular
  magical item/effect does (GM determines difficulty of the test). It
  also conveys the ability to read and write magical script.  This
  skill does not convey the ability to use magic; this skill does not
  provide MP.
\item Magical Control gives additional $(LVL^2*WIL)*2$ MP, and is used
  to strike with and to control the effects of some spells.  While
  this skill is not required for magic use, it is commonly used.
\item Mysticism gives an additional $(LVL^3 * WIS)$ MP.  This is the
  governing skill for learning Mystic / Psionic spells.
\item Faith is not technically a magic skill.  However, it is used for
  Faith magic, and provides $(LVL^2 * WIL *1.5)$ MP.  Faith only
  provides MP to those characters who actually know Faith magic
  spells.  If a non-faith spellcaster has this skill (For example, a
  Wizard that also happens to be religious) then it does NOT provide
  any MP.
\item Ritual Magic is used for Ritual Magic spells.  No character can
  learn a ritual spell with a higher rank than her level of Ritual
  Magic.  This skill does not provide MP.
\end{itemize}


There exist different types or categories of magic.  The different
types of magic are:

\begin{description}
\item[Air] Air magic is varied and effective for combat mages.  The
  lower rank spells are a mixture between utility types and damaging
  spells.  The higher level combat spells are quite powerful.  Air
  magic contains Lightning spells, which have wildly varying damage
  potential.

\item[Life] Life magic is capable of curing the wounded, healing the
  sick, and even inflicting horrible diseases upon your opponents.  It
  is a favorite of religious orders, either for good or evil.  Most
  combat magic in this category is limited because the spells require
  touch to cast.  Some spells are capable of dealing a great deal of
  damage, but usually over a period of time.  Sufficiently advanced
  mages who practice life magic can raise the dead or kill the living
  with a simple gesture.

\item[Spiritual] Magic of the Spiritual category is uncommon at best.
  With relatively few low-rank spells it is usually reserved to
  advanced practitioners of magic.  However, it has some unique spells
  that make it very effective.  Spiritual magic is particularly well
  suited for combating demons or spirits.  Higher level mages can lay
  waste to legions of their enemies with single spell.

\item[Light] With few exceptions, Light magic is a utility category.
  It can be used to illuminate dark passages beneath the earth or
  deceive with illusion and tricks.  Light magic has many useful
  spells, many of which are available at low or middle levels.  It is
  a very useful type of magic for the rogue or healer-type character
  who wants to be useful but doesn<squo />t want to fight.  A few high level
  light spells can deal excellent damage, but only at a steep MP cost.

\item[Physical] Physical magic is an excellent choice for
  <dquo>dual-classed</dquo> characters that practice magic and also fight in
  melee combat.  At lower and middle levels, physical magic allows one
  to improve the abilities of others, and to manipulate the physical
  world with magic.  He can also repair broken weapons or mend
  machines.  At higher levels the school offers very effective
  defensive spells and the highly effective gravity-based offensive
  magic, which always deals solid damage.

\item[Arcane] Arcane magic has no direct offensive capability; this
  would appear to be a weak form of magic that has little use for an
  adventurer.  However, the ability of arcane magic to support,
  enhance, and empower the caster or his allies is unmatched.  It is
  excellent when paired with other magic of a more offensive nature.

\item[Fire] Fire magic is a traditional offensive type of magic.  It<squo />s
  spells deal high damage and are without equal for causing
  large-scale destruction.  They are also quite mana-efficient.
  However, fire magic is the easiest of all <dquo>supernatural</dquo> attacks
  to defend against.  Fire magic is also noisy, flashy, and very
  intimidating to one<squo />s opponents.  Unfortunatley, this makes it
  nearly impossible to use in a stealthy manner.

\item[Earth] Earth magic is a varied school that combines defensive
  and offensive spells.  While neither type of spell is the most
  powerful available, they are not to be underestimated.  Earth magic
  has a number of very useful utility spells.

\item[Water] At lower levels, water magic is fairly weak, though it
  does combine both offensive and defensive spells.  Middle ranks
  offer effective protection and more effective combat spells.
  Higher-level water magic is legendary in effectiveness and is very
  powerful, Though many of the offensive spells have a long casting
  time, they are nearly impossible to defend against.
\end{description}

\section{General notes}

Most spells automatically succeed (there is no skill check necessary,
except for Faith Magic).  Once cast, it is possible for some spells to
miss their target however.  For these spells, the player must make a
skill test using MAGICAL CONTROL (or \emph{Faith}) to determine if the
strike is successful.  See the spell listing for what spells are
auto-targeting and which aren<squo />t.

If the player is interrupted while spell casting (e.g. takes damage,
gets knocked around, etc.), he must make a skill check (Magical
Control).  If he fails the check, the spell fails, but the MP are not
lost.  \textbf{The caster may not speak while spellcasting, but may
  move around, dodge, and perform very simple physical activities.}

A spell always takes effect at the end of the casting period.  The MP
are consumed at the exact instant that the spell takes effect.

A casting time of zero (0) means that the caster may cast the spell
\textbf{anytime}, even in response to something (e.g. getting shot
at), or having no actions remaining at the time.  Such spells require
no command words, components, or gesturing of any kind.

Note that a Faith magic caster must always make a Faith roll in order
to cast any magical spell.

\subsubsection{Magic Points (MP)}

ANY character has innate MP; magic related skills will give additional MP.

MP are consumed whenever spells are cast.  They are recovered
according to the following chart:

\begin{table}[htb]
\begin{center}
\begin{tabular}{rl}
Strenuous Activity: & 0 per hour \\
Normal Activity: & 5 per hour \\
Restful Activity: & 10 per hour \\
Sleeping: & 20 per hour \\
Meditation or prayer: & 20 per hour \\
\end{tabular}
\caption{Mana Point replacement per hour.}
\end{center}
\end{table}

\subsubsection{Magical <dquo>Tools</dquo>: Books, Scrolls, and Potions}

There exist three different kinds of magic-related books.

Type 1 is a volume that contains detailed information on how to cast a
particular spell (or spells).  Books of this type are always written
in magical script.  Anyone with the Magical Studies skill can read and
understand this kind of book.  However, only those with Spellcasting
can make real use of the contents.  A Spellcasting character can
easily learn new spells from a book of this type.  Books of this type
are very valuable.  The cost of a spell book like this is equal to the
<dquo>Book</dquo> price column of the magic table.  If the book contains
multiple spells, then its value is equal to the sum of the costs for
each spell contained.  Generally speaking, books of this type only
contain multiple spells when they are compendia of low-level spells in
a given school of magic.

Type 2 is a volume that describes how to perform a particular spell in
ritual format.  Books of this type might be written in any language,
and are generally not written in magical script.  If a Ritual magic
using character can read a book of this type, it is something like a
cookbook for a particular ritual spell.  The character could keep the
book as a reference or could use it to learn the ritual described.
Ritual spell manuals are worth approximately one-quarter the cost of a
Type 1 book.

Type 3 is a book that describes other forms of magic.  Since Faith
magic, Martial Magic, and Mysticism are dependant on the individual
<dquo>casting</dquo> the spell and not on some external <dquo>formula</dquo> or
<dquo>recipe</dquo>, it is essentially impossible to truly document how to
perform a given spell with nothing more than a book.  These books are
of interest to sages and scholars who are interested in arcane
subjects.  However, they are little more than a curiosity to actual
practitioners of magic.  Given enough books of this type it is
possible for a sufficiently advanced magical practitioner to figure
out a new spell or technique, but it is no guarantee.  Books of this
type are worth approximatley one-tenth of the listed price, depending
on the exact subject matter contained within the volume.  They may be
written in any language.

Magical potions or scrolls are single-use magic spells.  Anybody can
use a potion or scroll<mdash />whether they know anything about magic or
not.  Spells that affect the caster or a willing recipient are
potions.  Drinking the potion causes the spell to take effect.  Other
spells take the form of scrolls.  Reading the scroll aloud causes the
spell to take effect.  Both scrolls and potions are single-use only.
Potions can be drunk in one action.  Scrolls ALWAYS take 3 actions to
use regardless of the spell.

A person with Magical Studies can create Books, Scrolls, and Potions
of any spell he/she knows.  Producing such works is a long and
expensive process:

\begin{description}
\item[Books] Requires a number of days equal to twice the rank of the
  spell squared to create.  No MP are required, but special precious
  inks and paper are required.  These materials cost approximately
  half the value of the book.

\item[Scrolls/Potions] Require $LVL^2$ hours to create.  They require
  MP equal to a normal casting of the spell to be expended, and
  require precious materials valued at not less than three-quarters
  the value of the finished item.  A ritualist may prepare a potion in
  half that time.
\end{description}

\subsubsection{Magic Terminology \& Misc.}

\begin{description}
\item[Black] spells are those associated with evil, death energy, and
  demonic influence.  While they are not necissiarily evil, they are
  generally indescriminate in their targets.  Black spells never
  affect demons, undead, or targets that are not living, such as
  ghosts or robots.

\item[White] spells are those that are comprised of life energy.  They
  are technically not <dquo>good</dquo> aligned, though they are a favorite of
  <dquo>good</dquo> religious orders.  White spells that cause damage only
  affect demons, undead, and spirits.

\item[Auto] spells are those that are automatically targeting.  That
  means that once cast the spell will never miss or malfunction.
  However, in order to cast an auto-targeting spell, the caster must
  be able to clearly see and identify his target.

\end{description}
