The governing skill for Martial Arts is the Academic Skill,
\emph{Martial Arts}, which is rank 2.  All combat skills fall under
the \ae{}gis of Martial Arts. This means that a character engages in
combat---whether or not they have any facility in \emph{Martial
  Arts}---they must still engage as a Martial Artist. For instance,
the Academic skill \emph{Hunting} covers many areas (ambuscade, rifle
or bow shooting, skinning, stalking etc.), however these skills are
related to a stress-free environment. Combat, on the other hand, is a
high-stress environment requiring special training for weapons. Those
skills considered useful for hunting have little use in a combat
arena---and may even be a detriment.

\textscbf{Please see section~\ref{sec:adv-ma} for the dedicated
  Martial Artist.}

\subsection{Rules}

\emph{Martial Arts} covers the use of all weapons (ranged, melee,
ultratech, mecha, et al.), in addition to hand-to-hand (HtH) combat,
and the set of specialized skills that are associated with the
\emph{Martial Arts} governing skill.  The maximum level that any
character may have in any Martial Arts skill is less than or equal to
his level in Martial Arts, i.e. if he has 3 Levels in Martial Arts,
then all of his Martial Arts skills must be Level 3 or less.  The
basic Martial Art skill confers upon the martial artist the ability to
purchase the different skills listed in this section.

For each level the martial artist has in Martial Arts they may choose
one martial arts skill; for this skill, any time the martial artist
would have to \emph{roll} the skill, they get an extra die to do so. A
skill may only be chosen
$\lfloor{{Level-in-Martial-Arts}\over{2}}\rfloor$ times (the Lvl in
Martial Arts, divided by two, rounded-down).

\quotexample[Kim improves his Nyrkkeily]{Kim NG\c{o}c G\v{u}ay is an
  expert in Norse Boxing (Nyrkkeily), he chooses, when he gets his
  first level in Martial Arts, he also buys a level in ``Punch/Kick;''
  in particular, he also assigns his level-bonus to ``Punch/Kick.''
  This means when he rolls to strike with a punch he rolls \emph{two}
  dice and not one.}

After the \emph{Martial Arts} skill has been purchased, skills from
the four basic categories may be acquired; the four categories
represent different capabilities that \emph{Martial Arts} covers. The
five categories are \textscbf{Basic}, \textscbf{Combat},
\textscbf{Defensive}, and \textscbf{Modifier}. The different categories
focus on different phases of Combat, such that \textscbf{Combat}
skills are designed to aid the \emph{Martial Artist} with movement,
entering/exiting combat, the knowledge of combat, deployment, and
higher-level strategy. \textscbf{Basic} skills are involved in the
nitty-gritty (basic) postures and moves to actually \textsl{strike}
the opponent and inflict \textsl{harm} or
\textsl{death}. \textscbf{Defensive} skills are designed to deflect,
dodge, and/or otherwise avoid the mishap of being harmed by another
pugilist. \textscbf{Modifier}s represent ways to modify
\textscbf{Basic} skills.

\begin{description}
\item[Basic] These skills represent Basic combat actions, such as
  punching or kicking, grappling, the use of weapons and other
  primitive attacks. \emph{Only Basic attacks may be \emph{modified}
    by a Modifier.}
\item[Combat] These are the skills for being in, around, or preparing
  for combat. For example, knowledge of the terrain, etiquette, being
  prepared to move quickly and with assurance.

\item[Defensive] These are the skills used for defending against an
  attack. They are the foil to the Basic skills.

\item[Modifier] A Modifier is a skill used to modify a Basic attack. A
  Modifier changes the behaviour of a Basic attack; the Basic attack
  is now considered a Special attack of the same name with the
  Modifier as the adjective, for instance, if the Martial Artist is
  \emph{modifying} a Punch with a Bash, then this is now a Special
  attack, ``Bashing Punch.'' Since a modified Basic attack counts as a
  Special attack, it may not be further modified, \emph{as only Basic
    attacks can be modified}. There are certain Combat skills which
  enable the Martial Artist to consider certain specific combinations
  of Modifier/Basic to be Basic attacks.
\end{description}

\noindent {\large \textscbf{A Modifier may not be used more than Lvl
    times per turn.}}

\quotexample[Candlestick modifies his punch]{Candlestick has two
  levels in \emph{Martial Arts}; he has bought \emph{Punch} at rank 3
  and level 3, and chosen it as his bonus; he also has \emph{Bash} at
  level 1 and \emph{Feint} at level 1. He is striking at the Giant
  Hispaniola Solenodon with a \emph{Bashing Punch}. Because he is
  using \emph{Bash} with his \emph{Punch} his punch is now considered
  ``special'' and may not be further modified, even though he could
  still use his \emph{Feint} this turn.}

\section{Futher Commentary}

You will note that there is no ``chi'' or ``super-powers'' in
\emph{Martial Arts}. In fact, the skills are deliberately kept
pedestrian---those possessed (or assumed to be possessed, if possible)
by the greatest ``non-mystical'' martial artists in the `real world.'
This is done deliberately to provide a clear separation between the
Supernormal---governed by \emph{Magic} and \emph{Traits}---and the
mundane reality of kicking, spitting, biting and \textsl{hurting}. If
you want to create a character that has truly jaw-dropping magical
abilities, like flying-through the air, or punching through a
fired-brick wall, bending steel, or leaping from roof-top-to-roof-top,
consider looking into the \emph{Magic} and \emph{Traits} section. In
\emph{Magic} there a several categories that would benefit a Martial
Artist tremondously.

\subsection{Advanced Martial Arts}
\label{sec:adv-ma}

Also there are several \emph{para-skills} which are tremondously
important to the \textsl{dedicated} Martial Artist: \emph{Kata},
\emph{Kung Fu}, \emph{Waza}, and \emph{Jitsu}. These skills allow the
Martial Artist to transform his very Basic and Defensive attacks,
allowing him to perform prodigious feats: a Savage Feinting Death
Strike is only possible by using these skills.

\begin{description}
\item[Jitsu] 
\end{description}