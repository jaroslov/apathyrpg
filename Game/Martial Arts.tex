The governing skill for <martial-arts /> is the Academic Skill,
<martial-arts />, which is rank 2.  All combat skills fall under
the <ae />gis of <martial-arts />. This means that a character engages in
combat<mdash />whether or not they have any facility in Martial
  Arts<mdash />they must still engage as a Martial Artist. For instance,
the Academic skill Hunting covers many areas (ambuscade, rifle
or bow shooting, skinning, stalking etc.), however these skills are
related to a stress-free environment. <combat />, on the other hand, is a
high-stress environment requiring special training for weapons. Those
skills considered useful for hunting have little use in a combat
arena<mdash />and may even be a detriment.

\subsection{Rules}

<martial-arts /> covers the use of all weapons (ranged, melee,
ultratech, mecha, et al.), in addition to hand-to-hand (HtH) combat,
and the set of specialized skills that are associated with the
<martial-arts /> governing skill.  The maximum level that any
character may have in any <martial-arts /> skill is less than or equal to
his level in <martial-arts />, i.e. if he has 3 Levels in <martial-arts />,
then all of his <martial-arts /> skills must be Level 3 or less.  The
basic Martial Art skill confers upon the martial artist the ability to
purchase the different skills listed in this section.

For each level the martial artist has in <martial-arts /> they may choose
one <martial-arts /> skill; for this skill, any time the martial artist
would have to roll the skill, they get an extra die to do so. A
skill may only be chosen
Level-in-<martial-arts />/2 times (the Lvl in
<martial-arts />, divided by two, rounded-down).

\quotexample[Kim improves his Nyrkkeily]{Kim NG<cedilla>o</cedilla>c G<vcap>o</vcap>ay is an
  expert in Norse Boxing (Nyrkkeily), he chooses, when he gets his
  first level in <martial-arts />, he also buys a level in <dquo>Punch/Kick;</dquo>
  in particular, he also assigns his level-bonus to <dquo>Punch/Kick.</dquo>
  This means when he rolls to strike with a punch he rolls two
  dice and not one.}

After the <martial-arts /> skill has been purchased, skills from
the four basic categories may be acquired; the four categories
represent different capabilities that <martial-arts /> covers. The
five categories are <basic />, <combat />,
<defensive />, and <modifier />. The different categories
focus on different phases of <combat />, such that <combat />
skills are designed to aid the Martial Artist with movement,
entering/exiting combat, the knowledge of combat, deployment, and
higher-level strategy. <basic /> skills are involved in the
nitty-gritty (basic) postures and moves to actually strike
the opponent and inflict harm or
death. <defensive /> skills are designed to deflect,
dodge, and/or otherwise avoid the mishap of being harmed by another
pugilist. <modifier />s represent ways to modify
<basic /> skills.

\begin{description}
\item[<basic />] These skills represent <basic /> combat actions, such as
  punching or kicking, grappling, the use of weapons and other
  primitive attacks. <note>Only <basic /> attacks may be modified
    by a <modifier />.</note>
\item[<combat />] These are the skills for being in, around, or preparing
  for combat. For example, knowledge of the terrain, etiquette, being
  prepared to move quickly and with assurance.

\item[<defensive />] These are the skills used for defending against an
  attack. They are the foil to the <basic /> skills.

\item[<modifier />] A <modifier /> is a skill used to modify a <basic /> attack. A
  <modifier /> changes the behaviour of a <basic /> attack; the <basic /> attack
  is now considered a <special /> attack of the same name with the
  <modifier /> as the adjective, for instance, if the Martial Artist is
  modifying a Punch with a Bash, then this is now a <special />
  attack, <dquo>Bashing Punch.</dquo> Since a modified <basic /> attack counts as a
  <special /> attack, it may not be further modified, as only <basic />
    attacks can be modified. There are certain <combat /> skills which
  enable the Martial Artist to consider certain specific combinations
  of <modifier />/<basic /> to be <basic /> attacks.
\end{description}

<note>A <modifier /> may not be used more than Lvl times per turn.</note>

\quotexample[Candlestick modifies his punch]{Candlestick has two
  levels in <martial-arts />; he has bought Punch at rank 3
  and level 3, and chosen it as his bonus; he also has Bash at
  level 1 and Feint at level 1. He is striking at the Giant
  Hispaniola Solenodon with a Bashing Punch. Because he is
  using Bash with his Punch his punch is now considered
  <dquo>special</dquo> and may not be further modified, even though he could
  still use his Feint this turn.}

\section{Futher Commentary}

You will note that there is no <dquo>chi</dquo> or <dquo>super-powers</dquo> in
<martial-arts />. In fact, the skills are deliberately kept
pedestrian<mdash />those possessed (or assumed to be possessed, if possible)
by the greatest <dquo>non-mystical</dquo> martial artists in the <squo>real world.</squo>
This is done deliberately to provide a clear separation between the
Supernormal<mdash />governed by Magic and Traits<mdash />and the
mundane reality of kicking, spitting, biting and hurting. If
you want to create a character that has truly jaw-dropping magical
abilities, like flying-through the air, or punching through a
fired-brick wall, bending steel, or leaping from roof-top-to-roof-top,
consider looking into the Magic and Traits section. In
Magic there a several categories that would benefit a Martial
Artist tremondously.

\subsection{Advanced <martial-arts />}
\label{sec:adv-ma}

Also there are several para-skills which are tremondously
important to the dedicated Martial Artist: Kata,
Kung Fu, Waza, and Jitsu. These skills allow the
Martial Artist to transform his very <basic /> and <defensive /> attacks,
allowing him to perform prodigious feats: a Savage Feinting Death
Strike is only possible by using these skills.

\begin{description}
\item[Jitsu] 
\end{description}