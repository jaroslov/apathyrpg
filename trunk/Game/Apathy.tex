
\documentclass[twoside]{book}
\usepackage{include}
\usepackage{pslatex}
\usepackage{psfonts}
\usepackage{multicol}
\usepackage{newcent}
\usepackage{ncntrsbk}
\usepackage{rotating}
\usepackage{tabularx}
\usepackage{array}
\usepackage{longtable}
\usepackage{multirow}
\usepackage{graphicx}
\usepackage{multicolumn}
\usepackage[T1]{fontenc}
\usepackage{hyperref}
\usepackage{wrapfig}
\usepackage[text={5.5in,8in},textheight=8in]{geometry}

\begin{document}

\newfont{\GIANT}{rpncr scaled 9500}
\newfont{\Giant}{rpncr scaled 4500}
\DeclareFixedFont{\apathyscbf}{OT1}{pnc}{b}{sc}{8}
\DeclareFixedFont{\rulescbf}{OT1}{pnc}{b}{sc}{11}
\DeclareFixedFont{\apathymn}{OT1}{pnc}{m}{n}{8}
\newcommand{\halfline}{\vspace{.5ex}}
\newcommand{\textscbf}[1]{\textsc{\textbf{#1}}}
\newcommand{\APATHY}{\textscbf{Apathy}}
\newcommand{\rulename}[1]
{
\noindent\rulescbf{#1}
}
\newcommand{\ruledesc}[1]
{
\parindent=5pt
\everypar{\hangindent=20pt \hangafter=1}
\normalsize #1
}
\newcommand{\ARPG}{
{\GIANT A}
\hspace{-3ex}\raise5ex\hbox{\Giant P}
\hspace{-4ex}{\GIANT A}
\hspace{-4ex}\raise5ex\hbox{\Giant THY}
\raise13ex\hbox{
\hspace{-23ex}\textscbf{R}\textsc{ole-}\textscbf{P}\textsc{laying}
\textscbf{G}\textsc{ame}
}
}
\newcounter{ExampleCounter}
\setcounter{ExampleCounter}{1}
\newcommand{\quotexample}[2][~]
{
\addcontentsline{lof}{section}{\arabic{ExampleCounter} \textsc{#1}}
\vbox{
\textscbf{\noindent Example \arabic{ExampleCounter} {\small \textsc{#1}}}
\begin{quotation}
{\small #2}
\end{quotation}
}
\addtocounter{ExampleCounter}{1}
}

\newcommand{\BracedWrapPicture}[2]
{
\begin{wrapfigure}{#1}{.40\textwidth}
  \vspace{-20pt}
  \begin{center}
    $\overbrace{\hspace{.40\textwidth}}$
    \vspace{-10pt}
    \includegraphics[width=0.38\textwidth]{#2}
    \vspace{-10pt}
    $\underbrace{\hspace{.40\textwidth}}$
  \end{center}
  \vspace{-15pt}
\end{wrapfigure}
}

\newcommand{\VBoxColumnPicture}[1]
{
\vbox{
\begin{center}
$\overbrace{\hspace{.8\columnwidth}}$
\vspace{-10pt}
\includegraphics[width=.76\columnwidth]{#1}
\vspace{-10pt}
$\underbrace{\hspace{.8\columnwidth}}$
\end{center}}
}


\begin{titlepage}
~\\~\\~\\~\\~\\~\\~\\~\\~\\~\\
\begin{center}
\ARPG \\

\hspace*{6em}
\vbox{\vspace{-2em}
\small Allan Moyse \\
Nathan Jones \\
Jacob Smith \\
Noah Smith \\
Chris Cook \\
Josh Kramer}

\vskip 2in
\textsc{Revision \#1.100000 (06 September 2007)}
\end{center}

\end{titlepage}

~
\setcounter{page}{1}
\pagenumbering{roman}
\setcounter{tocdepth}{3}
\tableofcontents
\newpage
\listoftables
\newpage
\listoffigures
\newpage
\pagenumbering{arabic}
\setcounter{page}{1}


  
\part{Game Manual}
    
\chapter{Core Rules}
    
\section{Introduction}
    \APATHY{}  is a paper \& pencil role-playing game.
            \APATHY{}'s authors have played a great many RPGs
            including D\&D, Heroes Unlimited, GURPS, Shadowrun, TMNT,
            various White Wolf games (Vampire, etc.), Deadlands, and a
            few others. While these games are all fun, we felt that each
            game had various shortcomings that could be improved upon.
            Enter \APATHY{}. \APATHY{}  is designed to be a very
            flexible role-playing system. \APATHY{}  can handle classic
            games with knights and dragons, as well as modern-day,
            Cyberpunk, superhero, or futuristic scenarios. It excels at
            fantasy games that blend all genres.
          \APATHY{}  is recommended for advanced role-players (or
            at least for experienced DM/GM/Storytellers). This is because
            the \APATHY{}  system is designed for a unique mix of speed
            and realism rather than carefully `balanced'
            rules. The \APATHY{}  rules make it very easy to make
            unbalanced (`min-maxed') or tweaked characters.
            \APATHY{}  is likewise not appropriate for `rules
            lawyers' or excessively anal GMs and players.
          \APATHY{}  was created around several core features.
            These are what make \APATHY{}  a unique game system:
            
              
               \APATHY{}  has no fixed character classes, such as
                `fighter' or `wizard', etc.
                Players can pick and choose every detail of their
                character, rather than being forced into set classes.
                
              
              
               \APATHY{}  contains rules for all manner of
                technological and supernatural powers, weapons, items,
                and skills for characters to use. These range from
                archaic concepts like martial arts and swords, through
                ultra-modern concepts such as cybernetics and
                mechaand the supernatural abilities of magic and
                psionics. These different elements can be used
                simultaneously without any one being excessively
                powerful.
                
              
              
                 There are no discrete `levels' for
                 character advancement. Characters develop and grow
                 constantly, rather than infrequent, sudden, increases
                 common to level-based games. There is a distinct and
                 obvious increase in `power' as characters
                 grow. This gives players the feeling that their
                 characters are constantly improving as the game
                 continues. 
              
              
                The game rules are purposely devoid of
                `fluff' rules. Time-consuming (and therefore
                not fun) look-up tables and thick books of rules are
                nowhere to be found. \APATHY{}  is designed to be fast
                to play. Most of the `tables' in the \APATHY{}  manuals are just to explain things to the players and
                aren't used during play.
                
              
              
                While \APATHY{}  is devoid of many `clutter
                rules', it does have an emphasis on realism.
                \APATHY{}'s combat system is an excellent
                compromise between speed and realism.
                
              
           In any game of \APATHY{}, the CHARACTERS are the
            heroes!
          
\section{Basic Terms}
      
              
              
                 Action 
                
                 An action is the most basic unit of time during
                 combat. It is roughly equal to one second of time. Most
                 characters will have three Actions per turn, though
                 certain things can increase or decrease this number.
                 Note that as long as a character is unconscious, he or
                 she always has at least one Action per turn. 
              
              
              
                 Base Armor 
                
                  A character's natural resistance to bodily
                 harm of unusual varieties. You can think of this as
                 `toughness' (not to be confused with the
                 skill of the same name). A character's base armor
                 reduces the damage they suffer from U (undefined)
                 damage. Base Armor does not protect against any other
                 form of damage, such as P, S, C, or D. Note, that if a
                 character suffers any U damage, no matter how high their
                 Base Armor might be, they always loose at least 1 HP.
                 
              
              
              
                 C 
                
                An abbreviation for 
                  crushing  . When used in conjunction
                  with a weapon, it indicates that the weapon in question
                  deals Crushing type damage. When used in conjunction
                  with armor, it indicates how much Crushing damage the
                  armor will defend against.
                
              
              
               CP   
                  Character Points. These are points you use to
                 `purchase' skills, equipment, etc. for your
                 character. CP can also be used to increase attributes
                 according to a special table. See the section on
                 character creation. 
              
              
               Character   
                  A persona within the game world that is adopted
                 by a player. The player sees through the
                 character's eyes, controls the character's
                 Actions, and speaks through the character's mouth.
                 
              
              
               D   
                  An abbreviation for Direct Damage. D damage is
                 applied directly against the target's HP. Armor,
                 including Base Armor, cannot protect against D damage.
                 Fortunately, D damage sources are few and far between.
                 
              
              
               Game Master   
                  A person who is in the game that is not a player.
                 The Game Master controls the game world that the players
                 inhabit. He/she controls everything that is not the
                 players. This includes other people (sometimes called
                 non-player characters), monsters, natural events, etc.
                 This is usually abbreviated as `GM'. 
              
              
               HP   
                  Abbreviation for Hit Points. Hit Points are used
                 to represent a character's health. If a character
                 gets injured, their HP decreases. If a character reaches
                 0 HP, they fall unconscious. If a character reaches -X
                 HP, where X is equal to their maximum HP, they are dead.
                 Note that if a character is below 0 HP, they will loose
                 1 HP per turn (1 HP per 3 seconds when not in combat)
                 due to bleeding, unless they are given medical
                 attention. 
              
              
               HTH   
                see Hand-to-Hand   
              
              
               Hand-to-Hand   
                  Hand-to-hand is a special type of combat. It
                 takes place when the combatants use no weapons of any
                 kind. Thus, in hand-to-hand combat, only punches, kicks,
                 and similar attacks will take place. 
              
              
               Luck Points   
                  Usually, this is short for Luck Points. Luck
                 Points are accumulated during gameplay, and players
                 redeem them for combat bonuses, or for saving their
                 character's butt if things get rough. This is used
                 to help unLuck Pointsy die rolls from screwing over
                 players. All major character in the game, including
                 significant NPCs, villains, etc. will have Luck Points.
                 The GM is responsible for distributing Luck Points.
                 
              
              
               MA   
                  Abbreviation for Martial Arts. It is sometimes
                 used to refer to the Martial Arts skill. See the Martial
                 Arts section for more information. 
              
              
               MP   
                  Magic Points, also called Mana. Magic points
                 measure how much magic a character can use in a certain
                 amount of time. Each time magic is used (a spell is
                 cast, or the character makes a rune) MP are consumed. If
                 a character has no MP, they cannot use any magic until
                 the MP replenish. MP replenish naturally over time. See
                 the Magic section for details. 
              
              
               Melee   
                  Melee is a special type of combat. Melee combat
                 takes place when there are no projectile (launched,
                 thrown, shot, etc.) weapons or attacks of any kind.
                 Melee combat includes things like hand-to-hand fighting,
                 or using weapons like swords, clubs, etc. 
              
              
               Missile   
                  A missile weapon is one that flies through the
                 air in some manner. Throwing knives, slings, thrown
                 stones, spears, and arrows are examples of missile
                 weapons. 
              
              
               Non-Player Character   
                  A person controlled by the Game Master who
                 interacts with the players. A non-player character is
                 anyone in the game world not controlled by a player.
                 
              
              
               NPC   
                See Non Player Character.   
              
              
               P   
                  An abbreviation for Piercing. When used in
                 conjunction with a weapon, it indicates that the weapon
                 in question deals piercing type damage. When used in
                 conjunction with armor, it indicates how much piercing
                 damage the armor will defend against. 
              
              
               PC   
                  Power Cells. This is a measurement of energy or
                 power for things like mecha, vehicles, robots, and power
                 armor. Power, measured in PC, is produced by things like
                 engines and is consumed by movement systems and weapons,
                 etc. PC can also mean `player character'.
                 
              
              
               Player   
                  An actual person, like the authors of this
                 manualor youwho controls a character in
                 the game world. 
              
              
               Round   
                  A `round' is the most general unit of
                 time during combat. Any sequence of combat is broken
                 down into one or more rounds. During each round, each
                 combatant gets to take one turn. A round is over
                 whenever all combatants have taken a turn (or are unable
                 to for some reason). 
              
              
               S   
                  An abbreviation for Slashing. When used in
                 conjunction with a weapon, it indicates that the weapon
                 in question deals Slashing type damage. When used in
                 conjunction with armor, it indicates how much Slashing
                 damage the armor will defend against. 
              
              
               Size   
                  A representation of the approximate Size of
                 objects such as people, mecha, etc. Humans are Size
                 level 2. Each Size level is approximately double the
                 height and weight of the Size below it. So, Size 2
                 covers things up to about 7' tall. Size 3,
                 therefore, is about 14' tall. 
              
              
               Turn   
                  A turn is a unit of time during combat. Each
                 round of combat is broken down into one turn per
                 combatant. During each player's turn, they get to
                 do whatever they want. A turn is roughly equal to three
                 seconds of time. Each turn is subdivided into Actions.
                 
              
              
               U   
                  An abbreviation for undefined damage. Undefined
                 damage is a special type of damage that cannot be
                 prevented by normal armor. It is usually caused by
                 supernatural powers, magic, high tech weapons, and
                 similar unusual sources. Normal (P/S/C) armor cannot
                 protect against U damage. With very few exceptions, the
                 ONLY thing that can protect a character from U damage is
                 his or her Base Armor. Note, that if a character suffers
                 any U damage, no matter how high their Base Armor might
                 be, they ALWAYS loose 1 HP. 
              
          
\section{Game World}
    \APATHY{}, like most role-playing games, can be
            thought of like a play or a movie. The players are like the
            actors; their characters are the heroes and heroines of the
            story. Like a movie, a game can take place in any number of
            settings. The GM is like the director, and decides the
            intricacies of the game world.
           The GM should talk with the players and decide upon
             what kind of game world they would like to use. The GM can
             then finalize the details of that world, and should tell the
             players about it before the players make their characters
             and the game starts.  Some things to think about when creating the game
             world are the following:   
              
                 Time Period: Medieval? Modern-day? Futuristic?
                 
              
              
                  Location: on Earth? A different planet? A
                 particular city, perhaps? 
              
              
                Level of technology   
              
              
                  Allowed game subsections: Magic? Psionics?
                 Supernatural Powers? etc. 
              
           The \APATHY{}  system contains sections for many
            different types of games and many different historical
            settings. The GM should decide which ones to allow for each
            game.
          
\section{Creating a Character}
     The first step in playing \APATHY{}  is determining
            your character concept. Talk to the GM and your fellow
            players for ideas about the game setting and what kind of
            character you might want to play. Your imagination is the
            only real limit as to what your character can be. Perhaps you
            want to play a heroic knight that fights for honor and puts
            monsters to the sword? Or maybe you'd rather play a
            cunning and seductive thief with a hidden agenda. You could
            be a young mecha pilotinexperienced, but eager to
            fight for his cause. Your character is a hero in the game.
            You should take this opportunity to have fun and do things
            you can't do in real life.
           If you are having trouble thinking up a character
             concept, you might want to look to books, movies, or even
             historical precedent as to what kind of character you want
             to play. You could be a space mercenary like Han Solo from
             Star Wars, or maybe a secret agent like James Bondor
             perhaps an old-west gunslinger.  Once you have an idea so to what kind of character you
             want to play, it's time to develop that idea. Think
             about what your character looks like. How old are they? How
             do they dress? What kinds of possessions do they have? What
             do they do for a living? What was their past? How would they
             react in a fight? What would they do if they were suddenly
             given a lot of money? What are their fears? Their hopes?
             Their goals?  Now you should have a good understanding of your
            character's appearance, personality, skills, etc. What
            you need to do next is represent that idea in the \APATHY{} 
            system rules. Grab a character sheet, and follow along.
          
\subsection{Character Basics}
     The first part of describing your character is very
               simple. Decide what your character's basic
               description is like. Are they male or female? How tall?
               What do they weigh? Fill in age, sex, height, weight, eyes
               color, and hair color/style on the upper right portion of
               your character sheet. Decide on your character's
               name, and fill that in too. There is also a spot on the
               character sheet for your name. Now you can move on to the
               details of your character. 
\subsection{Attributes}
    \APATHY{}  models characters by means of a core
              attribute system. These attributes are numbers that
              describe the most basic characteristics of any given
              character. They are:
              
                
                  
                   Abbr.   
                   Name   
                   Definition   
                  
                  
                   STR   
                   Strength   
                   raw physical strength, muscle power   
                  
                  
                   DEX   
                   Dexterity   
                     quickness, coordination, fine motor control,
                     balance 
                  
                  
                   BLD   
                   Build   
                     the toughness and durability of your body,
                     stamina, `guts' 
                  
                  
                   INT   
                   Intelligence   
                     computational power of the mind, ability to
                     grasp complex problems 
                  
                  
                   KNO   
                   Knowledge   
                     experience and background knowledge, common
                     sense, etc. 
                  
                  
                   WIL   
                   Willpower   
                     strength of the mind, ability to stick
                     it out' when things get tough 
                  
                  
                   CHA   
                   Charisma   
                     friendliness, dealing with people, etc.
                     
                  
                
              Basic Terms  
             Throughout this document the attributes will usually
               be mentioned using the three-letter abbreviation instead
               of the full name.  The first three attributes (STR, DEX, and BLD) are
               considered Physical Attributes. The second three
               attributes (INT, WIS, and WIL) are Mental Attributes. The
               seventh, CHA, is a little of both, but is generally
               considered to be a Mental Attribute.  Attributes are rated on a scale. 1 is the lowest
               possible attribute score. There is no top end on the
               scale, though an attribute of 20 is the peak of human
               ability. A human character can never have an attribute
               higher than 20 without some supernatural effect (super
               power, magic, cybernetics, etc).  When creating a character, the GM allows you a
               certain number of attribute points. Attribute Points are
               used to purchase the attribute values you want your
               character to have. A normal human has about 40 attribute
               points. An above average human would have about
               105140 attribute points. We recommend that 120
               Attribute Points be assigned to starting
               charactersafter all, player characters are supposed
               to be heroicthough the GM has the final word.
                The cost of an attribute is computed like a skill
              with Rank 1 (see the skill section later on). However, the
              final value for the attribute (for a Human) is increased by
              8; that is, if you buy 5 levels of STR (for 15 AP), then
              the final value of the STR is  5  +
               8  =    13 .
             In most cases no player may have an attribute below
               8. Attributes below 8 are considered handicapped and/or
               disabled, and such a character would probably not be
               adventuring!  If the GM deems it OK, then you may reduce an
               attribute below 8, but you get NO benefits (you
               wouldn't get points in another attribute for
               example). Why would you do this? Simple, it's in
               your character concept. Would a 6-year-old child have an 8
               STR? Not likely.  When recording your attributes on the character
               sheet, record the number of Attribute Points you spent on
               that attribute, this will help you when you wish to
               increase them later.  No attribute can be above a 20 for a natural human.
               A 20 is considered the peak of human ability. 11 or so is
               average. Below 8 is considered handicapped for an adult
               human. Note that certain things can increase attributes
               above 20, such as supernatural powers, magic, cybernetics,
               etc.  Character Points (see next section) can be used to
               purchase Attribute Points for increasing your attributes
               later on. 
\subsection{Sub-Attributes}
     There are a number of `sub-attributes'
               that rely on your attributes to determine their value. The
               names, description, and method for determining their value
               are listed below.   
                
                  
                   Abbr.   
                   Definition   
                   Computation   
                  
                  
                   Lift   
                     This determines how much weight the character
                     can lift in pounds, in a `good'
                     situation, such as a weighted barbell or a trunk
                     with handles. 
                           STR\ensuremath{\times}STR\ensuremath{\times}BLD
                        10      +   
                     STR   
                    
                  
                  
                   Carry   
                     This is the amount of weight the character can
                     carry about his/her person for a reasonable length
                     of time. 
                           Lift    2  
                       
                    
                  
                  
                   Encumbrance   
                     This represents how much gear the character
                     can wear before encountering adverse effects due to
                     the weight and impaired mobility. 
                             BLD  +
                      DEX  +  STR  
                        6     
                    
                  
                  
                   Base Armor   
                     Determines how resilient the character is to
                     `odd' damage, such as lightning and
                     fire. 
                           BLD  +
                      WIL    6    
                     
                    
                  
                  
                   Movement   
                     How fast the character is in `squares
                     per move-equivalent Action ' 
                           2  \ensuremath{\times}
                      STR  +  DEX 
                       15     
                    
                  
                  
                   Perception   
                     How well the character can notice things
                     around him. Used just like an attribute check.
                     
                             DEX  +
                        2  \ensuremath{\times}    KNO
                      +    INT      4
                         
                    
                  
                  
                   Interaction   
                     The basic ability of a character to interact
                     with other people. 
                           CHA  +
                      KNO  +  INT 
                       3     
                    
                  
                  
                   Hit Points   
                     The amount of damage you can receive before
                     dying. 
                       BLD  +  WIL
                     
                    
                  
                  
                   Magic Points   
                     The number of Magic Points you can draw on to
                     cast magic spells or perform supernatural feats
                     
                           (    WIL  
                      +    BLD    )   
                     \ensuremath{\times}    10     
                    
                  
                
              Sub-Attributes  
             If the above equations generate any fractions, round
               them to the nearest whole number.  Encumbrance, Base Armor, HP, and MP will be
               discussed in more detail in later sections. Keep this in
               mind as you may need to update them later. 
\subsection{Character Points}
     The remainder of the character creation process is
               done through the expenditure of Character Points,
               otherwise known as CP. Talk with your GM about how many CP
               you should use to create your character with. We have
               found that for a group of 3-4 players 125 CP is about
               right. Typically the more players, the fewer CP characters
               should have. This will keep the group from becoming too
               powerfulunless this is the GM's intent.
                You will use your CP to `purchase' the
               remainder of your character. The rest of your character is
               broken into three categories where you will spend these
               points:   
                
                 Mundane 
                    
                      
                       Attributes   
                        increasing starting attributes   
                      
                      
                       Skills   
                          normal skills, proficiency and knowledge
                         of various topics 
                      
                      
                       Combat Skills   
                          techniques for fighting and defense.
                         
                      
                    
                  
                
                
                 Arcane / Magic 
                  
                    
                       Spellcasting 
                      
                    
                    
                       Psionics 
                      
                    
                    
                       Witchcraft or Ritual Magic
                       
                      
                    
                    
                       Supernatural abilities (super powers)
                       
                      
                    
                  
                  
                
                
                 Technology 
                  
                    
                       Mecha and Robotics 
                       Robotic bodies, limbs, powered armor, and
                       robotic suits 
                    
                    
                       Augmentations 
                       Cybernetics, genetic mutations, and chemical
                       augmentations 
                    
                    
                       Weapons 
                      
                    
                    
                       Armor 
                      
                    
                    
                       Equipment 
                        \& other material possessions 
                    
                  
                  
                
            \APATHY{}  does not have `character
              classes' that are common to many role-playing
              systems. The \APATHY{}  system allows the player to decide
              every detail pertaining to his character, rather than being
              stuck with a `default' set of abilities and
              skills. It is up to the player to use his CP to develop a
              well-rounded character. Note that there are NO skills that
              come `free.' If your character is literate,
              then that skill needs to be purchased. Know how to drive a
              car? Buy the skill.
            
\subsection{Increasing Attributes (optional)}
     If your attributes aren't quite where you
               desire, you will need to get more Attribute Points so you
               can purchase a better attribute. Each Attribute Point
               costs 1 Character Point. Once you purchase the number of
               Attribute Points you want, just look at the table found in
               the Attributes section above to see how many total
               Attribute Points you must spend to get that attribute.
                You must pay a number of Attribute Points equal to
              the difference between what you want, and what you already
              have. For example: if you currently have an 11 STR (6
              Attribute Points) and wish to increase it to a 14 (21
              Attribute Points) that will cost you 15 Attribute Points (
               9  -  4  =
               5  ). Since Attribute Points cost
              1 Character Point each, going from an 11 in an attribute to
              a 14 will cost 15 CP.
             Remember to keep track of how many Attribute Points
               you have spent on your attributes so you can easily
               increase them at a later date.  Note that attribute increases are major additions to
               your character, and they should be role-played as such.
               Significant attribute changes represent months if not
               years of hard work! Here are some examples of possible
               means of increasing attributes:   
                
                 STR   
                    Strength raw physical strength. Muscle power.
                   
                
                
                 DEX   
                   Dexterity quickness, coordination, fine motor
                   control, balance 
                
                
                 BLD   
                   Build the toughness and durability of your body,
                   stamina, `guts' 
                
                
                 INT   
                   Intelligence computational power of the mind,
                   ability to grasp complex problems 
                
                
                 WIS   
                   Wisdom experience, worldliness, background
                   knowledge, and common sense 
                
                
                 WIL   
                   Willpower strength of the mind, ability to
                   stick it out' when things get tough
                   
                
                
                 CHA   
                   Charisma appearance, personal presence, ability
                   to make friends or influence others 
                
            
\section{Skills}
     All characters have skills. Skills represent what your
             character has learned and what they know how to do. Other
             than being useful in game, skills help to distinguish your
             character from everyone else in the world. Are you going to
             play a diplomat? You better have diplomacy, law, and similar
             skills. A warrior? Better have that shield and/or weapon
             skills handy!  Skills have a number of pieces of information
             associated with them. They have a Rank, Associated
             Attribute, Description, and Level.  The rank determines how hard a skill is to learn.
             Higher the rank, the more time and CP it will take to get
             better at it. Rank 1 skills are the simplest and easiest to
             learn. Skills that are more difficult have higher ranks.
             Rank can also be thought of as the `cost' or
             `price' of a skill.  An associated attribute tells you what attribute is
             used in the skill roll. This is the attribute that is used
             along with any particular skill. This is discussed in more
             detail below.  The description will let you know the scope of Actions
             that a skill encompasses. The things listed are not the only
             things available under that skill, but just some of the more
             common applications of the skill. If you have a question,
             ask your GM about his take on a particular skill.  Level is how good your particular character is at the
             skill. You will use CP to purchase levels in a skill, which
             is discussed below. If you don't know a skill at all,
             you have Level 0 in that skillmeaning you don't
             know anything about it. The more you know about the skill,
             the higher your level. This is often abbreviated as
             `Lvl' (alt. `LVL'). You must
             purchase your skill levels with CP. Below is a simple chart
             indicating what different levels correspond to:   
              
                
                 Level   
                 What it means   
                
                
                 0   
                 No knowledge of the skill.   
                
                
                 1   
                 Amateur. You know something about it.   
                
                
                 2   
                 Had some training or experience   
                
                
                 3   
                   Beginning professional or serious hobbyist
                   
                
                
                 4   
                 Professional   
                
                
                 5   
                 Experienced professional   
                
                
                 6   
                   Exceptionally good. Among the best in the
                   country. 
                
                
                 7   
                 Among the best in the world.   
                
                
                 8+   
                 Among the best ever   
                
              
              Skill Level and Descriptive Equivalents
           Like most of character creation, skills are purchased
             using CP. The higher the rank, the more expensive it will
             be. The higher the level you want in the skill, the more
             expensive it will be.  For those that are mathematically inclined, here is
             the formula for determining CP of a particular skill level
             and rank:     
                   ∑    i
                    =  0    N  
                      i    =   
                    
                     N  2 
                       +    N     
                   2        For the rest of us, use this nifty table, and multiply
             the value listed by the rank.   
              
                
                 Desired Level   
                 1   
                 2   
                 3   
                 4   
                 5   
                 6   
                 7   
                 8   
                 9   
                 10   
                
                
                 CP Cost for Rank 1   
                 1   
                 3   
                 6   
                 10   
                 15   
                 21   
                 28   
                 36   
                 45   
                 55   
                
              
            CP cost of Desired Level, precomputed  
           See the trend?  If not, here's yet another way to look at
             purchasing skills.  To get level 1 of a skill, you must spend the Rank in
             CP. To get to the next level, add the rank of the skill to
             the amount of CP you spent the last time you increase the
             skill (and just that last increase, not the total amount
             you've spent on the skill) and that is the amount that
             you need to spend. 
\quotexample[Computing Skill Cost]{ I want to purchase the shield skill (rank 2). To get
               Lvl 1 I would need to spend 2 CP. Now, if I want to go
               from Lvl 1, to Lvl 2, I would add the rank (2) to the
               amount I spent on the last level increase (2 CP), to get
               the new cost, in this case I would need to spend a total
               of 4 CP (rank of 2 + 2 CP from last purchase). So now the
               total amount I have spent on this skill is 6 CP. Lets say
               I'm not happy with my Lvl 2 shield skill, and wish
               to upgrade it to level 3. I would again take the rank (2)
               and add it to the amount I spent on the last increase (4)
               to get the number of CP I must spend to upgrade the skill.
               In this case, I must spend an additional 6 CP. Therefore,
               to get a level 3 shield skill (or any skill or rank 2), I
               must spend a total of 12 CP. }
   If you take a look at the little chart above, you will
             see that at the desired skill level (3) the CP cost is a 6,
             which I then multiply by the rank (2) to get 12 CP, the same
             thing I got when I did it out by hand.  Note that the chart will only tell you how many total
             Character Points. To get the number of CP needed to increase
             to the next level, just subtract the upper number from the
             lower one. A more complete version of the above table can be
             found at the end of this manual.  Once you are in game and earning experience, you can
             increase the level of a skill you currently have at least
             one level in by paying the appropriate CP cost. If you wish
             to learn the first level of a skill you don't have
             yet, it takes a number of weeks of intense studying equal to
             the rank of the skill. 
\subsection{Skill Specialization (optional)}
    
\subsubsection{Specialization in most Skills}
     It is possible to specialize in many skills. When
                 specialized, a character knows more about a certain
                 aspect of the skill than other parts of the skill. In
                 game terms, you are +1 LVL when performing a check in
                 the area of your specialization, but are -4 to your roll
                 when you are performing a check outside of your
                 specialization. Specializing does NOT cost any extra CP.
                 
\quotexample[History$\rightarrow$Spanish]{ Ralf Gr--sse-Kunstleve is a historian.
                   Specifically, he is knowledgeable about Spanish
                   history. He has a LVL 3 History skill with a
                   specialization in Spanish History. This means that if
                   Juan had to make a skill check related to Spanish
                   history, he would roll 4 dice. (3 from his LVL, +1
                   from the specialty.) However, if he had to make a
                   History roll about another kind of history, he would
                   only roll 3 dice, and be at a -4 to his roll. }
   Note that you can choose to buy a particular skill
                 more than once with different specialties, if you feel
                 like it. When this happens the roll for `outside
                 of specialty' skill checks is the highest of the
                 skill levels -4 +1 for each skill beyond the first.
                 
\quotexample[Chemistry$\rightarrow$WMD]{ Dr. Klotzenstein is a scientist. He has LVL 3
                  Chemistry with a specialty in Chemical weapons and
                  poisons. As well, he has LVL 4 Chemistry with a
                  specialization in the chemistry of explosives. Dr.
                  Klotzenstein's rolls are as follows:
                    
                    
                       For Weapons/poisons: 4 dice 
                      
                    
                       For Explosives: 5 dice 
                      
                    
                       For Non-specialty rolls, 5 dice -3. 
                      
                    
                }
   It is possible to specialize in most skills. Talk
                 to your GM about specializing to see which skills you
                 think are appropriate and which ones are not.  Specializing in skills is purely optional. 
\subsubsection{Specialization applied to Weapon skills}
     It is also possible to specialize in weapon
                 skills. If a character has a specialized weapon skill
                 that means that he is especially proficient with a very
                 specific type of weapon. When wielding a weapon that
                 matches a character's specialty, that character
                 receives +1 LVL to the skill. This bonus applies to
                 attack (strike) rolls, and defensive rolls (parry), if
                 applicable. When using a weapon that is outside the
                 character's specialty, the character is at -1 LVL.
                 Note that Weapon skill specialization does NOT affect
                 damage rolls.  Specializing in a weapons skill costs 3 CP in
                 addition to the normal cost of the skill. Specializing
                 can be applied to all weapon proficiency type skills and
                 the shield skill. 
\quotexample[Axes$\rightarrow$Nordic War Axe]{ Bj---rgen is a warrior with a liking for
                   the Nordic War Axe. He has the skill `Archaic
                   Weapons: Axes' at level 3. As well, he is
                   specialized in the Nordic War Axe. The total skill
                   cost is 9 CP. That is 6 CP for Level 3 in a Rank 1
                   skill, Plus 3 CP from the specialization. When welding
                   a Genuine Nordic War Axe$^{TM}$ in combat, Gunnar has
                   4 dice. However, when using any other kind of axe
                   Gunnar only has 2 dice. }
   Note that some unusual or esoteric weapons require
                 specialization. This is because those weapons are
                 different enough from common weapons that they require
                 special knowledge and experience in order to use them
                 properly.  Most characters that specialize in weapons are
                 very particular about their weapons. For the character
                 it is almost an obsessive-compulsive issue to wield a
                 particular weapon. Generally the weapon must be very
                 specific in order to satisfy the specialty. In medieval
                 campaigns, a weapon specialty might require a sword
                 forged by a particular smith, or arrows made from a
                 certain type of wood. In a modern campaign, a weapon
                 specialty might entail a particular make and model of
                 rifle.  Specializing in skills is purely optional. 
\section{Rules (Rolls)}
     What would an RPG be without rules? \APATHY{}  has a
            set of rules that allow for just about any occurrence that an
            imaginative GM can put the players through.
           Lets start with the basics (some of this will be
             review):  In \APATHY{}, there are three different types of die
            rolls. Every time you roll a die when playing \APATHY{},
            you will use one of these sets of rules.
          
\subsection{Attribute Checks}
     Attribute checks are required anytime the GM wants
               to see if you did something solely based on an attribute.
               They are used whenever a player wants their character to
               do something tricky. Attribute checks are purely based on
               a character's core attributes. They are completely
               separate from skills. If a player needs to do something
               based on a particular skill, then see the skill roll
               section below. 
\quotexample[Catch a Pack]{ Barnacleez is trying to catch a pack that was
                 thrown at him. The GM requires him to make a DEX check
                 to see if he successfully caught the pack. }
  
\quotexample[Swim a Long Distance]{ Thomas is trying to swim a long distance while
                 carrying a heavy load. He has to roll a BLD check to see
                 if he has enough stamina to make the swim. }
  
\quotexample[Decipher Symbols]{ Wendy is trying to decipher the meaning of some
                 strange symbols written on a wall. Since she has no
                 applicable skills, the GM determines that she needs to
                 make an INT check in order to figure out what they mean.
                  Attribute checks are always rolled the same. Roll
                 ONE twenty-sided die, add the attribute to the die, and
                 add any bonuses or negatives. If the total (die roll +
                 attribute + bonuses + negatives) is 25 or higher, then
                 the check succeeded. If you roll a 1, you fail
                 automatically, no matter what your bonuses or attributes
                 are. If you roll a 20, you succeed automatically, no
                 matter what the negatives might be. For an attribute
                 check, you always roll just one 20-sided die. }
  
\subsection{Contest of Attributes}
     Contests of Attributes occur whenever two characters
               are competing at something that is largely dependent on an
               attribute and nothing else. Arm wrestling is a contest of
               strength. Racing to complete a puzzle is a contest of
               intelligence. A trivia game would be a contest of wisdom,
               etc.  A contest of attributes occurs only when there are
               multiple competitors (generally two) and only when there
               is no particular skill involved. For example, arm
               wrestling is a contest of strength but a game of 1-on-1
               basketball is generally not, because basketball is a
               complex game.  When a contest of attributes occurs, all competitors
               roll 14and add the result to their attribute. The
               highest score wins, and like scores tie. The GM may wish
               to impose negatives and/or bonuses depending on the exact
               circumstances of the contest. 
\quotexample[Douglas vs. Enemy Soldier]{ Douglas (STR 13) is fighting with an enemy solider
                 (STR 12) over the possession of a single rifle. The GM
                 declares that a contest of strength will resolve who
                 manages to gain possession of the rifle. The GM rolls
                 for the solider (a three) and Douglas rolls a 1. The
                 enemy soldier has a total of 15, and Douglas has 14.
                 Therefore, the enemy soldier wrestles the rifle out of
                 Douglas's hands. }
  
\subsection{Skill Rolls}
     Skill rollsare used whenever a player wants to use
               one of his character's skills. They are also used
               whenever two characters are competing at something
               (Contest of skills) related to a skill. Skill rolls are
               also used `to strike' or `to
               defend' in combat.  To determine if you are successful in performing a
               skill, roll a number of D20s equal to the Level you have
               purchased the skill in. To each die, add your current
               attribute referenced by the associated attribute value of
               the skill. Each time a die with this value added to it
               reaches 25 or above, this is called a success. If you have
               at least one success, you successfully performed the
               skill. Rolling a `20' on a die will give you
               +1 successes to however many you rolled normally. Note
               that this makes it possible to have more successes than
               you have levels in a given skill. Likewise, a roll of
               `1' is -1 successes to your total. If you have
               negative successes (caused by rolling more 1s than
               successes) then you have a critical failure! In the case
               of a critical failure the GM will determine what
               (negative) outcome happens!   
                
                  
                   Roll   
                   Sample Outcome   
                  
                  
                   Strike (sword)   
                   Sword breaks or is dropped   
                  
                  
                   Strike (gun)   
                   gun jams or misfires   
                  
                  
                   Parry   
                     you drop your weapon and are struck by the
                     incoming blow 
                  
                  
                   Diplomacy   
                     you make a faux pas that offends the parties
                     involved 
                  
                  
                   Deception   
                     you are caught `red handed' in a
                     blatant lie 
                  
                  
                   Medicine   
                     you make an outright mistake, harming the
                     patient. 
                  
                
              Sample outcome for Skill Rolls  
             The GM will give penalties and/or bonuses to your
               skill roll based on the difficulty of the Action you are
               performing and what circumstances may be surrounding it.
               When such a negative or bonus is given, this is subtracted
               from the associated attribute value, not the number of
               dice rolled for a skill.  Basically stated, skill rolls work like this: Roll
               120for each level you have in the skill. To each die,
               add your associated attribute and any negatives or bonuses
               the GM gives you. If the total on each die is 25 or
               greater that is a success. If the die rolls a 20,
               regardless of bonuses, attributes, etc. you get +1
               successes. If a die rolls a 1, regardless of bonuses, etc.
               you get -1 success.  Note that the Maximum you can ever add to your die
               rollno matter how high your attribute or bonuses to
               succeed might beis 20.  Also, if you are doing something simple with respect
               to your skill level then you don't have to roll dice
               at all. It's assumed that you succeed examples.
               
\quotexample[Lookup Internt Address]{ Amy has the Computer Use skill at level 3. She
                 wants to look up an Internet address on the computer
                 that she thinks might give the players some needed
                 information. Since this is a simple task and she has a
                 significant level in the associated skill (Computer
                 Use), then the GM rules that she does so automatically,
                 with no roll needed. }
  
\quotexample[Investigate Mysterious Person]{ Amy is still on the Internet. Now she is
                 attempting to dig up some information about a mysterious
                 person. The GM determines that Amy needs to use her
                 Level 2 Investigation skill. Amy has a 13 WIS, the
                 associated attribute for Investigation. The GM tells her
                 that this is a normal difficulty skill, and there is no
                 bonus or negative. So Amy rolls 220for the skill and
                 rolls a 6 and a 13. After adding her WIS to each roll,
                 she has a 19 and a 26. Only the 26 is a success since it
                 is 25 or above. Since she has at least one success, Amy
                 is able to find the information she was looking for.
                 }
  
\quotexample[Fix a Beater]{ Chris is trying to fix an old car so he and his
                 fellow characters can use it. The GM determines that
                 this falls under the Mechanic: General skill. Chris has
                 a level 2 in that skill. Furthermore, the GM determines
                 that since the car is in really bad shape, that Chris
                 has a -3 to his roll. Chris has a 12 DEX. Chris rolls
                 his dice and gets a 1 and a 16. (Chris is rolling two
                 dice because he has a level 2 skill) Adding his DEX to
                 those rolls, Chris now has a 13 and a 28. The 13
                 isn't 25 or above, so it fails, but the 28
                 succeeds. So far, Chris has 1 success. However, the
                 `1' Chris rolled is -1 success. So, this
                 gives Chris a total of 0 successesmeaning the car
                 didn't get fixed. }
  
\quotexample[Shoot a Large Robot]{ Billy is a cybernetic gunfighter with a 22 DEX. He
                 is trying to shoot an enemy robot. The GM determines
                 that Billy gets a bonus of +3 to shoot the robot because
                 it's so large. Normally, Billy would roll his dice
                 and add +25 to each one: +22 from his DEX and +3 from
                 the Size bonus. However, this gets capped at +20. Thus,
                 Billy only gets to add +20 to each of his dice. }
  
\quotexample[Called Shot to Arm]{ Billy is still fighting the same giant robot. This
                 time, he's trying to make a called shot and shoot
                 the robot in the arm. The GM determines that Billy gets
                 a +3 bonus due to the robot's Size, but also a -4
                 to strike because of the called shot to the arm. Thus,
                 Billy adds up his bonuses: +22 from DEX +3 from Size and
                 -4 for the called shot. This adds up to +21, but the
                 bonus is capped at +20. Again, Billy adds +20 to his
                 dice. Notice that the `+20 cap' is applied
                 AFTER all the bonuses and negatives are added up.
                 }
  
\subsection{Contest of Skills}
     A contest of skills occurs when two people are
               trying to defeat each other in some manner with their
               skills. If two characters were competing to see who could
               sing better, the character with the largest number of
               successes of the singing skill would win. The most common
               example of this is combat. In order to be successful in a
               combat situationsuch as attempting to hit an
               opponent with a sword while he is blocking with a shield,
               the initiator (or attacker) of the Action must make more
               successes than the defender. 
\quotexample[Tom and the Shopkeeper in a Contest]{ Tom is attempting to purchase an expensive suit of
                 armor, and doesn't want to pay full price. So he
                 is attempting to bring the price down by using the
                 Bargaining skill. The shopkeeper also has the Bargaining
                 skill, so a contest of skills ensues. Tom has a skill
                 level of 3, and a WIS of 13 (the associated attribute
                 for Bargaining); the shopkeeper has a level 4 Bargaining
                 (he's been at it longer than Tom has), and a WIS
                 of 10.   Tom rolls 3 dice and gets a `14',
                 `20', and a `3'. Tom now adds
                 his wisdom (13) to each of these. The GM didn't
                 give any negatives or bonuses, so that is all that gets
                 added. After adding, the totals are 27, 33, and 16. Two
                 of these are 25 or higher, so Tom has two successes so
                 far. Note that Tom rolled one 20which gives him
                 an additional success. Tom has a total of three
                 successes.   The shopkeeper rolls four dice (remember, he has
                 a level 4 Bargaining), and gets a `1',
                 `12', `15'', and an
                 `18'. After adding the shopkeeper's
                 WIS of 10, we get the following rolls: `11',
                 `22', `25', and
                 `28'. The 25 and 28 are both successes, but
                 the `1' is a -1 success, so the shopkeeper
                 has 1 total success.   Tom has 3 successes, and the shopkeeper has 1.
                 Since Tom initiated the bargaining, and has more
                 successes than the `defender', he is
                 successful in his attempt to reduce the price of the
                 armor. How much is determined by the GM, but it should
                 be commensurate with the number of successes that Tom
                 beat the shopkeeper by. }
   The winner of a contest of skills is whomever rolled
               the highest number of successes. The margin of success is
               based on the difference between the two rolls. For
               example, if the `winner' had only 1 more
               success than the `loser', then the contest was
               close. However, if the `winner' had three more
               successes than the loser, then it was a decisive victory!
                Ties are left up to the GM to resolve. The suggested
               protocol is as follows:  If the tie occurred on a contest that entails a
               single past event or is something that is
               `judged' by outsiders, then the winner is the
               character with the higher associated attribute. (Example:
               who can paint a better picture, etc.)  If the contest entails an ongoing event (such as the
               haggling over a suit of armor detailed above) then it is
               assumed that the competitors `keep at it' and
               a second roll is in order.  The rules for combat are slightly more complicated;
               see below. 
\subsection{Skill Rolls and Related Skills}
     If a character needs to make a skill roll and that
               character knows other skills that are somehow related to
               (or are similar to) the skill to be rolled, that character
               is generally entitled to a bonus to his or her roll. The
               exact terms of the bonus, including what skills qualify as
               `related', is determined by the GM on a
               case-by-case basis. The general rule is that each level of
               a related skill conveys a +1 to the roll. The maximum
               possible bonus from related skills is +5, regardless of
               the circumstances. 
\quotexample[Paul repairing a device]{ Paul is attempting to repair an electronic device.
                 The GM asks Paul to roll his Electronics skill, which
                 Paul has at lvl 4. Paul mentions that he also knows the
                 skills Electrical Engineering: 2, Electrician: 3,
                 Mechanic: General: 3, and Communications/Telco: 3. The
                 GM determines that since this device is purely
                 electronic, Paul's Mechanic skill is of no use. As
                 well, the GM decides that since this device is unrelated
                 to communications, Paul's Communications skill
                 does not apply either. However, the GM decides that
                 Paul's Electrical Engineering and Electrician
                 skills are relevant. The GM assigns Paul a bonus of +5
                 to his roll from his levels in those related skills.
                 }
   Related Skill bonuses NEVER apply to combat skills
               or any skill used in a combat situation. 
\subsection{All Other Rolls}
     This type of die rolling is used for just about
               everything else in the game: rolling damage from combat,
               the duration of magical spells, or the number of monsters
               appearing in a group.  Rolling dice is represented in the
                ±RXY±ZK that is,
                ±RXY±ZK format; for
               example 34+1. This means that you roll three individual
               4-sided dice, and add 1 to each die.  If any die roll with its bonus added in equals or
               betters the maximum value on a die, then you get a
               rollover. When a rollover happens, re-roll that die,
               without any bonus, and add its value to the total. If the
               roll again has the highest value on the die, then another
               rollover happens. This keeps going until the highest value
               on the die is not rolled. Remember, after the first roll,
               the bonus is not applied; it's just the raw die
               roll.  This may sound confusing, so we'll work with
               the example from above. Let's say you have to roll
               34+1. You get some four-sided dice and roll three of
               them. Suppose the results are: 1, 3, and a 4. After adding
               the +1 (from the die bonus), we now have: 2, 4, and 5. We
               add all the totals up and get 11. Now we have to deal with
               the rollovers. Because the highest value on a four-sided
               dice is a 4, we have two rollovers: the 4 and 5. So we
               roll each of these again, without the original bonus, and
               get a 2 and 4. We add these to the total yielding 17. But
               wait, we have yet another rollover (the 4). So we roll
               that die again, and get a 3. We now have a total of a 20
               for our 34+1die roll.  The same rules apply for any die. Lets say you are
               rolling six-sided dice. If any die (after bonuses) has a 6
               or higher, then you get a rollover. On an eight-sided die?
               You guessed it; a roll of 8 or higher will yield a
               rollover.  If you have bonuses or negatives to a roll, add them
               up first, then perform the final roll.  Note that if there is a minus to the die roll once
               all the bonuses/penalties are added up, then no rollover
               is possible. An example is 14-1. Even if you roll a 4 on
               the die it will be reduced to a 3, which does not yield a
               rollover. No die (even after negatives) can give a value
               less than 1. So, if you have a 16-4, and roll a 2 (which
               would theoretically total -2), then the actual total would
               be 1.  Along with what is given above, die rolls might have
               a raw bonus. This is represented by a bonus in brackets
               such as this: 34+1+4(alt. 434+1). Raw Bonuses are
               added to the resultant total of a die roll, NOT to each
               die; thus they do not affect rollovers at all. This bonus
               is added in after all the rollovers are done with.
               Consider the prior example of 34+1but with the [+4]
               added. That would be written 34+1+4. The total using
               the above numbers would be 24.  Unless otherwise stated, all non-skill die rolls use
               this system. 
\section{Combat Sequence \& Initiative}
     Combat is based on a system of Rounds, Turns, and
             Actions. A `round' lasts about five seconds. It
             is the length of time in which all combatants have done
             something. Each round is broken down into Actions. An Action
             lasts about 1 second. Each combatant gets one or more
             Actions during each round. When all combatants have used up
             their Actions, the current round is over and a new one
             starts.  During your turn, you determine what you wish to do
             with your available Actions. Almost everything that you do
             of significance during your turn consumes an Action Simple
             examples of what can be done in an Action are shooting a
             gun, swing a sword, moving, or using a mystical ability.
             Keep in mind that a lot of `boring' combat stuff
             also takes up Actions: reloading your weapon, moving around,
             ducking behind cover, or taking an item out of your pack.
             Some things that you will want to do will consume multiple
             Actions per turn.  All characters, by default, have three Actions per
             round. Certain skills or in-game effects might change this.
             When combat is going to start, all players need to figure
             out how many Actions they get. (typically, this will be 3
             each).  At the beginning of combat, all characters must roll
             initiative. This determines who attacks first in the turn.
             When combat starts, participants must roll a 120for their
             initiative. If any characters have more than 3 Actions per
             turn (due to a skill or magical effect or somesuch), then
             those characters must roll initiative twice.  Various bonuses or penalties apply to initiative
             rolls. Certain skills and magical effects can alter
             initiative. As well, the GM may apply initiative bonuses or
             penalties depending on game situations, such as surprise
             attacks or ambushes.  Most characters will have 3 Actions per turn. All
             three Actions are taken at the same time, on that
             character's initiative. 
\quotexample[Tom's Turn in Combat]{ Tom is in combat. Tom has three Actions and he
               rolled an 11 as his initiative. When it's
               Tom's turn in combat (11), he takes all three
               Actions at that time. }
   The character that has the highest initiative goes
             first, and then the character with the next highest
             initiative takes their turn, and so forth down the line
             until all combatants have taken a turn.  Once all turns are over, the current round of combat
             is over. It then restarts with everyone's initiative
             being the same as it was in the previous round. This
             progression continues until combat is over. If the
             combatants change significantly during combatfor
             example, if new characters show up on the scenethe GM
             might want to make everyone re-roll their initiative. If
             something happens during combat that changes how many
             Actions the combatants have, the GM should ask that
             initiative be re-rolled.  Some things that you might want to do will take up
             multiple Actions. A common example of this is spell casting,
             where many spells take several Actions to cast. If you do
             this, you may need to drag this type of Action across
             multiple Actions or even multiple turns. For example,
             let's say that you want to cast a spell that takes 3
             Actions. You can use your first Action to start casting, and
             then use your second and third Actions to finish casting the
             spell. At any time you may cancel (stop casting) the spell
             or ability with no penalty. When an ability or activity
             requires multiple Actions, it always takes place on the
             final Action that completes that activity. Also, it is
             possible to be disturbed or disrupted if you are attacked
             while performing a multiple-action activity. The GM should
             take note of this and might impose skill checks or other
             chances of failure if you get knocked around too much.
              Note that everything you may wish to do in combat
             requires one or more Actions. This includes moving around,
             drawing a weapon, using items, etc. Undoubtedly, many of
             your Actions will be used for things other than attacking.
              When in combat, some characters may want to delay
             their Action until later. A character can always choose to
             `act later' in combat. When doing so the player
             states what new initiative number he wishes to act on. The
             new initiative must be lower than his current initiative
             number. Once a character's initiative is delayed it
             remains delayed (though it could be delayed even further).
             That characters initiative remains at the new, lower, number
             for the remainder of combat, unless the GM asks all players
             to re-roll initiative. Actions cannot be split between
             different initiative numbers. So, a character must either
             take all his Actions at his normal initiative, or he must
             delay all his Actions to a later initiative. 
\quotexample[Tom's initiative]{ Tom's initiative roll is an 18. He and his
               fellow players were cornered by ogres for whom the GM has
               secretly rolled initiative. Tom has the highest initiative
               roll, so he goes first. Tom isn't exactly sure what
               to do, so he delays his initiative until `10'.
               The GM then resumes combat as normal, with the next
               highest initiative going first. Tom has initiative 10 for
               the remainder of combat. }
   Some game effects last a certain number of turns, or
             have a certain effect each turn. These effects are counted
             and take place on the affected character's turn. They
             only are counted and occur if the character begins that turn
             with the effect in place. These effects occur at the
             beginning of that character's turn in a given round.
             These effects are automatic and do not consume any time from
             the character. Resolve all effects before that character
             takes any Actions. 
\quotexample[Lucy's turn in combat]{ It is Lucy's turn in combat and she casts
               Regeneration on herself. The effect of the spell is that
               each turn she gains 14+1Hit Points. Now the spell has
               taken effect, but she does not gain any HP yet, nor does
               this turn count against the duration of the spell. Now all
               other combatants take their turns. When it is again
               Lucy's turn (next round) she gains her 14+1HP,
               just before her first Action This is also considered one
               turn out of the spell's duration. }
   It is possible to `hold an Action ' for
             later. Holding an Action means that the character is
             maintaining his current initiative number, but is saving an
             Action for a specific situation. A character may only hold
             one Action at a time. Furthermore, you must declare what
             your held Action will be, and under what circumstances. Held
             Actions are resolved as soon as their conditions are met.
             Held Actions `expire' if their declared
             conditions never occur at the beginning of your next turn.
             
\quotexample[Tom holds an Action]{ Tom's initiative is 10. During his turn he
               spends one Action to move behind some cover and another to
               ready his longbow. Tom declares that he is going to hold
               his final Action `I am going to cover the hallway
               with my bow. If I see a goblin come out of the hallway,
               I'm going to shoot at it'. If a goblin does
               indeed come out of the hallway (and Tom can see this),
               then Tom may immediately take his held Action to fire an
               arrow at it. However, if no goblin comes down the hallway
               before Tom's next turn, then the held Action is
               wasted. }
   Notes for held Actions:   
              
                You may only hold one Action at a time   
              
              
                  You must state specifically what the conditions
                 for your held Action must be, and what your Action will
                 be once those conditions are met. 
              
              
                  Your held Action can be any activity you could do
                 in one Action (including attacking, moving, using a
                 magical ability, etc.) 
              
              
                  Held Actions are lost if their triggering
                 conditions never occur. 
              
          
\subsection{Weapons \& Equipment}
     Most characters will carry equipment in combat. This
               usually includes weapons and armor. There are no
               `class' restrictions for weapons or armor. Any
               character can wear any armor and wield any weapon,
               provided that the character has the appropriate attributes
               and skills.  Weapons have a minimum strength requirement (MSR).
               The MSR indicates what strength is required of the
               character to properly wield a given weapon. If the
               character's strength is less than the MSR, then the
               character has -1 to strike and [-1] damage for each point
               of strength difference.  The raw damage bonus (RDB) is calculated by
               subtracting 8 from the wielder's Str. The RDB is
               added as raw damage to total of the weapon's damage.
               However, all weapons have a maximum strength bonus (MSB);
               if the RDB is greater than the MSB, then the MSB is used
               as the RDB instead. 
\quotexample[MSR for Bosk]{ Bosk has a STR of 17 and is brandishing a Hercules
                Club (36C, MSR: 12, MSB: 13). Because Bosk's Str
                is higher than the minimum strength requirement of the
                Hercules Club (12), he is entitled to a damage bonus.
                Subtract 8 from Bosk's Str score to calculate his
                bonus:  17    -  8
                 =  9  . So this means
                that Bosk deals 936C damage.
              }
  
\quotexample[Bosk with Knife]{ Suppose Bosk pulls out a long knife (16+1S, MSR:
                 8; MSB: 4). Bosk's Str is significantly higher
                 than the minimum required for the weapon. Calculating
                 his theoretical bonus, we can write: 17 - 8 = 9. At
                 first glance, it appears that Bosk is entitled to a [+9]
                 bonus. However, the maximum strength bonus for the long
                 knife is 4. Therefore, Bosk's damage roll would be
                 416+1. }
   Bows never grant a damage bonus to characters with
               higher than required STR. However, it is possible to
               purchase stiffer (stronger) bows that can take advantage
               of a character's higher STR score. High-draw bows
               cost 20\% more for every 1 STR point added to the MSR. For
               each point, add [+1] damage. This can be stacked. So, a +1
               bow costs 20\% extra, a +2 bow costs 40\% extra, etc. Note
               that the maximum STR requirement that can be added to a
               bow of a given type is equal to half of its original MSR.
               Note that the damage and strike penalty for having a
               lower-than-required STR does apply to bows. Modern
               compound bows are adjustable and can be set to a higher
               than normal strength value without incurring additional
               cost.  Characters will have two hands that they may use to
               wield weapons. All weapons are either one-hand or
               two-handed. Two-handed weapons, of course, require both
               hands to properly wield. One-handed weapons require only
               one hand. Therefore a player could wield one single-hand
               weapon, one two-handed weapon, two single-hand weapons, or
               one single-hand weapon and a shield or other item.
               However, when wielding two weapons the character can have
               penalties. Also, wielding two weapons does not grant the
               ability to the character to strike with both weapons in
               one Action In general, wielding two weapons gives the user
               -3 and -6 to strike with the primary and off -hand
               (respectively), and requires an additional Action These
               penalties can be mitigated with skills from Martial Arts.
                Characters that use magic, aside from Martial Magic,
               must have at least one hand free when casting spells.
               Characters that use Faith magic must have a holy symbol in
               their grasp when casting magic; this requires a free hand
               to hold. Spellcasters are free to use two-handed weapons
               so long as they aren't wielding the weapon at the
               same time as they are casting spells. (Spellcasters can
               put away a weapon, cast magic freely, and then draw their
               weapon again, though this takes time)  All firearms are assumed to be 2-handed by default.
               However, pistols and submachine guns may be fired with one
               hand. (See rules for using 2-handed weapons with one hand
               below). Larger firearms including rifles, shotguns, and
               heavy weapons (rocket launchers, etc.) must always be used
               2-handed. It always requires two hands to reload a
               firearm, no matter what kind it is.  Bows are always two-handed weapons. Hand (small)
               crossbows can be aimed and fired with one hand. However,
               cocking and re-loading a crossbow always requires two
               hands. Larger crossbows always require two hands to fire
               as well as load. Siege weapons, cannons, heavy weapons,
               etc, always require both hands to be free in order to
               load, aim, and fire.  It is possible to use two hands to fight with a
               one-handed weapon. In this case, the wielder has -1 to
               strike and effectively adds +3 to his strength.  It is possible to use some two-handed weapons with
               one hand. To do so, add 4 to the minimum strength
               requirement of the weapon. Note that many two-handed
               weapons always require two hands, period. These are large
               firearms (rifles, shotguns), staves, pole arms, chains,
               and bows. 
\subsection{Dual-wielding Weapons or Weapon \& Shield}
     As noted above, it is possible for characters to
               wield two weapons in combat, or a weapon and a shield. A
               character cannot attack with both weapons during the same
               Action unless he takes the Multi-weapon style skill in
               Martial Arts. A character using a weapon and a shield may
               not attack with his weapon and shield bash in the same
               Action for the same reason. When attacking with both
               weapons the character makes two separate attack rolls, one
               for each weapon; each attack is resolved separately. This
               takes two separate Actions.   
                
                 Two normal weapons   
                   When wielding two weapons the weapon in the
                  character's `good' hand receives a -3
                  to strike and the weapon in the character's
                  `off' hand receives a -6. Some characters
                  can have `even-more-offhand' weapons, such
                  as a kick (or two), or literally having extra limbs. To
                  calculate the negative to strike for wielding multiple
                  weapons, first find the number of weapons being used,
                  then order them (the strike order). The first weapon is
                  at -3 to strike, the second at -6, the third ad -9, and
                  so on, i.e. the k
                   th   weapon is at -3*k to strike. 
                  
                
                
                 Paired weapons   
                  , or one normal weapon and one backup weapon:
                  Some weapons are well suited to, or even designed for,
                  use in the `off' hand. When using these
                  weapons the penalty for the offhand weapon is -3 as
                  opposed to -6. The penalty for the primary weapon
                  remains -3. Specifically, these combinations are:
                    
                    
                         Any type of knife may be used in the
                         off-hand in combination with any standard (non
                         exotic) one-handed weapon in the main hand.
                         
                      
                    
                       Two Kris, or one Kris Datoh (main hand) and
                       one Kris (off hand). 
                      
                    
                       One Katana (main hand) and one Wakizashi
                       (off hand), but only if the wielder is specialized
                       in both weapons and takes the appropriate penalty
                       for using the Katana one-handed. 
                      
                    
                       Any weapon that is normally used in pairs,
                       such as Sai, Kama, tonfa, nunchaku, or Iron Claws.
                       
                      
                    
                       Two identical handguns, but only with the
                       appropriate penalty for using them one-handed.
                       
                      
                    
                  
                
                
                   One normal weapon and a shield or item
                   
                    If the character is wielding a 1-handed weapon
                   in his `good' hand, and his other hand is
                   carrying a shield, a non-weapon holy symbol, or a
                   utility item then there is NO strike penalty at all.
                   For the purposes of this rule a utility item is
                   something such as a torch, walking stick, magical
                   totem, leash, or bag. Utility items are not weapons
                   and must not require particular attention to carry.
                   
                
             If a character is wielding two weapons, he may
               always choose to attack with the main weapon only at no
               penalty. In this case the weapon in his off hand is
               considered merely carried, and is not `ready'
               for combat. An attack with the off-hand weapon is always
               made with the full penalty, regardless of what the
               `main hand' is doing.  Note that the Ambidexterity skill from Martial Arts
               can reduce the dual-weapon penalty; and that the
               Multi-weapon Style skill allows the character to strike
               more than once during a single Action A different approach
               is the use of the Multitarget skill which allows the
               martial artist to strike multiple times with a single
               weapon in a single Action 
\subsection{Combat Overview}
     Resolving an attack is a three-step process:   
                
                   The attacker makes a strike roll. A strike roll
                   is an attempt to hit an opponent using a weapon or
                   ability. 
                
                
                   The defender may attempt a defensive maneuver. A
                   defensive maneuver is a special skill that is used to
                   avoid or reduce the damage from an incoming attack.
                   
                
                
                   If the attack was successful, damage is rolled,
                   calculated, and applied to the target. 
                
             This process is explained in detail below. 
\subsection{Attacking}
     An attack is nearly anything that you try to do to
               an opponent during your turn. Typically an attack is a
               physical strike against your opponent with a weapon (such
               as your fist or a sword), but it could also be the use of
               a supernatural ability, magic spell, or a modern device
               such as a gun.  When you attack, you use one of your skills in an
              attempt to damage your opponent. Normally, this will be a
              `basic attack' skill from Martial Arts. Any
              time you make an attack like this, you roll a Strike Roll.
              Strike roll is just a special name for a skill roll in
              combatthey work just like normal skill rolls. To
              recap, roll a number of D20s equal to your level of the
              skill you are using to attack with (i.e. use the Large
              Sword skill to strike with your bastard sword), add your
              attribute and any bonuses or negatives. For each die that
              hits 25 or better, you have a success.
                If you have at least one success, you strike the
                target.   Otherwise you miss your target.
             Some attacks don't require a skill roll. This
               includes special attacks that automatically hit your
               opponent (such as certain abilities from Martial Arts) as
               well as magical spells or supernatural powers that are
               auto-targeting. For example, the magic spell Ray of Heat
               automatically hits your opponent, with no strike roll
               needed. 
\subsection{Defending}
     The target of an attack usually receives an attempt
              to defend himself.
                Defending is using your skills to avoid getting hit
                by an incoming attack.   There are many different
                skills that are available for defense. They are called
                defensive maneuvers. A defensive maneuver is a special
                skill or ability that you use specifically in response to
                being attacked; otherwise they work like normal skills. A
                target of an attack may attempt to use any one defensive
                maneuver in order to protect himself from the attack. The
                most common are parrying, blocking, and the Dodge skill.
                These are described below. Other defensive maneuvers may
                be found in the Martial Arts section.
              
                
                 Parry   
                   Parrying is using your weapon to deflect a blow
                  from another weapon. Parrying may only be performed
                  with melee weapons, against either a melee or hand to
                  hand attack. Parrying is ineffective versus other
                  attacks, such as explosives, bullets, chemical
                  splashes, magical energy, etc. Parry is not a skill
                  itself, rather the skill roll for parrying is based on
                  the weapon skill you are trying to parry with. For
                  example, you would use your Knife skill to parry using
                  your dagger. You must have your weapon drawn and ready
                  in order to use it to parry. Note, that this is the
                  same skill that is used for attacking with your weapon,
                  so most characters will have a good chance at parrying.
                   A weapon or similar combat skill is used for
                   parrying. Also note that parrying is always subject to
                   GM approval. The GM determines the outcome of a
                   successful parry, though it usually means that the
                   defender avoids all damage. However, if you are
                   parrying an attack from a heavy or large weapon with a
                   small weapon, the GM might want to impose a different
                   outcome. 
                  
                
                
                 Block   
                   Blocking is putting a shield between you and an
                  incoming attack.
                    Upon a successful block, a shield's armor
                    values are added to any armor that you are currently
                    wearing   (see below). You can attempt to
                    block any incoming attack that you are aware of.
                    Blocking can be used against melee attacks, missile
                    weapons, arrows, gunshots, and unusual attacks such
                    as a fireball or an acid splash. Larger shields will
                    make it easier to block, more info can be found in
                    the shields section.
                    The Shield skill is used for blocking. Blocking
                    means that the blow hits, but the defender gets to
                    add the armor of the shield to his own armor value.
                      In the case of a non-standard attack, such
                    as a splash of acid, the shield simply stops a
                    certain percentage of the damage outright.
                  
                
                
                 Dodge   
                   one of the best types of defense. Dodge is an
                   (expensive) skill; dodging is moving your body out of
                   the way of an incoming attack. By sidestepping,
                   ducking, throwing yourself to the ground or similar
                   Action you can avoid any attack completely and take no
                   damage. The Dodge skill is used for dodging. On a
                   successful dodge the blow doesn't hit, period.
                   However, dodge is a high rank skill, so most
                   characters will not be very good at it. 
                
             Now that you know what the types of defenses are, we
               are going to lump them together and call them
               `defense' for the sake of simplicity for what
               is to follow.  To defend against an attack, you must succeed at
               your defending skill roll. And you must have MORE
               successes at the attacker had in his attack roll. For
               example, if an attacker has 3 successes to strike you,
               then you would need at least 4 successes at your defensive
               skill to protect yourself. If you have fewer successes
               than the attacker, you are struck by the attack. You can
               only attempt to defend against an attack with one type of
               defense. You can only try to block, dodge, or parry once
               per attack but not multiple of these at a time (i.e. you
               cannot attempt to block and dodge the same attack). You
               can attempt to parry one attack and then dodge the next
               attack, however.  If you are struck you will likely take damage from
               the attack. Also note that you cannot defend (parry,
               block, or dodge) an attack that you did not know about!
                Note that there are other defensive maneuvers
              aside from Dodge, Parry, and Block. These include
              certain magic spells (Blink, for example), martial arts
              abilities (such as Preemptive Dodge), or supernatural
              powers. Regardless of what defense a character chooses
              to use, only ONE defense maneuver may be used against
              any one incoming attack. Some magical spells or
              super powers of a defensive nature are active for a set
              durationusually several rounds. Such spells, such
              as full barrier, are not defensive maneuvers. These
              spells are independent of defense. A character using
              such a spell may still use a defensive maneuver if she
              chooses to.  Magical spells, martial arts techniques,
              and supernatural abilities are considered defense
              maneuvers whenever they are single-use effects that are
              used against only one specific attack (such as
              Blink). Some attacks, such as
              automatically targeting magical spells, will strike your
              opponent regardless of defense. Defensive maneuvers are
              ineffective against these spells. For example, Ray of
              Heat as discussed above will always strike its
              target. Call Lightning will always strike its target, if
              the caster succeeds on his magical control roll to cast
              the spell. For these abilities, defensive maneuvers have
              no effect. 
            
\subsection{Damage}
     When an attack hits its target, it will inflict
               damage upon it. Every weapon has a damage rating
               associated with it as well as a damage type (for example:
               a dagger is a 16P weapon). The damage rating is the
               amount of damage it will do, and the damage type reflects
               how the weapon deals its damage.  Here's a key of the different damage types
               with abbreviations in parenthesis and what they are
               typically associated with:   
                
                 Piercing (P)   
                    using a sharp tip to penetrate the target
                   creating a hole or deep cut within the target. Some
                   melee weapons, most bullets, and other penetrating
                   weapons deal piercing damage. Armor can protect
                   against this type of damage. 
                
                
                 Slashing (S)   
                    using a cutting edge to slice the target
                   creating a fairly shallow long cut within the target.
                   Most melee weapons such as knives and swords, as well
                   as lasers deal Slashing damage. Armor can protect
                   against this type of damage. 
                
                
                 Crushing (C)   
                    using a heavy blunt edge in an attempt to
                   fracture or break the target. This is typically the
                   hardest damage to protect against, and as such, is a
                   favorite type of damage for melee combat. Punches,
                   clubs, staves, and explosions deal Crushing damage.
                   Armor can protect against this type of damage. 
                
                
                 Undefined (U)   
                    using energy to cause damage within the target.
                   This can include electricity, heat, magic, sickness
                   and other `odd' types of damage. Normal
                   armor cannot protect against this type of damage. Only
                   your character's Base Armor (BA) can protect you
                   from undefined damage. Any time someone takes
                   undefined damage he or she will always take at least
                   one point of damage, regardless of the Base Armor.
                   
                
                
                 Direct (D)   
                    extremely damaging attacks deal D damage. There
                   is no way to protect yourself from Direct Damage, as
                   it will bypass all armor and other protections that
                   you have on. Damage from D attacks is subtracted
                   directly from Hit Points (HP). 
                
             Consider the dagger again: 16P  This tells us that a strike from a Dagger deals 16damage. Also, we know that it deals piercing damage.
                Ok, so now we've got our weapon's damage
               rating and it's damage type. Roll the given damage
               as described in the dice rolling section at the beginning
               of the rules section. The higher the damage roll, the more
               severe the attack was upon the target.  After the damage roll is determined, subtract that
               damage from the target's Hit Points (HP). Armor can
               reduce the damage taken, which brings us to...
               
\subsection{Armor}
     Armor is any covering that you are wearing or behind
               that will absorb some of the damage applied to you. All
               armor is given three numerical ratings for piercing,
               Slashing, and Crushing protection. The higher any one
               number is, the better it will protect you from that type
               of damage. Typically this value sequence is represented as
               P/S/C (Piercing/Slashing/Crushing). An example would be
               3/2/3, which provides 3 points of piercing protection, two
               points of Slashing protection, and 3 points of Crushing
               protection.  When taking damage subtract the armor value, which
               matches the damage type of the weapon, from the damage
               amount.  Here are a few examples: 
\quotexample[Carlos vs. the Bandit]{ Carlos was attacked by a bandit with a short
                 sword. The bandit made his skill roll to attack and had
                 three successes. Carlos attempts to parry and had only
                 one success. Since Carlos has fewer successes than the
                 attacker, he fails to parry and takes full damage. The
                 GM rolls damage for the bandit and determines that the
                 bandit's sword blow deals 8 damage. Carlos is
                 wearing soft leather armor, which has a P/S/C rating of
                 2/3/0. We know the short sword dealt 8 damage. But it is
                 a Slashing weapon (denoted by the S in the weapon
                 damage), so the armor provides 3 points of protection
                 against this attack. So we take the damage of the weapon
                 (8) and the amount of protection the armor gives (3),
                 and subtract the armor value from the weapon damage. So,
                 Carlos just took 5 damage (8-3=5). }
  
\quotexample[Lira the Gladiator]{ Lira is fighting a gladiator-style duel. Her
                 opponent attacks with a spear and has two successes to
                 hit. Lira wants to block and rolls her shield skill. She
                 has four successes on her skill roll. This means that
                 she successfully blocked the incoming attack with her
                 shield. The GM rolls damage for the attack and gets 14
                 P. (that's a strong blow!) Lira is using a Splint
                 Shield (P/S/C of 9/9/9) and is wearing Hard Leather
                 armor (3/4/1). First, apply the damage against
                 Lira's shield. Her shield has an armor of 9 versus
                 P. This means that her shield stops 9 points of the
                 incoming damage. However, since the spear blow dealt
                 more damage than her shield has armor, the weapon
                 continues through her shieldwhich now has a hole
                 in it. Lira now deals with the remaining 5 damage. (14
                 damage was dealt, minus the 9 taken by the shield = 5).
                 Her armor has a P value of 3, meaning that she only
                 takes 2 points of damage to her HP total. (14 damage - 9
                 from shield - 3 from armor = 2) }
   Now that you have that, lets introduce a few tricks.
               The entire next two paragraphs are incredibly important,
               so pay attention!  Weapons that deal undefined or Direct Damage (U or D
               damage) bypass all armor values. Undefined damage can be
               protected against by your character's Base Armor
               (BA) sub-attribute. Just subtract the BA from the damage
               taken as if the BA was normal armor. Note that Base Armor
               is not used for protecting against anything but undefined
               damage! A target will always lose at least one HP when hit
               with an attack that deals undefined damage.  Direct Damage cannot be protected against. Not by
               normal armor, Base Armor, not anything. It's harsh,
               so be careful when on the receiving end of something that
               deals D damage. D damage is applied directly to your Hit
               Points. 
\subsubsection{Armor Piercing}
     Some weapons have a note that says
                 `AP' followed by some number. This kind of
                 weapon has an unusual ability to pierce armor. If an
                 armored target is struck by a weapon with
                 `AP', the armor is reduced by the AP number
                 for the purposes of damage calculation. Note, that this
                 cannot go negative. If an `AP: 10' weapon
                 strikes someone wearing 5 armor, then the armor is
                 totally ignored. It does not, however, go to -5. 
\quotexample[Lira Strikes With Gladius]{ Lira is wielding a gladius, which is a type of
                   short sword that has AP: 5. She successfully strikes
                   her opponent, and rolls 11 damage. Her opponent has an
                   armor of 7 versus Slashing damage. First, deduct the
                   AP from the armor: 7 armor - 5 AP = 2. Then, deduct
                   the armor from the damage like normal: 11 damage - 2
                   armor = 9 damage. Lira's opponent takes 9
                   damage. }
    Consider the same situation as above, but assume
                 that Lira's opponent's armor is only a 3
                 versus Slashing. First, deduct the AP form the armor: 3
                 armor - 5 AP = -2. However, this cannot be negative, so
                 we cap it at zero. Now, deduct the result from the
                 damage: 11 damage - 0 = 11. Lira's opponent takes
                 all 11 damage.  AP is effective against armored targets, but it
                 does nothing against unarmored targets. Note, that most
                 weapons with AP don't deal very much damage. It is
                 usually a good idea to use AP weapons versus armored
                 targets only, and use `normal' weapons
                 versus unarmored targets. 
\subsubsection{Fire (F)}
     Some weapons and spells have a note that says
                 `F', `Fire', `Fire
                 Effect', or 'U Fire'. These weapons
                 deal fire damage. Unless stated otherwise Fire weapons
                 deal U damage. As well, Fire effect weapons will ignite
                 the target, as well as readily flammable materials
                 within a 3' radius of the blow. A target that is
                 `on fire' will suffer an extra 16U damage
                 each turn that the fire is still going. This starts on
                 the `next' turn after the target is struck.
                 Certain types of armor have Fire Protection. This means
                 that this armor has the unique ability to resist heat
                 and the damage caused by fire-effect sources. If a
                 target with fire protection is struck with a fire-effect
                 attack, damage dealt from the fire is zero and the
                 target is not ignited. Fire protection on armor will
                 only work three times before the armor must be repaired
                 or replaced. Some armor is rated fire protection
                 \ensuremath{\times}2, which will take six fire-effect hits before
                 loosing it's fire resistance, etc. Note that fire
                 protection armor only resists the ignition effect and
                 any heat (U) damage from the fire specifically.
                 Therefore, if a weapon deals normal damage (P/S/C) AND
                 has either Fire effect or Fire damage (U) then the
                 fire-related effects are prevented but the standard
                 damage (P/S/C) is treated as normal. Targets that are
                 `on fire' suffer the damage on each of their
                 turns. 
\quotexample[David deals Fire damage]{ David shoots a Ghoul with a Flame blaster. This
                   weapon deals 26U Fire. David rolls damage and ends
                   up with 10 U. The target suffers the 10 damage. As
                   well, the Ghoul suffers the fire effect and has been
                   set on fire. On the ghouls next turn it will
                   suffer 16U from being on fire. This will continue
                   until the fire is somehow put out. }
  
\quotexample[Travis's armor resists Fire damage]{ Travis is wearing Ceramic Plate armor and is
                   shot with a 12-ga shotgun loaded with a meteor round.
                   The meteor round deals 56C + Fire and Travis's
                   armor has 8 points of protection vs. C, and has Fire
                   protection. The attacker rolls 19 C damage. Travis
                   takes 11 damage from the C portion of the blow, but
                   his armor resists the ignition effect so he is not set
                   on fire and suffers no further damage from the attack.
                   }
  
\subsection{Combat with Different types of Weapons and Effects}
     Apathy includes rules for combat with melee weapons,
               ranged weapons, firearms, magic, and so forth. The rules
               for attacking and defending with different kinds of
               weapons and effects are detailed here.   
              
                 Hand-to-Hand attacks:   
                  Hand-to-hand attacks include things like punches,
                 kicks, and body weapons (teeth, claws, etc.).
                 Hand-to-hand combat may be Parried, Dodged, or Blocked.
                 
                
              
                 Melee Weapons:   
                  Melee weapons are handheld weapons that are not
                 thrown or launched. This includes swords, knives, axes,
                 hammers, most polearms, etc. Melee weapon attacks may be
                 Parried, Dodged, or Blocked. 
                
              
                 Missile Weapons:   
                  Missile weapons are archaic type weapons which
                 are thrown or launched at the target. This includes
                 throwing knives, thrown spears, arrows, crossbow bolts,
                 sling bullets, etc. Missile weapons may be Dodged, or
                 Blocked normally but may not be parried (except for
                 special abilities that may allow one to parry them)
                 
                
              
                 Projectile Weapons:   
                  Projectile weapons are guns, shotguns, rifles,
                 pistols, lasers, particle beams, and other fast moving
                 kinetic weapons. Projectile weapons may be Dodged or
                 Blocked, but NOT Parried. 
                
              
                 Explosives:   
                  `Explosives' in this case includes
                 such things as grenades, molotov cocktails, bottles of
                 acid, etc. These attacks may be Dodged or Blocked. Note,
                 these attacks have a blast radius. If the defender has 1
                 more success than the attacker, it means that a direct
                 hit was avoided but the defender still suffers the
                 radius damage. If the defender has two or more successes
                 over the attacker it means the entire attack was
                 defended against. If the attacker ties or beats the
                 defender then the defender takes full damage. Note that
                 most true explosives, such as grenades, missiles, or
                 bombs, deal double damage on a direct hit and normal
                 damage in their blast radius. 
                
            
\subsection{Hit Points}
     Hit Points are numbers used to represent how injured
               or healthy a character is. A character will have TWO Hit
               Point numbers. Your maximum Hit Points determines how
               tough your character is at full health. This number will
               rarely change. It is determined by your BLD and WIL
               attributes, and will rarely change unless your attributes
               change.  Your current Hit Points is the most important number.
              It will likely change frequently during combat. Hit Points
              are generally represented as a fraction: 
              Current HP  Max HP  . When the
              game begins, your character will be fully healthy and
              uninjured. Thus, when starting, your current HP will be
              equal to your maximum.
             As you get injured, any damage you take (after
               applying armor, etc.) will be applied to your current HP.
               As you get injured, your current HP drops. Over time, your
               body will heal itself, and your HP will go up. Seeking
               medical attention will speed this process. Other things,
               like magic or cybernetic devices, can heal you as well.
                Your current HP can never exceed your maximum HP. As
               a general rule for minor injuries (less than 5 HP), you
               gain back 1 HP per day of average activity. For
               significant HP losses, it takes   
                         (    Maximum  
                      -    Current    )  
                        2 
                         days to fully recover your HP.
             If a character is resting and is under basic medical
               care, then your healing time is cut in HALF. If your
               character is under the care of a professional Doctor or
               Healer, healing time is cut to one QUARTER of normal time.
                If your current HP falls below 0, your character
               falls unconscious. It is possible to remain conscious by
               making a willpower check at -X, where X is the amount
               below zero your HP are. If your current HP are below 0,
               your character is bleeding badly (internal bleeding is
               possible too) and you will loose 1 HP per round, unless
               you receive medical attention. If your current HP ever
               reaches the negative of your maximum HP, you have died.
               
\subsection{Called Shots}
      Whenever a character attacks he is aiming for the
               most convenient blow that he can make. It is assumed that
               this is a `center mass' attack: In other
               words, the attacker hits the largest, most obvious,
               portion of the target's body. For humans and most
               animals this is the torso area.   However, there are times when a character may want
               to strike a specific part of the target's body.
               Attacks that are aimed at a specific portion of a target
               are called shots. Called shots may be attempted with any
               kind of attack or weapon. The attacker must be able to see
               and identify the exact target that he is trying to hit.
                 When performing a called shot the attacker must
               declare in advance that he is making a called shot, and
               where he is aiming. It takes one action for the character
               to identify and focus on the specific target that he wants
               to hit. A called shot always incurs a strike penalty.
               Consult the called shot table on the reference sheet for
               the appropriate negative. If the attacker fails his strike
               roll, then the attack missies completely. (It does not,
               for example, hit `normally' as opposed to the
               called target). If the attacker makes his strike roll,
               then the target may defend as normal.   A called shot cannot be attempted with an
               auto-targeting magic spell, such as Ray of Heat; these
               effects always strike `center mass'. However,
               called shots may be attempted with magical spells that
               require a roll, such as Fireball or Call Lightning.
                 The effects of a called shot are left up to the GM
               to resolve on a case-by-case basis. The effects would
               depend on the location of the called shot and also the
               amount and nature of the damage dealt. 
\quotexample[Philip misses the troll]{  Philip has engaged a troll in combat, and
                 declares that he wishes to make a called shot, aiming
                 for the troll's head. The GM assigns a strike
                 penalty of 8 for the called shot. Philip fails
                 his strike roll, and therefore misses the target
                 completely. }
  
\quotexample[Morgdar is fighting a lizardman.]{ He attempts to make a called shot targeted at the
                 lizardman's eye. He successfully makes his called
                 shot and the lizardman fails to defend. The GM asks that
                 damage be rolled. Morgdar's spear deals 3 damage.
                 The GM rules that enough damage has been dealt to put
                 out the lizardman's eye, thereby preventing it
                 from attacking properly in combat. The GM notes that had
                 Morgdar rolled higher damage, then the blow could have
                 been instantly lethal. }
  
\subsection{Cover}
      Cover is something that a character or a creature
               is obscured by. Cover has two effects:   
              
                  Cover can make a target, or a specific part of a
                 target, difficult to see and therefore difficult to hit
                 in combat. 
                
              
                 Cover can act like armor, which can reduce or even
                 prevent damage that would otherwise harm a target.
                 
                
              A target has cover if there is something between
               the attacker and the target. This something could be
               foliage, walls, other combatants, etc. The exact amount of
               cover should be determined by the GM. A target has total
               cover when the target is completely obscured by the
               covering object. In the case of total cover, the attacker
               cannot see the target.   An attacker has two options when confronted with a
               covered target.   Attempt to strike around cover: The attacker takes
               aim at part(s) of the target that are not behind cover.
               This assigns a negative to the attacker's strike
               roll, which is determined by the GM. Refer to the
               reference sheet for sample strike penalties for cover. The
               attacker may also attempt a called shot against a portion
               of the target's body that is not behind cover using
               the normal called shot negatives. If the attacker fails
               the strike roll, then the attack misses, period.   Attempt to strike through cover: The attacker
               strives to hit the target like normal, trusting that the
               weapon will penetrate the cover and strike the target
               anyway. When doing so, the attacker has a 2 to
               strike regardless of the severity of the cover. This
               2 is in addition to any other penalties that might
               be present as well. Then, deduct the armor value of the
               cover from the damage dealt by the attack, just as if the
               defender were using a shield.   A target that has total cover is completely hidden,
               and may not be targeted at all.   A magic using character may still attack using an
               auto-targeting spell if ANY portion of the target's
               body is visible. This incurs no strike penalty whatsoever.
               However, if the magical effect is impeded by the cover,
               then damage may be reduced.   The act of ducking behind cover will give a bonus
               to Dodge rolls. However, the cover does not provide a
               strike penalty or armor unless the targeted character was
               already behind the cover before the attack started (in
               which case there is no dodge bonus).   Targets that are actively attempting to hide behind
               cover are often restricted in their movements. When using
               cover, defenders are at 3 to all defensive maneuver
               rolls that they choose to use. Note that if the cover is
               sufficiently large then it doesn't require any
               special effort to hide behind it. In this case there is no
               penalty to defensive maneuvers. 
\quotexample[Darin and her comrades are fighting a group of
                 trolls.]{ One troll has ducked behind some rocks. The GM
                 states that the troll has half cover due to hiding
                 behind the rocks. If Darin wants to shoot the troll with
                 her crossbow, she has 8 to strike, OR she may
                 choose to make a called shot against some part of the
                 troll that is not behind cover. In that case she must
                 take the called shot penalty for whatever target she
                 selects. }
  
\quotexample[Mikus is in a gunfight.]{  His opponent, a local gangster, has ducked behind
                 the door of a nearby car. The GM rules that the gangster
                 has dense cover, and thus Mikus has 12 to strike
                 him. Mikus decides he'd rather take chances with
                 the armor value of the car's door rather than try
                 and shoot the gangster at such a high penalty. Mikus
                 takes aim at where he thinks the gangster's body
                 is situated behind the car doorthereby incurring
                 a 2 for attempting a shot through cover. Mikus
                 makes his roll. Rolling damage for his .224 BOZ, he
                 deals 19 damage. The car's door provides 8
                 armornot enough to stop the bullet. The
                 unfortunate gangster suffers the remaining 11 damage.
                 }
  
\part{Rule Sections}
    
\chapter{Academics}
    
\chapter{Augmentations}
    
\section{Additions}
     There are two basic types of addition augmentations.
             The first are small devices that don't take up a lot
             of room. These are easy to add to the body because they
             don't require any power or a lot of space. As a
             general rule, this type of augmentation is very small and
             its function is often limited.  The second type are larger, complex systems, which
             take up a lot of space in the body. Generally, the only way
             to add something like this to your body is to remove, move,
             replace, or otherwise modify some other part of your body in
             order to accommodate it. This works on a system of slots
             (see below) which is similar to the Mecha system. 
\section{Replacements}
     Augmentation replacements, in their most basic forms,
             are exactly the same function-wise as normal body parts. Of
             course, many augmentation parts are upgrades of some sort.
             They might be stronger, faster, or more durable than normal.
             Some augmentation replacements have new functions or
             different abilities than normal. 
\section{Augmentation Limbs and Bodies}
     Entire limbs or even entire bodies may be replaced
             with augmented versions. There are two basic types of
             replacement limbs and bodies: Robotic or Cybernetic.  Robotic limbs are built entirely from non-organic
             materials. They are machines made of steel, high-tensile
             alloys, wires and high pressure hoses. Movement comes from
             hydraulics, electric motors, gears and levers. Robotic body
             parts contain no flesh or blood.  Cybernetic limbs are basically organic. They consist
             of bone, tendons, and muscles just like
             `natural' limbs. However, cybernetic limbs are
             enhanced with technology including electric implants,
             genetically modified tissues, and some synthetic material
             reinforcements.  These rules apply to cybernetic and robotic
             components: 
\subsection{Robotics}
      
                  Do not bleed when injured (though they might leak
                 oil or make sparks) 
                Have unnatural heat signatures   
                  Do not look or feel realistic in the slightest;
                 rather they look like machinery. 
                  Take double damage from electrical sources.
                 
                  Are immune to disease, poison, and magical
                 effects that work on living targets (a robotic body part
                 is not alive). Note that this means that healing magic
                 or typical healing/medical items have no effect on
                 robotics. 
                Have double the HP of non-robotic parts.   
                  Robotic limbs cannot be upgraded directly. They
                 can, however, be easily removed and replaced with
                 upgraded parts. 
                  Used robotic parts generally have high resale
                 value and demand. Used parts are regularly available.
                 Many robotic parts are interchangeable. 
                  Robotic parts are considered fairly easy to
                 replace and repair. Some parts can be swapped in a
                 matter of minutes. 
                  Robotic parts do not heal automatically. If
                 damaged, repairs must be made by a mechanic or by
                 replacing the damaged parts with new ones. 
            
\subsection{Cybernetics}
      
                Bleed when injured   
                  Appear normal when viewed casually, or on thermal
                 sensors. Many cybernetic limbs will appear 100\% normal
                 even under close visual inspection. However, inspection
                 by a knowledgeable person (Doctor, Nurse, Cyber Doctor,
                 etc.) will reveal the true nature of the limb. 
                  Usually feels realistic to the touch, but not
                 necessarily. 
                  Are fully susceptible to disease, poison, and
                 life-affecting magic just like any other living tissue
                 (cybernetic body parts are living tissue) 
                  Cybernetic parts can be upgraded at a later date
                 without total replacement. However, this usually
                 requires surgery. 
                  Cybernetic parts are usually built around the
                 user's body. Used cybernetic parts are valuable,
                 but the potential market is limited to those people who
                 are very similar to the original owner. Therefore the
                 used cybernetic parts market tends to be flooded with
                 supply for which it is difficult to find a buyer. It is
                 often time consuming to find a donor or source for a
                 particular part. Cybernetics parts prices for both
                 buyers and sellers will vary wildly depending on market
                 conditions, which are nearly impossible to predict.
                 
                  Repair to cybernetic limbs is a combination of
                 electro-mechanical work and traditional medicine.
                 Repairing significant injuries or adding/removing
                 cybernetics often entails surgery, a hospital stay, and
                 may well require lengthy recovery time. 
                  Minor injuries to cybernetic limbs heal just like
                 injuries to a normal human. Magical healing and
                 regeneration works for cybernetics in most cases, though
                 healing may be incomplete if the damage is severe.
                 
            
\section{General Augmentation Rules}
      
                All humanoid characters have, by default, (size)
               free slots in the body. These are assumed to be located in
               the torso. (Humans, which are size 2, start with +2 slots)
               If you want more slots to store things in, you have to
               purchase augmentations that give you back additional
               slots. 
                After all augmentations are chosen for a character,
               add up the slot total. This number must be positive. (In
               other words, augmentations cannot be installed unless
               there are slots available to house those parts) 
                A robotic limb may be attached to any kind of
               torso. However, a cybernetic limb can only be attached to
               a natural (non-robotic) torso. This is because all
               cybernetic limbs depend on the support organs and blood
               supply found in the torso. 
          
\section{Determining the Cost of an Augmentation body part}
     This section is incomplete. 
\section{Arm, Hand}
    
\section{Body}
    
\section{Head}
    
\section{Leg}
    
\section{Torso}
    
\chapter{Equipment}
    
\section{Ammunition}
    
\section{Armor}
    
\subsection{Basics for Armor and Damage}
      
                  Armor has 3 main values: Piercing, slashing, and
                 crushing. Armor values are represented by 3 numbers
                 separated by slashes (represented by: P/S/C;
                 piercing/slashing/crushing). Example: Armor with values
                 of 4/3/7 would have 4 points of protection vs. Piercing,
                 3 vs. Slashing, and 7 vs. crushing damage. 
                  Armor does not protect from Undefined (U) or
                 Direct (D). 
                  A character's Base Armor is the only thing
                 that can protect vs. U damage. Nothing can protect
                 against D damage. 
                  All non D/U damage has an associated damage type
                 to is; either P, S, or C. Subtract all matching armor
                 values of the character from the damage being done. If
                 the damage value is still above 0, then the character
                 takes that amount of damage. 
                  A weapon with an armor piercing (AP) attack will
                 list the amount of AP it has. Subtract this AP number
                 from all of the armor's damage values when
                 determining its protection against the attack. This
                 cannot drop the armor's value below 0. 
                  GMs determine when armor is damaged beyond
                 use/protection. 
                  Hard Armor is any armor that uses hard
                 ceramic/plastic/metallic/wooden plates or coverage to
                 protect the character. Hard armor is MUCH more durable
                 than non-hard armor. Hard armor is denoted by
                 `hard' on the armor sheet. Note that the
                 components of hard armor are hard. They can't be
                 bent without damaging the armor. Hard armor is generally
                 made specifically for the wearer, it will only fit
                 others with nearly identical body proportions. 
                  Soft armor (armor that isn't considered
                 hard) is at least somewhat flexible. It will generally
                 fit anyone of similar size/stature. 
                  Armor with the descriptor of `fire'
                 has fire protection. This means the armor has the
                 `natural' ability to deflect and protect
                 against fire attacks. The armor will protect from all
                 fire damage (assuming the fire hits the armor, so full
                 suits are better for this). 
                  Armor can be layered, but only one hard suit can
                 be worn at a time. A person can have different hard
                 armors on different parts of the body. Non-hard armors
                 can be layered indefinitely. Add all armor values of the
                 armors together to give the full armor protection. If
                 any armor has fire protection, fire will be stopped, but
                 any armor on top of it is still susceptible to the fire.
                 
                Encumbrance is determined by: BLD/3   
                  A character can wear (without penalty) as much
                 armor in size levels as his Encumbrance score will
                 allow. Add up all layers of armor to get the complete
                 Size level of the armor, if this number is equal or less
                 than Encumbrance, then no negatives are given to the
                 player. 
                 When a character has more armor than his
                Encumbrance (ENC) allows, then the player has the
                following penalties against him:
                  
                    -2 DEX per pt over ENC   
                    -1 action per 2 full points over ENC   
                  
                
                  An armor's Class is a general idea of what
                 period of history/future the armor would appear. A later
                 period can find the earlier armor. 
            
\subsection{Shields}
      
                  When creating a shield, there are 2 factors. The
                 material the shield is made of, and the type of shield.
                 Choose both of these values when creating a shield to
                 get the full set of statistics for the shield. 
                  On a successful block, a shield will subtract its
                 appropriate armor value from the damage. Any damage that
                 penetrates the shield will then be applied against the
                 user's armor. 
                 Shields can be used to block U attacks that are
                `thrown' at them. These include fireballs,
                lightning strikes, breath weapons, chemical splashes,
                etc. To block one of these attacks, the user must make a
                block roll at -3. If successful, then damage will be
                reduced based on the type of shield:
                  
                    Buckler: Full damage   
                    Small:  1  4
                   
                    
                    Large:  1  2
                   
                    
                    Tower:  3  4
                   
                    
                  
                
            
\section{Energy Weapons}
      Most energy weapons are powered by a battery pack.
             These are interchangable between weapons. Energy within the
             packs is consumed as the weapon is fired, like normal
             bullets. They are called `batts' or
             `b-paks' in slang. B-paks store energy in units
             of EU. Different weapons consume EU at different rates.
             Smaller weapons use 1 EU per shot. Larger weapons might use
             more. Pistols can hold ONE pack. Rifles/cannons can usually
             hold 3. B-paks can be recharged. Recharging a pack takes
             about thirty seconds per EU that has been used. As a general
             rule, it costs \$1 per EU to have pack recharged. HD paks
             require special equipment to charge and these cost about
             \$1.50 per EU to charge. 
\section{Melee}
      For damage type listed as X/Y, the weapon can deal
             either type of damage for a given attack. For a P/S weapon,
             the attcker can choose to do a P or S attack. For every
             point of STR under Min. STR, the attack gets -1 damage and
             -2 to strike/parry. STR Req is the minimum strength required
             to wield the weapon. For every point of STR over the
             Strength Requirement, add one point of damage to the raw
             roll, up to a maximum of the MSB (Max Strength Bonus) A raw
             bonus is always indicated with square brackets. So 34+1+4would be a normal 34+1roll, with an additional 4
             added to the final result. When wielding a 1-H weapon with
             two hands, assume the wielder has +2 STR.   
            
                Hands   Number of hands
                required to use the weapon
              
            
                Rec   Number of actions it
                takes to recover after attacking with the weapon. This
                many actions must be spent to get the weapon ready for
                attack again.
              
            
                Enc   Represents the weight of
                the weapon. This consumes Encumbrance just like armor.
                Only use this when the weapon is actually being used.
              
            
                Range   the effective
                `Reach' of the weapon in feet. If your target
                is beyond your range, you cannot strike them. You can
                spend an action to gain range on your opponent. If your
                target is closer than 1/2 range, then you are at -6 to
                strike the target.
              
          
\subsection{Options for Melee Weapons}
      Prices listed are for basic weapons with enough
               durability to function in real combat situations. They are
               not showpieces or wallhangers. However, other options are
               available for an extra cost:   
                
                  
                     Quality
                     
                   Bonus to damage   
                   Cost factor   
                   Rarity   
                  
                  
                  
                   Exceptional Quality   
                   +1 damage   
                   double cost 
                   Adds 1 to rarity   
                  
                  
                   Master Quality 
                   +2 damage 
                   10 times cost 
                   Adds 3 to rarity   
                  
                  
                   Legendary Quality 
                   +3 damage 
                   100 times cost 
                   Adds 5 to rarity   
                  
                  
                   Vibro-edge (bladed only) 
                   +10 AP 
                   5 times cost 
                   ULTRAMODERN   
                  
                  
                   Wooden Training 
                   1/2 damage, C 
                        
                             1  10
                              th 
                           cost     
                    
                   Common   
                  
                
                Options for melee weapon construction
              For other than normal sized weapons (mecha, etc.),
               all stats are 
                     2      Size    -
                      1   
                     normal. This applies to strength
                  requirement, strength bonus, damage dice number, etc.
            
\subsection{Axes}
    
\subsection{Blunt}
    
\subsection{Body}
    
\subsection{Chained}
    
\subsection{Exotic}
    
\subsection{Knives}
    
\subsection{Large Swords}
    
\subsection{Monofilament}
      The Monofilament Whip is length of metallic,
               diamondoid, or silicoid wire that is the thickness of a
               single molecule, with a weight on one end. The Archaic
               WeaponChain skill is used. A skill roll of 4 or
               less is considered a critical failure (instead of the
               normal 1). This modified critical failure is there for
               people who are skilled. Those who are not get a critical
               failure on a 9 or below.   The recover time of the Monofilament Whip cannot be
               reduced by strength, and will always require a 1 action to
               recover from safely. The user can attempt to reduce the
               recover time to 0, but there is a 50\% chance that the user
               will strike himself with the whip dealing full damage to
               himself. 
\subsection{Plasma}
      Plasma weapons are designed for a specific set of
               modifications. These are:   
                
                  
                   Type 
                   Effect 
                   Cost Factor   
                  
                  
                   High temperature   
                     Blade is hotter than normal, dealing +1
                     damage, can be purchased up to 3 times (each time
                     double cost) 
                   x2   
                  
                  
                   Over powered field   
                     The magnetic field is stronger than normal,
                     giving a +5 STR when breaking from a lock 
                   x2   
                  
                  
                   High speed blade   
                     The plasma weapon can be turned on in 0
                     actions 
                   x1.5   
                  
                  
                   Power burst   
                     Once per minute the blade's power can be
                     increased to deal +3 damage 
                   x1.5   
                  
                
              Plasma Weapon Modifiers  
            
\subsubsection{Notes: specific to Plasma Weapons}
      
                   Plasma weapons are a bladeless handle (typically
                   looks like a normal weapon handle or a flashlight)
                   from which a glowing blade of extremely hot plasma
                   vents forth. The plasma is held in the shape of a
                   cylinder (or blade) by a magnetic field. They take an
                   action to turn on. 
                   Due to the difference between using a plasma
                   weapon verses normal weapons, they require the Archaic
                   Weapons - Energy skill to use. Anyone can use their
                   normal sword skill (or axe skill respectively) but
                   will get a critical failure on a 4 or below (instead
                   of the normal 1). 
                   Plasma weapons cannot be parried by non-plasma
                   weapons, likewise, they cannot parry non-plasma
                   weapons. Similarly, shields cannot block plasma
                   weapons. Plasma weapons can parry other plasma
                   weapons, and force fields protect against them as
                   normal If a plasma weapon comes in contact with
                   another one, there is a 25\% chance the magnetic fields
                   of the blades will lock together. It is easy to tell
                   if combatants are locked as an extremely bright glow
                   emanates from the contact point of the weapons.
                   Getting out of a lock requires an action where a
                   strength contest ensues. If there is more than a
                   difference of 8 between the two strength rolls, the
                   lock is broken and the loser loses his or her next
                   action. If no break occurs, the combatants must
                   attempt again on the next action. Either combatant in
                   a lock can turn off their weapon, in which case he or
                   she automatically loses the lock challenge, and gets
                   the same negative as if he lost due in a STR contest.
                   
                   One side effect of having a magnetic field
                   around the blade, is that it is possible to parry
                   projectile energy attacks (such as shots from a beam
                   rifle). Doing so the wielder is at a -8. 
              
\subsection{Polearms}
    
\subsection{Rocket Assisted}
      Rocket-assisted weapons are modern versions of
               traditional archaic weapons. They have small rocket motors
               contained in the weapon. When the weapon is swung, the
               motor ignites, giving extra force to the blow. They are
               used with the same skill as normal archaic weapons.
                 If such a weapon is used, and the target is missed,
               then it takes an extra two actions to recover. If the
               target is hit but the blow is parried or blocked, then it
               takes the normal time to recover.   When using rocket assisted weapons, you need an
               extra success to parry melee attacks. (At least two better
               than attacker.)   Each weapon can store up to 5 `rocket
               charges.' When these are used up, they must be
               refilled before the weapon can be used. Each charge costs
               \$100 to refill. 
\subsection{Staves}
    
\subsection{Thrown}
    
\section{Miscellaneous}
    
\section{Ranged}
    
\section{Shields}
    
\chapter{Magic}
    
\section{Introduction}
      Magic exists in many forms. It is a supernatural
             energy that permeates everything in the world. Characters
             with the right skills can bend this energy to their
             willand perform amazing feats in the process. Magic
             is capable of healing the sick, raising the dead, creating a
             great fireball, protecting the caster from harmand
             nearly anything else you can imagine.   Magic, whatever type it might be, is all about
             manipulating MANA. Mana, as mentioned above, is a
             supernatural energy that is present in everything. It is in
             the air, the ground, in your food, even in your body. Mana
             is what is responsible for `enforcing' physical
             laws such as gravity, etc. When a person uses magic, they
             are manipulating mana to do what they want. Normally, this
             is the mana that is in their own body. It is possible to
             work with mana outside one's body as well. Nobody
             really knows exactly how mana works, but special gestures
             and body movements handed down throughout time are believed
             to initiate complex quantum-mechanical events that result in
             seemingly impossible things taking placemagic.
               Mana is found in all things. However, more mana is
             found in some things than others. Gases (air, for example)
             have very little mana. Denser substances such as water and
             stone have more mana. Complex things such as man-made
             objects (pottery, woodcarvings, machinery, etc.) contain
             quite a lot of mana. Living things, from mice to humans to
             whales, contain even more mana. Mana is released back into
             the environment when living things die, or when mana-rich
             objects decay, are burned, or broken. Other objects tend to
             `soak up' mana that is released near them. Thus,
             older items tend to contain more mana than do
             `new' ones. Whenever an item is broken it is
             possible to extract some of its mana as it dissipates back
             into the environment. This is also possible when a living
             creature dies. In game terms, we call mana `MP'
             (Mana Points / Magic Points).   Examples of locations/things that are very rich in
             mana:   
              Graveyards, grave markers   
                Large Churches, Cathedrals, holy symbols, idols,
               and relics 
                Museums, famous artwork or historical objects
               
              Antiques   
              Ancient ruins, crypts, etc.   
            Examples of locations/things that have a lot of mana:
               
              Old houses, family heirlooms   
              Smaller churches and shrines.   
                Local landmarks such as wishing wells, old trees,
               popular meeting places 
                Very old people, especially those that are revered
               by others. 
                Important people with a lot of public exposure or
               reknown (royalty, politicians, popular musicians, etc)
               
              Slaughterhouses   
              Fossils, natural gemstones   
            Examples of locations/things that have very little
             mana:   
                Barren landscapes such as desert or arctic regions
               
              Sky   
              Some underground caves or dungeons   
                `Ghost towns' / looted or abandoned
               buildings 
              Newly cleared land   
          
\section{Types of Magic}
    
\subsection{Invocation Magic (spellcasting)}
      Invocation spellcasting is the most common and
               basic form of magic. This is the art of the wizard and
               sorceress, where characters use innate mana to cast
               spells. Invocation spells are based on methodical,
               practiced, gestures and chanting of magical incantations.
                 Incantation magic is learned. The invocation
               magician will have a list of spells that she has studied
               and memorized. As long as she has mana (MP) available, she
               may cast any spell that she knows. Because invocation
               magicians use innate mana and well defined, repeatable,
               forms it is the easiest type of magic to learn.   Spells are learned from other magicians, studious
               research, or a combination of the two. Research requires
               access to various books on the subject of magic, including
               book(s) that are specifically about the spell to be
               learned. True books on magical subjects are few and far
               between, and books about high level spells are extremely
               valuable. Invocation magicians have invented a special
               language to record and describe magic spells and the
               procedures required to cast them. All true magical books
               will be written, at least in part, in this special
               language. Wizards and Sorceresses might hail from
               different countries, races, and moral
               standpointsbut they all share common ground in the
               ability to read magical script. Magical script is
               specially designed to describe and transcribe magical
               spells, therefore it is rarely used for other purposes. It
               is difficult to carry on a conversation, written or
               spoken, in the language of magic.   More often than not a wizard will belong to a
               particular clan or school, where many wizards pool their
               knowledge and references for members to share. In such
               situations there will usually be more experienced
               spellcasters that may teach their skills to junior
               members. Wizards may be children of existing wizards or
               they might be apprentices or students at an established
               wizard's school or place of business. Wizard secret
               societies are common as well. Some are devoted to
               particular schools of magic. Others are devoted to a
               certain cause and happen to share arcane secrets to persue
               common goals. Wizard secret societies and wizard schools
               are places where other wizards can learn new spells and
               find magical items. However, most magic secret societies
               have a strict hierarchy and code of conduct. Special
               oaths, tithes, and obedience to the order are often
               required, and favors always entail repayment.   Spells are purchased (learned) similar to skills.
               However, there are no levels for spells. A character
               either knows a given spell, or they don't. If you
               like you can think of this as a `level 1
               skill'.   There exist different schools (aka categories) of
               magic. Some Wizards specialize in certain schools, but
               that is purely optional. Any character may learn spells
               from any categorybut with the following
               restrictions. A character may only learn a given spell if
               she knows at least one spell of lower rank from the same
               school. As well, the rank of the spell must be no higher
               than your Spellcasting skill level. For example, you
               cannot learn a rank 4 `Air' magic spell unless
               you already know a rank 3 `Air' spell, AND you
               must have Level 4 or higher Spellcasting.   You can learn any spell, regardless of type, if you
               have a spellcasting skill at least three levels higher
               than the rank of the spell. For example, if you have a
               spellcasting skill level of 6, then you could learn any
               spell of rank 3 or lower, regardless of school or
               prerequisite lower-rank spell.   A character with level 1 or higher spellcasting can
               always learn Rank 1 magical spells from any school without
               restriction.   Note that a Wizard or Sorceress character must have
               at least one level in Magical Studies in order to read
               spell books and to learn the intonations required for
               magical incantations. Beyond that Magical Studies is a
               purely optional skill.   Characters that cast Invocation magic must have at
               least two free points of Encumberance available. If not,
               they are not able to cast magic. 
\subsection{Ritual Magic}
      Ritual magic is performed in elaborate
               `performances' involving a much human
               involvement. Magic rituals involve material components
               (ingredients) and a great deal of chanting, mumbling,
               gesturing, and the like. The traditional image of a witch,
               bent over a bubbling cauldron containing newt eyes and
               salamander toenails is a good example of ritual magic.
               Whereas traditional spellcasting is an almost scientific
               pursuit, ritual magic is more of an art. Rituals are
               conducted with a trained eye and experience. Because
               ritual ingredients vary in quality and potency, precision
               has little use. Ritual magicians are shamans, witches and
               warlocks.   Ritual magic is not religious. In fact, it is no
               more religious than making a bowl of oatmeal or mixing
               some cement. However, in many cultures ritual magic is
               heavily associated with religion, especially in primitive,
               animistic, or tribal societies. In these cultures ritual
               magic is the domain of the `witch doctor' or
               `medicine man'people who hold important
               positions in the tribe, often including that of a kind of
               priest or spiritual leader. Therefore, there is a strong
               association between ritual magic and primitive, pagan,
               religions. Many ritual magicians are religious figures,
               members of mysterious cults, etc. Some ritual magicians
               have been known to practice animal or even human sacrifice
               (an excellent, though horrific, source of MP for rituals).
               In these societies ritual magic is often believed to be
               religious in nature despite the fact that it is decidedly
               not.   Ritual magic users do not use innate MP to perform
               magic. Rather, they use material components (ingredients)
               to create magical effects. Performing rituals requires a
               large amount of space and various supplies and components.
               Ritual preparation and casting is a time consuming and
               often messy proposition. Therefore, most ritualists have a
               set area where they can store their materials and perform
               rituals: usually a quiet location away from prying eyes.
               This might be a purpose-built room in the home or a secret
               `lair' in a basement or abandoned building. It
               might even be a cave or hidden woodland grotto. Such an
               area is a very personal space for the ritualist, and while
               it might be neatly organized or highly messy, it will be
               arrayed exactly how its owner prefers.   Most ritual supplies are re-useable, and a
               ritualist's `kitchen' will be well
               stocked with such items. These include such things as
               wands, knives, scales, measuring devices, a fireplace,
               cauldrons, bottles, pots, strainers, mirrors, precious
               gems, candles, assorted tools, and so on. Approximately
               \$1000 * Rank worth of supplies is required in a
               ritualist's `kitchen'. (In other words,
               in order to perform the Ritual Stoneform, which is a rank
               6 spell, \$6000 worth of supplies are required to be on
               hand) These supplies are reuseable and are used for many
               different rituals.   Specific spells require different ingredients.
               Ingredients could be nearly anything, including herbs and
               roots, body parts (animal and human), minerals, chemicals,
               plants, and so forth. Some of these are common and will be
               used for many different spells. Others are quite specific
               and rare, and have very specific applications. A ritualist
               will usually have a personal stockpile of ingredients and
               components stored away. This stockpile will usually
               include all but the most expensive of components. Of
               course, some components can be difficult to find and some
               must be fresh, so any ritualist will definately have the
               means to locate components that she needs. Some components
               might be gathered personally, through the use of skills
               such as herbalism, alchemy, hunting, or scrounging. Other
               components are purchased by the ritualist. Some components
               may be purchased at common merchants while others must be
               purchased from special (and sometimes illicit) vendors.
               Most ritualists will have known connections through which
               different components and supplies might be shared,
               purchased, or loaned.   The cost of ingredients for a particular spell is
               approximately 1/10 of the `scroll or potion'
               cost from the spell chart. This assumes fair prices in an
               open market. However, this cost might vary greatly
               depending on various circumstances. For example, in a
               certain town components for Fire spells might be common
               and therefore very inexpensive, while components for Earth
               spells might be rare and therefore much more expensive.
                 In addition to its specific `recipe',
               any ritual spell will also require the infusion of mana.
               As mentioned above, in ritual magic the MP does not come
               from the caster. The caster must gather objects containing
               sufficient MP to conduct the spell. As mentioned above,
               mana is present in nearly everything, but generally not in
               sufficient quantity to do any good. Therefore, the
               ritualist must seek out items that are rich in MP to use
               when conducting a ritual. This process results in the
               destruction of the items used! This might even entail the
               sacrifice of living thingsif such an amount of MP
               is required. Generally, items yielding MP for ritual use
               will cost about \$1 per MP required, though this will vary.
               Much like individual spell components, a ritualist will
               tend to hoard items that may be sacrificed for MP. Common
               items are living plants and small animals, antiques, old
               artwork, fossils, old holy items, old clothing (especially
               from special occasions) and well-used household goods.
               MP-source items are not spell-dependant. MP items can be
               used for ANY spell.   
                  NO MP from the character are required for ritual
                 magic 
                  Each spell's components cost 1/10th of the
                 `scroll/potion' cost from the spell table,
                 PLUS \$1 for each MP required. Components are consumed
                 and can only be used once. (note that costs are
                 approximate and should be determined by the GM) 
                  The time required to prepare for a given ritual
                 is equal to Rank-1 hours. Once preparations are made,
                 the ritual must be conducted within 12 hours or the
                 preparations are wasted and must be begun again.
                 Preparation uses up one half of the required ritual
                 ingredients. 
                  The time required to perform a given ritual is
                 [Time] minutes. Conducting the ritual itself must be
                 done within 12 hours of preparations being made, and
                 uses up the remaining half of the ritual ingredients.
                 
                  Ritual spells cost 1 less Rank to learn than
                 normal 
                  Ritual magicians learn free rituals as they gain
                 levels in Ritual Magic, according to the chart below;
                 note that they are free to purchase other spells as they
                 desire. 
              
              Rank of Ritual Spell  
                
                  
                   Level of Ritual Magic   
                   1 
                   2   
                   3 
                   4   
                   5 
                   6   
                   7 
                   8   
                   9 
                   10   
                  
                  
                   1 
                   3   
                  
                  
                   2 
                   5 
                   2   
                  
                  
                   3 
                   7 
                   4 
                   2   
                  
                  
                   4 
                   8 
                   5 
                   3 
                   2   
                  
                  
                   5 
                   9 
                   6 
                   4 
                   3 
                   2   
                  
                  
                   6 
                   10 
                   7 
                   5 
                   4 
                   3 
                   2   
                  
                  
                   7 
                   11 
                   8 
                   6 
                   5 
                   4 
                   3 
                   2   
                  
                  
                   8 
                   12 
                   8 
                   7 
                   6 
                   5 
                   4 
                   3 
                   2   
                  
                  
                   9 
                   12 
                   8 
                   7 
                   7 
                   6 
                   5 
                   4 
                   3 
                   2   
                  
                  
                   10 
                   12 
                   8 
                   7 
                   7 
                   6 
                   5 
                   4 
                   3 
                   3 
                   1   
                  
                
                Rank of Ritual Spells per Ritual Magic Level
              Ritual magicians may also perform a ritual in a
               special manner, infusing the spell into an object, such as
               a weapon or magical totem. Infusing a spell takes the same
               amount of time as performing the ritual in normal fashion.
               Once the spell is infused in an item, it may be released
               very quickly. Releasing a charge (the energy and magical
               effects of a stored ritual) requires 1 action, regardless
               of the original casting time of the spell. Infused ritual
               items are only good for twenty-four hours, at which point
               the charge is lost and the item returns to normal. The
               ritualist may create charges of extended duration when he
               first performs the infusion, but each 24 hours of time
               requires another batch of the requisite spell ingredients
               (not Mana). Therefore, it can become very expensive to
               make long duration infused items. An infused item may be
               used by anyone, not just the ritualist. 
\subsection{Religious \& Faith Magic}
      Religious magic is magical effects that are brought
               into reality by the combined will of the caster and the
               caster's god(s). Whereas primitive religions and
               cults tend towards the use of ritual magic, Religious or
               Faith magic is practiced by priests of clearly defined and
               well organized religions. Most religions that practice
               Faith magic have a clearly organized hierarchy within the
               church. Religious dogma and/or scripture is well defined
               and is extensively documented. Faith-magic using religions
               are not necessarily `good', but they are
               usually known to the general public and often actively
               recruit new members.   Religious magic is taught by a particular Church.
               Each faith has one or more magic schools associated with
               it. These nearly always include Spiritual and Life magic,
               and may include others. Most faiths have 3 or 4 schools of
               spells. A faith caster has access to all spells in the
               school(s) associated with his Faith, except those whose
               rank is higher than his level of faith, and those which
               are specifically prohibited by his Church. For example, a
               `good' tending religion would typically deny
               harmful Life-magic spells to their adherents, for example
               curse.   
                MP are consumed as normal spellcasting   
                  A holy symbol is required for religious magic
                 casting 
                  A successful Faith check must be made for casting
                 faith magic. 
              Priests, monks, and warriors of faith practice
               religious magic. All religious spells are channeled
               through a holy symbol (see below).   Once a day the caster must engage in prayer. The
               exact means of praying is left up to the players \& GM,
               but during this time the priest will choose which spells
               he as access to for the remainder of the day. The caster
               may choose any combination of spells he has access to,
               provided that he has enough MP (at the time of prayer) for
               that particular combination. The caster must allocate MP
               and spells during this prayer session. Until the next
               prayer session those are the spells the caster may use. A
               holy symbol of at least +0 quality is required for this
               prayer, otherwise the caster cannot allocate spells for
               that day.   A faith magic user always uses the Faith skill for
               any and all magic-related skill rolls, including those
               that would normally require Spellcasting or Magical
               Control, etc.   A Holy Symbol is always required to cast Faith
               magic. If a holy symbol is not available, then Faith magic
               simply cannot be cast. Thankfully, a simple gesture or
               trinket often suffices. There are different types of holy
               symbols. `Better' and more significant holy
               symbols offer a better bonus to Faith magic casting:
                 
                
                  
                   Bonus   
                   Description   
                   Cost   
                  
                  
                   -1   
                     A gesture made by the Faithful, or a crude
                     marking (such as a cross hastily drawn on a
                     parchment) 
                  
                  
                   0   
                     An inexpensive `trinket' type holy
                     symbol, or a common copy of a holy book / scripture.
                     Generally mass-produced. Might be given away at a
                     particular church, or might be available for
                     purchase. Also a holy symbol made by a marginally
                     skilled craftsman or apprentice. Example: prayer
                     beads, a simple carved wooden cross, a mass-printed
                     holy book, etc. Cost: \$5 or more. 
                  
                  
                   1   
                     An unusually high quality trinket made by a
                     good craftsman (such as a fine gold crucifix
                     necklace). Or, a good quality weapon bearing
                     religious inscription and symbols. Or, an item from
                     a particular church which is not available to the
                     general public (such as the robes of an ordained
                     priest), or a fine quality holy book. 
                     Cost: At least \$500 plus the cost of the
                     `base item' 
                  
                  
                   2   
                     A trinket or jewelry made by a master
                     craftsman and of fine pedigree and history. Or, a
                     weapon or other object of superior quality with
                     appropriate engravings and blessings. 
                     Cost: at least \$5000 plus the base cost of the
                     item. 
                  
                  
                   3   
                     An artifact or relic of religious
                     significance, generally has a famous history within
                     the church, or a Legendary quality weapon blessed by
                     a member of the church with Faith: 7 or higher.
                     
                   Cost: \$1,000,000 or more.   
                  
                  
                   4   
                     An artifact or relic of tremendous religious
                     significance. Items like this are literally
                     worshipped in the particular religion. There will
                     only be a handful of items like this in existence
                     for any one religion at any given time (if any at
                     all). Examples: Spear of Destiny, Holy Grail, piece
                     of the true cross, skull of a major saint. 
                   Cost: priceless.   
                  
                
              Bonuses provided by a Holy Symbol  
              Anytime a Faith magic caster has to make a
               magic-related roll (`Magical Control' roll, or
               a strike roll associated with a magical spell, etc), use
               the caster's Faith skill. Whatever holy symbol that
               the caster has is then applied as a bonus to that roll.
               
\subsection{Martial Magic / Super Power Combat}
      Martial magic is the ability to perform
               supernatural `attacks' in conjunction with
               normal combat, such as fist fighting, martial arts, or
               combat with weapons.   Martial Magic always works in conjunction with some
               kind of verbal or somatic component. This means that the
               fighter must perform a particular combat form (kata),
               stance, or must shout a particular phrase in order to use
               the ability.   Martial magic is always taught from master to
               student. It is a closely guarded secret of warrior secret
               societies, secluded martial arts temples, or families with
               great fighting traditions. Once a warrior masters the
               traditional (non-supernatural) combat arts, he may be
               taught martial magic. Since martial magic is highly
               dependent on particular combat forms and stances, it is
               nearly impossible to learn on one's own or from a
               book.   Magical effects that ordinarily require touch do so
               for the martial magician as well. This is normally done by
               striking the target with some kind of melee attack. The
               character does not have to deal damage per se (if the
               target is wearing armor) but he must connect with a solid
               blow. The magical energy (effect) is transferred through
               the blow. Martial magic users may choose to use a normal
               touch instead.   Magical effects that are projectile in nature (such
               as Fireball) and normally require a strike roll must be
               used in conjunction with a missile or projectile weapon of
               some kind (thrown weapon, arrow, dart, or even a bullet).
                 Magical effects that are autotargeting when
               performed by a spellcaster (such as Lightning Bolt) are no
               longer autotargeting. The Super Power combatant must
               strike with any attack of his choice, which may be a
               ranged attack or a melee attack.   The effects of a martial magic spell are in
               addition to those normally associated with a strike in
               combat. If a martial artist punches a target and also
               performs the spell Vampiric Drain, then the target will
               suffer normal damage from the Punch, as well as the damage
               and effects from Vampiric Drain. Likewise, if a martial
               magician shoots a target with an arrow in conjunction with
               Energy Blast, then both the arrow and Energy Blast deal
               damage.   Martial magic works similar to spellcasting, with
               the following notes:   
                MP are consumed just like spellcasting   
                  Touch spells require a successful strike roll in
                 combat 
                  Ranged spells require a missile weapon of some
                 kind 
                  Auto-target spells are now `cast' by
                 a WIL check of the user 
                  Martial Magic spells are learned/purchased just
                 like normal Spells 
                  Martial Magic users must have at least 1 level in
                 Mysticism 
              Rather than purchasing the Spellcasting skill, a
               Super Power Fighter adds together his skill level of
               Martial Arts and Mysticism. The total is considered the
               skill level of `Martial Magic' 
\subsection{Mysticism / Psionics}
      Psionics is the ability to perform magic or
               supernatural acts by pure force of will. No incantations,
               materials, or special preparation is required. A character
               that can perform these feats is called a Mystic.   Whereas most magic users have trained to develop
               their abilities, Mystic characters are more often
               `gifted' with magical abilities. Quite often
               these powers manifest themselves as a result of a
               traumatic event, such as a near-fatal accident, that is
               experienced by the character. They may also be abilities
               that are passed down through a particular familyin
               which case the character is simply born with his unique
               powers. Mystic characters have an innate ability to
               perform their abilitieswhich are completely
               independent of any kind of structured `method'
               or `technique'. While mystic characters might
               join magical societies they cannot `learn' or
               `teach' mystic techniques to others. Psionic
               abilities are something that a mystic discovers herself.
                 Mystic rules:   
                MP are consumed just like spellcasting   
                  No verbal, somatic, or material components are
                 required. A mystic can be totally still and silent and
                 can sill use his abilities, though the time required to
                 perform (use) a mystic ability is exactly the same as
                 standard spellcasting. 
                  Spells are purchased from the spell list at will.
                 There is no need to follow a particular category. The
                 only restriction is that learned spells must be no
                 higher level than the character's Mysticism skill.
                 
                  Mystics always use the WIL attribute for
                 magic-related rolls, such as Magical Control, regardless
                 of the normal attribute that would be used for that
                 roll. 
            
\subsection{Beast Magic (summoning)}
      Beast Magic is a unique magic school. It is the art
               of summoning creatures to aid their master. Any magic user
               (spellcasting, ritual, etc.) may purchase summoning spells
               as normal. However, a spellcaster that specializes in
               Beast Magic (and has no other magical aptitudes) may
               purchase summoning spells as if they are one rank lower
               than normal.   There are five basic summoning spells:   
              
                  Gate   allows the summoner
                  to call forth a creature that appears somewhere near
                  the caster. The location of appearance must be within
                  direct line of sight of the caster, and no more than 30
                  feet away. Once the creature appears, it is free to do
                  whatever it wants based on its own free willthe
                  caster simply transports it to a new location. It will
                  return to its original location after 36rounds.
                
              
                  Summon   allows the summoner
                  to call forth a creature near the caster (like Gate).
                  This creature is bound to follow SIMPLE directives
                  given by the caster. The creature will not act in a
                  hostile manner towards the caster. If a summoner casts
                  another Summon spell while another one is currently
                  active, he must make a Magical Control roll. Failure
                  means that the new spell fails and the first expires
                  immediately.
                
              
                  Pact   allows the caster to
                  form a complex bond with a creature, making that
                  creature his Familiar. The caster must first find a
                  creature to use pact withsuch as summoning the
                  creature by gate or summon) The caster may
                  `control' the creature in a manner similar
                  to summon. This is permanent and lasts until either the
                  caster or the Familiar is slain. If the caster casts
                  Pact a while he already has a Familiar, the original
                  spell expires and the caster suffers 16+3U. While
                  active, the Familar is aware of what happens to the
                  caster, and vice versa. A basic empathic link exists
                  between the creature and the caster. The caster may
                  make a Magical Control check in order to `listen
                  in' on any of the familiar's senses at any
                  point in time.
                
              
                  Control   is very similar to
                  summon except that the caster has exacting and specific
                  control over the creature(s) summoned. The caster can
                  literally operate the creature(s) as though by
                  `remote control'. While Control is active,
                  the summoner is responsible for literally every action
                  made by the summoned creatureit has no will of
                  its own.
                
              
                  Transform   is a powerful
                  technique in which the caster takes the form of the
                  creature summoned. The caster literally becomes the
                  creature in question, but retains his own mental state
                  and skills in addition to the skills of the summoned
                  creature. All physical attributes (STR, DEX, BLD) of
                  the summoned creature replace those of the character,
                  but the character retains his own mental attributes
                  (INT, WIS, WIL, CHA). This effect lasts 4x 14turns.
                  It cannot be ended at will. The transformation requires
                  three actions to complete. When the spell expires, the
                  caster will be very tired, and will suffer the effects
                  of slow, loosing 1 action per turn for 14+1turns.
                
              These summon spells are used in conjunction with
               whatever creature `spells' a summoner knows.
               In order to learn a creature spell, the Summoner must have
               encountered such a creature. The rank of example creatures
               is shown below. For others, see the Beastiary section.
                 These summon spells are used in conjunction with
               whatever creature `spells' a summoner knows.
               In order to learn a creature spell, the Summoner must have
               encountered such a creature. The rank of example creatures
               is shown below. For others, see the Beastiary section.
                 
                
                  
                   Rank   
                     Spell (choose specifically)
                     
                  
                  
                   1   
                     Insects, songbirds, rats, mice, other small
                     rodents, frogs/toads, salamanders, misc. other small
                     reptiles or amphibians. 
                  
                  
                   2   
                     Dogs, cats, exotic small animals, larger birds
                     (crow, raven, owl, etc.), common snakes. Raccoon,
                     possum, etc. Bats, small flying/gliding mammals.
                     Wallaby. 
                  
                  
                   3   
                     Very large dogs, Cow, goat, sheep, deer,
                     antelope. Large predatory birds (hawk, eagle), large
                     reptiles (small crocodile, monitor lizard, large
                     snake). Lesser monkeys. Lynx, Kangaroo 
                  
                  
                   4   
                     Crocodile, Horse, Bull. Exotic 4-legged
                     animals. Large constricting snakes. Apes. 
                  
                  
                   5   
                   Hippo, Cape buffalo, Lion, Black Bear   
                  
                  
                   6   
                     Elephant, Rhinocerous, Grizzly or Polar bear.
                     Mammoth 
                  
                
              Summonable animals by Rank  
              For generic animals follow the table above. For
               very specific, exotic, or rare animals jump to the next
               higher rank. (Example: Summon Frog: rank 1. Summon
               Brazilian Poison-dart frog: rank 2.)   Whenever a summoning spell is cast the Summoner
               must make a Magical Control skill check. The negative to
               the roll is the total summoning rank(s) of the creatures
               to be summoned. A summoner can attempt to summon multiple
               of the same creature simultaneously, but not a mixture of
               different creatures. The caster has a bonus to this check
               based on what `Beast Lore' skills he knows.
               
\subsection{Magic-Related Skill Summary}
      
                Spellcasting gives additional     
                       LVL  2   
                        \ensuremath{\times}  WIS  MP
                           . This is the governing skill
                      for learning Invocation spells.
                
                  Magical Studies is the academic study of magic
                 and how it works. It also gives the player the ability
                 to identify magical items or effects (a simple skill
                 test), and to determine what a particular magical
                 item/effect does (GM determines difficulty of the test).
                 It also conveys the ability to read and write magical
                 script. This skill does not convey the ability to use
                 magic; this skill does not provide MP. 
                Magical Control gives additional     
                       LVL  2   
                        \ensuremath{\times}    WIL   
                      \ensuremath{\times}    2    MP   
                        , and is used to strike with and to control
                      the effects of some spells. While this skill is not
                      required for magic use, it is commonly used.
                
                Mysticism gives an additional     
                       LVL    3   
                        \ensuremath{\times}    WIS    MP
                           . This is the governing skill
                      for learning Mystic / Psionic spells.
                
                  Faith is not technically a magic skill. However,
                 it is used for Faith magic, and provides   
                       LVL    2   
                        \ensuremath{\times}    WIL   
                      \ensuremath{\times}    3  2
                         MP      . Faith
                      only provides MP to those characters who actually
                      know Faith magic spells. If a non-faith spellcaster
                      has this skill (For example, a Wizard that also
                      happens to be religious) then it does NOT provide
                      any MP.
                
                  Ritual Magic is used for Ritual Magic spells. No
                 character can learn a ritual spell with a higher rank
                 than her level of Ritual Magic. This skill does not
                 provide MP. 
              There exist different types or categories of magic.
               The different types of magic are:   
              
                  Air   Air magic is varied
                  and effective for combat mages. The lower rank spells
                  are a mixture between utility types and damaging
                  spells. The higher level combat spells are quite
                  powerful. Air magic contains Lightning spells, which
                  have wildly varying damage potential.
                
              
                  Life   Life magic is capable
                  of curing the wounded, healing the sick, and even
                  inflicting horrible diseases upon your opponents. It is
                  a favorite of religious orders, either for good or
                  evil. Most combat magic in this category is limited
                  because the spells require touch to cast. Some spells
                  are capable of dealing a great deal of damage, but
                  usually over a period of time. Sufficiently advanced
                  mages who practice life magic can raise the dead or
                  kill the living with a simple gesture.
                
              
                  Spiritual   Magic of the
                  Spiritual category is uncommon at best. With relatively
                  few low-rank spells it is usually reserved to advanced
                  practitioners of magic. However, it has some unique
                  spells that make it very effective. Spiritual magic is
                  particularly well suited for combating demons or
                  spirits. Higher level mages can lay waste to legions of
                  their enemies with single spell.
                
              
                  Light   With few exceptions,
                  Light magic is a utility category. It can be used to
                  illuminate dark passages beneath the earth or deceive
                  with illusion and tricks. Light magic has many useful
                  spells, many of which are available at low or middle
                  levels. It is a very useful type of magic for the rogue
                  or healer-type character who wants to be useful but
                  doesn't want to fight. A few high level light
                  spells can deal excellent damage, but only at a steep
                  MP cost.
                
              
                  Physical   Physical magic is
                  an excellent choice for `dual-classed'
                  characters that practice magic and also fight in melee
                  combat. At lower and middle levels, physical magic
                  allows one to improve the abilities of others, and to
                  manipulate the physical world with magic. He can also
                  repair broken weapons or mend machines. At higher
                  levels the school offers very effective defensive
                  spells and the highly effective gravity-based offensive
                  magic, which always deals solid damage.
                
              
                  Arcane   Arcane magic has no
                  direct offensive capability; this would appear to be a
                  weak form of magic that has little use for an
                  adventurer. However, the ability of arcane magic to
                  support, enhance, and empower the caster or his allies
                  is unmatched. It is excellent when paired with other
                  magic of a more offensive nature.
                
              
                  Fire   Fire magic is a
                  traditional offensive type of magic. It's spells
                  deal high damage and are without equal for causing
                  large-scale destruction. They are also quite
                  mana-efficient. However, fire magic is the easiest of
                  all `supernatural' attacks to defend
                  against. Fire magic is also noisy, flashy, and very
                  intimidating to one's opponents. Unfortunatley,
                  this makes it nearly impossible to use in a stealthy
                  manner.
                
              
                  Earth   Earth magic is a
                  varied school that combines defensive and offensive
                  spells. While neither type of spell is the most
                  powerful available, they are not to be underestimated.
                  Earth magic has a number of very useful utility spells.
                
              
                  Water   At lower levels,
                  water magic is fairly weak, though it does combine both
                  offensive and defensive spells. Middle ranks offer
                  effective protection and more effective combat spells.
                  Higher-level water magic is legendary in effectiveness
                  and is very powerful, Though many of the offensive
                  spells have a long casting time, they are nearly
                  impossible to defend against.
                
            
\section{General notes}
      Most spells automatically succeed (there is no skill
             check necessary, except for Faith Magic). Once cast, it is
             possible for some spells to miss their target however. For
             these spells, the player must make a skill test using
             Magical Control (or Faith) to determine if the strike is
             successful. See the spell listing for what spells are
             auto-targeting and which aren't.   If the player is interrupted while spell casting (e.g.
            takes damage, gets knocked around, etc.), he must make a
            skill check (Magical Control). If he fails the check, the
            spell fails, but the MP are not lost.
               The caster may not speak while spellcasting, but may
               move around, dodge, and perform very simple physical
               activities. 
            A spell always takes effect at the end of the casting
             period. The MP are consumed at the exact instant that the
             spell takes effect.   A casting time of zero (0) means that the caster may
             cast the spell anytime, even in response to something (e.g.
             getting shot at), or having no actions remaining at the
             time. Such spells require no command words, components, or
             gesturing of any kind.   Note that a Faith magic caster must always make a
             Faith roll in order to cast any magical spell. 
\subsection{Magic Points (MP)}
      Any character has innate MP; magic related skills
               will give additional MP.   MP are consumed whenever spells are cast. They are
               recovered according to the following chart:   
                
                  
                    
                    
                  
                  
                   Strenuous Activity: 
                   0 per hour   
                  
                  
                   Normal Activity: 
                   5 per hour   
                  
                  
                   Restful Activity: 
                   10 per hour   
                  
                  
                   Sleeping: 
                   20 per hour   
                  
                  
                   Meditation or prayer: 
                   20 per hour   
                  
                
              Mana Point replacement per hour.  
            
\subsection{Magical `Tools': Books, Scrolls, and
               Potions}
      There exist three different kinds of magic-related
               books.   Type 1 is a volume that contains detailed
               information on how to cast a particular spell (or spells).
               Books of this type are always written in magical script.
               Anyone with the Magical Studies skill can read and
               understand this kind of book. However, only those with
               Spellcasting can make real use of the contents. A
               Spellcasting character can easily learn new spells from a
               book of this type. Books of this type are very valuable.
               The cost of a spell book like this is equal to the
               `Book' price column of the magic table. If the
               book contains multiple spells, then its value is equal to
               the sum of the costs for each spell contained. Generally
               speaking, books of this type only contain multiple spells
               when they are compendia of low-level spells in a given
               school of magic.   Type 2 is a volume that describes how to perform a
               particular spell in ritual format. Books of this type
               might be written in any language, and are generally not
               written in magical script. If a Ritual magic using
               character can read a book of this type, it is something
               like a cookbook for a particular ritual spell. The
               character could keep the book as a reference or could use
               it to learn the ritual described. Ritual spell manuals are
               worth approximately one-quarter the cost of a Type 1 book.
                 Type 3 is a book that describes other forms of
               magic. Since Faith magic, Martial Magic, and Mysticism are
               dependant on the individual `casting' the
               spell and not on some external `formula' or
               `recipe', it is essentially impossible to
               truly document how to perform a given spell with nothing
               more than a book. These books are of interest to sages and
               scholars who are interested in arcane subjects. However,
               they are little more than a curiosity to actual
               practitioners of magic. Given enough books of this type it
               is possible for a sufficiently advanced magical
               practitioner to figure out a new spell or technique, but
               it is no guarantee. Books of this type are worth
               approximatley one-tenth of the listed price, depending on
               the exact subject matter contained within the volume. They
               may be written in any language.   Magical potions or scrolls are single-use magic
               spells. Anybody can use a potion or scrollwhether
               they know anything about magic or not. Spells that affect
               the caster or a willing recipient are potions. Drinking
               the potion causes the spell to take effect. Other spells
               take the form of scrolls. Reading the scroll aloud causes
               the spell to take effect. Both scrolls and potions are
               single-use only. Potions can be drunk in one action.
               Scrolls ALWAYS take 3 actions to use regardless of the
               spell.   A person with Magical Studies can create Books,
               Scrolls, and Potions of any spell he/she knows. Producing
               such works is a long and expensive process:   
              
                  Books   Requires a number of
                  days equal to twice the rank of the spell squared to
                  create. No MP are required, but special precious inks
                  and paper are required. These materials cost
                  approximately half the value of the book.
                
              
                   Scrolls/Potions   Require
                     
                       LVL  2 
                        hours       to create.
                      They require MP equal to a normal casting of the
                      spell to be expended, and require precious
                      materials valued at not less than three-quarters
                      the value of the finished item. A ritualist may
                      prepare a potion in half that time.
                
            
\subsection{Magic Terminology \& Misc.}
      
              
                  Black   spells are those
                  associated with evil, death energy, and demonic
                  influence. While they are not necissiarily evil, they
                  are generally indescriminate in their targets. Black
                  spells never affect demons, undead, or targets that are
                  not living, such as ghosts or robots.
                
              
                  White   spells are those
                  that are comprised of life energy. They are technically
                  not `good' aligned, though they are a
                  favorite of `good' religious orders. White
                  spells that cause damage only affect demons, undead,
                  and spirits.
                
              
                  Auto   spells are those that
                  are automatically targeting. That means that once cast
                  the spell will never miss or malfunction. However, in
                  order to cast an auto-targeting spell, the caster must
                  be able to clearly see and identify his target.
                
            
\section{Schools(???)}
    
\subsection{Air}
    
\subsection{Arcane}
    
\subsection{Beast}
    
\subsection{Earth}
    
\subsection{Fire}
    
\subsection{Life}
    
\subsection{Light}
    
\subsection{Physical}
    
\subsection{Spirit}
    
\subsection{Water}
    
\chapter{Martial Arts}
    
\section{Rules}
      Martial Arts covers the use of all weapons (ranged,
             melee, ultratech, mecha, et al.), in addition to
             hand-to-hand (HtH) combat, and the set of specialized skills
             that are associated with the Martial Arts governing skill.
             The maximum level that any character may have in any Martial
             Arts skill is less than or equal to his level in Martial
             Arts, i.e. if he has 3 Levels in Martial Arts, then all of
             his Martial Arts skills must be Level 3 or less. The basic
             Martial Art skill confers upon the martial artist the
             ability to purchase the different skills listed in this
             section.   For each level the martial artist has in Martial Arts
             they may choose one Martial Arts skill; for this skill, any
             time the martial artist would have to roll the skill, they
             get an extra die to do so. A skill may only be chosen
             Level-in-Martial Arts/2 times (the Lvl in Martial Arts,
             divided by two, rounded-down). 
\quotexample[Kim improves his Nyrkkeily]{  Kim NGoc Goay is an expert in Norse Boxing
               (Nyrkkeily), he chooses, when he gets his first level in
               Martial Arts, he also buys a level in
               `Punch/Kick;' in particular, he also assigns
               his level-bonus to `Punch/Kick.' This means
               when he rolls to strike with a punch he rolls two dice and
               not one. }
    After the Martial Arts skill has been purchased,
             skills from the four basic categories may be acquired; the
             four categories represent different capabilities that
             Martial Arts covers. The five categories are Basic, Combat,
             Defensive, and Modifier. The different categories focus on
             different phases of Combat, such that Combat skills are
             designed to aid the Martial Artist with movement,
             entering/exiting combat, the knowledge of combat,
             deployment, and higher-level strategy. Basic skills are
             involved in the nitty-gritty (basic) postures and moves to
             actually strike the opponent and inflict harm or death.
             Defensive skills are designed to deflect, dodge, and/or
             otherwise avoid the mishap of being harmed by another
             pugilist. Modifiers represent ways to modify Basic skills.
               
            
                Basic   These skills represent
                Basic combat actions, such as punching or kicking,
                grappling, the use of weapons and other primitive
                attacks.
               Only Basic attacks may be modified by a Modifier.
               
              
            
                Combat   These are the skills
                for being in, around, or preparing for combat. For
                example, knowledge of the terrain, etiquette, being
                prepared to move quickly and with assurance.
              
            
                Defensive   These are the
                skills used for defending against an attack. They are the
                foil to the Basic skills.
              
            
                Modifier   A Modifier is a
                skill used to modify a Basic attack. A Modifier changes
                the behaviour of a Basic attack; the Basic attack is now
                considered a Special attack of the same name with the
                Modifier as the adjective, for instance, if the Martial
                Artist is modifying a Punch with a Bash, then this is now
                a Special attack, `Bashing Punch.' Since a
                modified Basic attack counts as a Special attack, it may
                not be further modified, as only Basic attacks can be
                modified. There are certain Combat skills which enable
                the Martial Artist to consider certain specific
                combinations of Modifier/Basic to be Basic attacks.
              
           A Modifier may not be used more than Lvl times per
             turn. 
\quotexample[Candlestick modifies his punch]{  Candlestick has two levels in Martial Arts; he has
               bought Punch at rank 3 and level 3, and chosen it as his
               bonus; he also has Bash at level 1 and Feint at level 1.
               He is striking at the Giant Hispaniola Solenodon with a
               Bashing Punch. Because he is using Bash with his Punch his
               punch is now considered `special' and may not
               be further modified, even though he could still use his
               Feint this turn. }
  
\section{Further Commentary}
      You will note that there is no `chi' or
             `super-powers' in Martial Arts. In fact, the
             skills are deliberately kept pedestrianthose
             possessed (or assumed to be possessed, if possible) by the
             greatest `non-mystical' martial artists in the
             real world.' This is done deliberately to
             provide a clear separation between the
             Supernormalgoverned by Magic and Traitsand the
             mundane reality of kicking, spitting, biting and hurting. If
             you want to create a character that has truly jaw-dropping
             magical abilities, like flying-through the air, or punching
             through a fired-brick wall, bending steel, or leaping from
             roof-top-to-roof-top, consider looking into the Magic and
             Traits section. In Magic there a several categories that
             would benefit a Martial Artist tremondously. 
\subsection{Advanced Martial Arts}
      Also there are several para-skills which are
               tremondously important to the dedicated Martial Artist:
               Kata, Kung Fu, Waza, and Jitsu. These skills allow the
               Martial Artist to transform his very Basic and Defensive
               attacks, allowing him to perform prodigious feats: a
               Savage Feinting Death Strike is only possible by using
               these skills. 
\section{Categories}
    
\subsection{Basic}
    
\subsection{Combat}
    
\subsection{Defensive}
    
\subsection{Modifier}
    
\subsection{Special}
    
\chapter{Mecha}
    
\section{Mecha Suits \& Powered Armor}
      A robot in which the pilot is completely protected by
             armor and other mechanisms.Mecha are controlled by abstract
             means and requires the Mecha Piloting skill.   Mecha denotes armored mechanical vehicles that are
             built in the form of a human. There are two basic types of
             mecha: Power Armor and Mecha.   Power Armor is an armored suit that is
             `worn' like clothing. It provides armor
             protection, but it has the function of increasing the
             wearer's strength, endurance, and weapon payload. It
             is easy to use. The internal computers, etc, in the suit
             allow it to follow the wearer's movements.   
               Power Armor does NOT require any skill to use
               
               Power Armor must be either the same size as the
               wearer, or exactly one level larger 
            True Mecha is best thought of as a vehicle, like an
             airplane or tank, which is often bipedal. Mecha is piloted,
             not worn like power armor. A mecha has a cockpit, and is
             controlled with levers, switches, and pedals. Piloting a
             mecha is something like flying an airplane or driving a car.
               
               The Mecha Piloting skill is required to operate a
               mecha 
               When piloting a mech, any skill you use is capped at
               your level of mecha piloting. So, if you had LVL 5 Dodge,
               but have a Mecha Piloting skill of 4, you would roll 4
               dice to dodge in your mech. 
               Mecha must be AT LEAST one size level larger than
               the pilot. Beyond that, they can be ANY size. 
            As a general summary: Power Armor tends to be a
             lighter and smaller than mecha. Power Armor can carry less
             weaponry (and other hardware), but is usually more
             manueverable. It is often possible to perform major feats of
             dexterity while wearing power armor. This is because the
             power armor does not hinder the wearer's natural
             skills. As well, power armor is easier to use because the
             Mecha Piloting skill is not required.   Mecha tend to be larger and more powerful than power
             armor. Mecha generally have more and heavier weapons, better
             armor, and superior auxiliary systems compared to powered
             armor. However, the pilot needs to be skilled to maximize
             the potential of a mecha. 
\subsection{Building a Mecha}
      
                
                Size and Starting Costs Step  
                  
                    
                      
                       Size   
                       1 
                       2 
                       3   
                       4 
                       5 
                       6   
                       7 
                       8 
                       9   
                      
                      
                       Height   
                       4'   
                       7'   
                       15'   
                       30'   
                       60'   
                       125'   
                       250'   
                       500'   
                       1000'   
                      
                      
                       Mecha Weight (tons)   
                       0.5   
                       1   
                       2   
                       16   
                       125   
                       1000   
                       8000   
                       65000   
                       500000   
                      
                      
                       Mecha CP Cost   
                       5   
                       10   
                       20   
                       40   
                       80   
                       160   
                       320   
                       640   
                       1280   
                      
                      
                       Power Armor CP Cost   
                       1   
                       4   
                       9   
                       16   
                       25   
                       50   
                       100   
                       200   
                       400   
                      
                    
                  Size Level and Starting CP Cost  
                  
                  
                     Mecha can be no smaller than one size level
                     greater than the wearer's size 
                     Humans are size level 2, weight is given in
                     tons 
                     Height and weight indicates the maximum size
                     for a humanoid shaped robot for a given size level
                     
                  
                
                
                Starting Values  
                  
                    
                      
                       Size   
                       1 
                       2 
                       3   
                       4 
                       5 
                       6   
                       7 
                       8 
                       9   
                      
                      
                       Mecha Slots   
                       10   
                       20   
                       40   
                       80   
                       160   
                       320   
                       640   
                       1280   
                       2560   
                      
                      
                       Power Armor Slots   
                       8   
                       20   
                       35   
                       60   
                       100   
                       190   
                       330   
                       575   
                       1000   
                      
                      
                       Mecha Armor   
                       5   
                       10   
                       20   
                       30   
                       40   
                       50   
                       60   
                       70   
                       80   
                      
                      
                       Power Armor Armor   
                       2   
                       6   
                       12   
                       20   
                       30   
                       40   
                       50   
                       60   
                       70   
                      
                      
                       Dex Minus   
                       -1   
                       -2   
                       -3   
                       -4   
                       -5   
                       -6   
                       -7   
                       -8   
                       -9   
                      
                      
                       A. Penalty   
                       3   
                       5   
                       10   
                       15   
                       20   
                       25   
                       30   
                       35   
                       40   
                      
                      
                       Hit Points   
                       75   
                       150   
                       300   
                       600   
                       1200   
                       2500   
                       5000   
                       10000   
                       20000   
                      
                      
                       Strength   
                       5   
                       10   
                       20   
                       40   
                       80   
                       160   
                       320   
                       640   
                       1280   
                      
                      
                       Power   
                       2   
                       5   
                       10   
                       20   
                       40   
                       80   
                       160   
                       320   
                       640   
                      
                    
                  Starting Values for Statistics  
                  
                  
                  
                     Slots 
                      The amount of equipment that can be installed
                     on the suit 
                    
                  
                     Armor 
                      Starting armor value of the suit (All equal
                     for P/S/C) 
                    
                  
                     Dex Minus 
                      Robots have some control-lag, this is
                     represented by a DEX minus to the pilots DEX
                     attribute 
                    
                  
                     A. Penalty 
                      Armor Penalty. For every full amount of this
                     value, the DEX of the pilot is reduced by 1. 
                    
                  
                     Power 
                      is the number of PC per turn consumed by the
                     robot, just to run basic needed systems 
                    
                  
                     Hit Points 
                      The starting number of HP that a mecha has
                     
                    
                  
                     Strength 
                      The starting strength value for the mecha
                     
                    
                  
                
                
                Frame Style  
                 This section is for Mecha only!   
                  
                    
                      
                       Frame Material (Added HP)   
                       Cost Factor   
                       Very Light (-20\%)   
                       Light (-10\%)   
                       Medium (0\%)   
                       Heavy (+10\%)   
                       Very Heavy (-10\%)   
                      
                      
                       Carbon Composite   
                       -20\%   
                       XXXX   
                       XXXX   
                       XXXX   
                       XXXX   
                       0   
                      
                      
                       Aluminum Alloy   
                       -10\%   
                       XXXX   
                       XXXX   
                       XXXX   
                       0   
                       20   
                      
                      
                       Standard Steel   
                       +0\%   
                       XXXX   
                       XXXX   
                       0   
                       2   
                       40   
                      
                      
                       Heavy Steel   
                       +20\%   
                       XXXX   
                       0   
                       2   
                       4   
                       60   
                      
                      
                       Duralimin Alloy   
                       +40\%   
                       0   
                       2   
                       4   
                       6   
                       80   
                      
                      
                       Matrix Composite   
                       +60\%   
                       2   
                       4   
                       6   
                       8   
                       100   
                      
                      
                       Adamantite Alloy   
                       +80\%   
                       4   
                       6   
                       8   
                       10   
                       120   
                      
                      
                       Gundanium Alloy   
                       +100\%   
                       6   
                       8   
                       10   
                       12   
                       140   
                      
                    
                  Frame Style  
                  
                    The number of HP and Slots that a mecha has can
                   be further altered by the internal frame's style
                   and material makup. Frame Style determines the
                   efficiency of how bulky the internal frame of the
                   mecha is, this adjusts the number of slots that are
                   within the mecha. Frame Material is the makeup of the
                   frame, what it is made of. Different materials offer
                   decreased or added hit points to the mecha. Choose the
                   type of Frame Style, as well as the Material. Cross
                   reference them on the chart below, and multiply the
                   number by the size of the mecha to get the CP cost for
                   that configuration. 
                    For example, if you chose a `Light Matrix
                   Composite' frame for a size 4 mecha, it will
                   receive -10\% to it's slot total, would gain +60\%
                   to it's hit points, and cost 16 CP (size of
                   4*4). 
                    If a mecha suit runs out of HP, it is rendered
                   useless until repaired. If it goes -HP then it is
                   destroyed. Any time the mecha takes 10\% of it's
                   total HP in one blow, the pilot takes 16+1U damage.
                   
                
                  Starting armor is already listed above, the mecha
                 suit starts with this value in P/S/C. Additional armor
                 is purchased by spending 1 CP per +1/+1/+1 armor value.
                 Different types of armors can be used on the mecha suit.
                 These different types modify the ratio of the three
                 armor values. Chose one, but they cannot be mixed.
                 
                
                Armor  
                  
                    
                      
                       Type   
                       Ratio of P/S/C   
                       Description   
                      
                      
                       Normal   
                       100\% / 100\% / 100\%   
                       Standard protective armor   
                      
                      
                       Reactive   
                       80\% / 80\% / 140\%   
                         Armor will detonate against all attacks,
                         blunting the blow of C damage 
                      
                      
                       Ultrahard   
                       130\% / 90\% / 80\%   
                         Extra hard armor fragments piercing
                         weapons 
                      
                      
                       Slick-steel   
                       90\% / 120\% / 90\%   
                         Slashing attacks tend to slide off the
                         armor 
                      
                      
                       Reflective   
                       70\% / 70\% / 70\%   
                          Armor will protect against energy based
                         attacks as if the attack was not U 
                      
                    
                  Armor  
                  
                
                
                Engines etc.  
                    Various parts of a mecha or powered suit
                   require power to operate. This can come from any
                   number of different sources. A mecha will generally
                   have an ENGINE of some sort that supplies power for
                   moving around, and operating equipment \& weapons.
                   Some mecha will have supplimentary power sources used
                   for especially power-hungry equipment. All power
                   sources take up slots within the suit based on the
                   amount of power that it produces. The more expensive
                   (in CP) a power system is, the more power can be
                   gotten from one slot's worth of the power
                   source. Power is measured in PCs (or Power Cells). All
                   power sources produce the number of listed PC at the
                   beginning of each of the character's turns. All
                   systems on a suit will draw power from this, and if
                   there isn't enough power available, that piece
                   of equipment will not function. So it is vital that
                   the suit has adequate power. 
                  
                    Engines (provide constant power while mech is
                     operating)
                    
                      
                       System   
                       PC / slot   
                       CP / slot   
                       Description   
                      
                      
                       Solar   
                       2   
                       .5   
                         Solar panels on the suit provide power. No
                         light = no power 
                      
                      
                       Fuel   
                       4   
                       1   
                         A standard fuel powered engine. Easily
                         accessible fuel (gasoline). 
                      
                      
                       Turbine   
                       6   
                       2   
                         A high efficiency fuel engine. Fuel is
                         hard to come by and expensive. 
                      
                      
                       Fission   
                       8   
                       4   
                         A small nuclear reactor provides power.
                         
                      
                      
                       Fusion   
                       10   
                       6   
                         Most efficient, and most expensive.
                         
                      
                    
                  Power Sources  
                  
                    When turned on, booster engines provide power
                   every turn for three turns in a row. They cannot be
                   shut offonce they are turned on. They must be refueled
                   before they can be used again. These are useful for
                   emergency applications or powering large engergy
                   weapons in combat. 
                    Booster Engines (provide temporary power for 3
                   turns only before refueling) 
                  
                    
                      
                       System 
                       PC / slot   
                       CP / slot 
                       Description   
                      
                      
                       LPG   
                       7   
                       1   
                         Liquified Petroleum Gas micro-turbine
                         
                      
                      
                       Alcohol   
                       10   
                       2   
                         Alcohol-driven high-speed rotary generator
                         
                      
                      
                       Hydrazine   
                       15   
                       3   
                         Hydrazine-Gasoline fueled generator.
                         
                      
                      
                       LOX   
                       22   
                       4   
                         Liquid-Oxygen / Aluminum fueled generator.
                         
                      
                    
                  Booster Engines  
                  
                    Power cells store energy like a rechargeable
                   battery. They contain a certain amount of power, which
                   can be drawn from them at will. Once the power is used
                   up, they must be recharged before they can be used
                   again. Better power cells can store more energy in a
                   smaller amount of space. 
                    Power Cells store power like a battery; can be
                   recharged. 
                  
                    
                      
                       System   
                       PC / Slot   
                       CP / Slot   
                       Description   
                      
                      
                       Standard   
                       50   
                       1   
                       Simple capacitors store power   
                      
                      
                       HG   
                       80   
                       2   
                       special high grade power cell   
                      
                      
                       HD   
                       120   
                       3   
                       High-density power cell   
                      
                      
                       Super   
                       170   
                       4   
                       Most efficent power cell   
                      
                    
                  Power Cells  
                  
                
                
                Agility and Strength Modification  
                  
                    
                      
                       Kind   
                       DEX Bonus   
                       STR Bonus   
                       Caveats   
                      
                      
                       Mecha   
                                   2
                            \ensuremath{\times}    Size CP  
                              +1    
                             
                        
                                 (   
                         Size    -    1    )
                            CP        +1
                               
                        
                         DEX can only be used to reduce the DEX
                         minus to a +0. 
                      
                      
                       Powered Armor   
                                   3
                            CP        +
                            (    Size    -
                            1    )    
                             
                        
                                   (
                            2    \ensuremath{\times}   
                         Size Cp    )      
                          +1         
                        
                         Power armor STR bonus increases the
                         wearer's own STR while wearing the armor
                         
                      
                    
                  DEX and STR modification.  
                  
                
                
                Movement Systems  
                    A mecha suit or power armor needs to move. All
                   movement systems use power. Unless otherwise stated,
                   the systems use their power only when activated (being
                   used). 
                  
                    
                      
                       Type 
                       Speed   
                       Altitude 
                       Power   
                       Slots 
                       CP 
                       Notes   
                      
                      
                       Legs   
                       20 mph   
                             5    \ensuremath{\times}
                            Size   
                        
                             (    Size  
                          -    1    )   
                         \ensuremath{\times}    6   
                        
                       0   
                       4   
                      
                      
                       Wheels   
                       50 mph   
                       0   
                             (    Size  
                          -    1    )   
                         \ensuremath{\times}    2   
                        
                       1   
                       3   
                      
                      
                       Treads   
                       30 mph   
                       0   
                       \$(Size-1)*4\$   
                       2   
                       4   
                      
                      
                       Micro-hover   
                       20 mph   
                       15'   
                             (    Size  
                          -    1    )   
                         \ensuremath{\times}    10   
                        
                       4   
                       8   
                      
                      
                       True-hover   
                       30 mph   
                       200'   
                             (    Size  
                          -    1    )   
                         \ensuremath{\times}    15   
                        
                       6   
                       10   
                      
                      
                       Jump jets   
                       0 mph   
                       100'   
                             (    Size  
                          -    1    )   
                         \ensuremath{\times}    10   
                        
                       1   
                      
                        
                          
                            Size2
                          
                        
                        
                      
                      
                       Mech. Jump   
                       0 mph   
                       50'   
                       0   
                       2   
                      
                           Size  2 
                          
                        
                      
                      
                       Manuever Jets   
                       40 mph   
                       5'   
                             (    Size  
                          -    1    )   
                         \ensuremath{\times}    20   
                        
                       1   
                               3 
                         \ensuremath{\times}    (    X  
                          \ensuremath{\times}    1    )
                              4  
                         
                        
                       Gives [+2X] to dodge   
                      
                      
                       Jet System   
                       200 mph   
                       Unlimited   
                             (    Size  
                          -    1    )   
                         \ensuremath{\times}    20   
                        
                       10   
                       15   
                      
                      
                       Added Speed   
                       +15 mph   
                       n/a   
                       n/a   
                       n/a   
                       1   
                       Not for jet system   
                      
                      
                       Added Speed   
                       +50 mph   
                       n/a   
                       n/a   
                       n/a   
                       1   
                       For jet system only   
                      
                    
                  Movement Systems  
                  
                
                
                Equipment  
                  Purchase equipment.   
                
                
                Weapons  
                    Purchase weapons from the generic weapons
                   sheet. 
                   A mecha suit can weild any hand-held weapon that
                   is no bigger than STR/2 slots. 
                
            
\section{Equipment}
    
\subsection{Auxiliary}
    
\subsection{Ballistics}
      Ballistic weapons are scaled firearms or missiles
               (not energy weapons). The following chart details the
               information for various sized ammuntion for firearm-like
               ballistics weapons.   
                
                  
                   Class   
                   Size   
                   Slots   
                   Capacity per   
                   Damage   
                   Cost   
                  
                  
                   0   
                   < 20mm   
                   1   
                   500 rounds   
                   As per normal   
                   Varies   
                  
                  
                   1   
                   20 mm   
                   1   
                   20 rounds   
                   2410P   
                   \$10   
                  
                  
                   2   
                   30 mm   
                   1   
                   10 rounds   
                   2610P   
                   \$20   
                  
                  
                   3   
                   40 mm   
                   1   
                   2 rounds   
                   24+110P   
                   \$50   
                  
                  
                   4   
                   50 mm   
                   1   
                   1 round   
                   3610P   
                   \$75   
                  
                  
                   5   
                   75 mm   
                   2   
                   1 round   
                   4610P   
                   \$100   
                  
                  
                   6   
                   100 mm   
                   3   
                   1 round   
                   44+110P   
                   \$150   
                  
                  
                   7   
                   125 mm   
                   4   
                   1 round   
                   4420P   
                   \$200   
                  
                  
                   8   
                   150 mm   
                   5   
                   1 round   
                   5420P   
                   \$400   
                  
                  
                   9   
                   200 mm   
                   6   
                   1 round   
                   5620P   
                   \$800   
                  
                
                Sizes, costs, capacity for firearm ballistic
                 weapons.
              
              
                 Class 
                  the size of the weapon being bought; used to
                 calculate various expenses 
                
              
                 Size 
                  the size of the mecha or powered armor 
                
              
                 Slots 
                  the number of slots a `pod' of
                 ammunition uses 
                
              
                 Capacity per 
                  the number of rounds per pod 
                
              
                 Damage 
                  the damage the ammunition does 
                
              
                 Cost 
                  the cost in dollars to purchase the ammunition
                 pod 
                
            
\subsection{Energy Weapons}
      Energy weapons draw energy straight from the
               engines or batteries. There are multiple different methods
               for producing the blast that comes from an energy weapon,
               but all of them share some things in common. First, the
               base damage-table is used to calculate the damage an
               energy weapon does based on it's class:   
                
                  
                   Class   
                   Energy Units   
                   Damage 
                   C.P.   
                  
                  
                   0 
                   5   
                   56P 
                   2   
                  
                  
                   1 
                   10   
                   1610P 
                   4   
                  
                  
                   2 
                   15   
                   2610P 
                   6   
                  
                  
                   3 
                   20   
                   4610P 
                   8   
                  
                  
                   4 
                   25   
                   8610P 
                   10   
                  
                
                Sizes, number of die with multiplier, and
                 energy usage for energy weapons.
              Energy weapons use Class+2 slots.   All Energy weapons can be `tuned' for
               various effects (one or the other), which takes (Class+1)
               hours to do in a professional shop. Unless otherwise
               stated all weapons have a range of (Class+3)*10000 feet.
                 
              
                 OverFire   
                  spends twice as much energy but deals 50\\% more
                 damage 
                
              
                 Flame   
                  spends twice as much energy but also causes the
                 target to suffer the damage as U (Undefined) Fire
                 
                
              
                 Drawing   
                  spends three times as much energy, deals S
                 (Slashing) damage instead of P 
                
              
                 Pulse   
                  spends 1/2 as much energy, use four-sided dice
                 (D4) instead of 6. 
                
              
                 Higgs Particle   
                  weapon deals C (Crushing) damage, but the maximum
                 range is (Class+1)*100 feet 
                
              
                 Area Effect   
                  uses twice the energy, but has (Class+1)*5'
                 area effect 
                
              When buying an energy weapon you may pay it's
               cost again to add an effect. Certain combinations of
               effects are `named' below.   After every twenty shots an Energy weapon must be
               retuned or else it will suffer damage. For every five
               shots after the twenty-limit, roll a 120with a minus
               equal to the number of shots, plus (Class+1)*2. If the
               die-roll fails, the weapon is slagged and requires
               (Class+1)*10 hours of repair in a professional shop.
               
\subsection{Weapons}
      Melee weapons for Mecha and Powered Armor are
               simply `scaled' up Melee weapons. The
               following table details the rule for scaling
               `human' sized weapons for Mechanized Battle
               Armor. For the Damage multiplier listed below, this is a
               multiplier to the number of dice.   
                
                  
                   Size 
                   C.P. Cost 
                   Slots 
                   Damage 
                   Max.Str.Bns. 
                   Min.Str.Req.   
                  
                  
                   2   
                   x1   
                   0   
                   x1   
                   x1   
                   x1   
                  
                  
                   3   
                   x2   
                   0   
                   x2   
                   x2   
                   x2   
                  
                  
                   4   
                   x3   
                   0   
                   x4   
                   x4   
                   x4   
                  
                  
                   5   
                   x4   
                   0   
                   x8   
                   x8   
                   x8   
                  
                  
                         X    >  
                      2   
                    
                         \ensuremath{\times}      X
                        -    1    
                     
                    
                   0   
                       \ensuremath{\times}   
                             2      X  
                            -    2   
                            
                    
                       \ensuremath{\times}   
                             2      X  
                            -    2   
                            
                    
                       \ensuremath{\times}   
                             2      X  
                            -    2   
                            
                    
                  
                  
                   Assisted Melee System   
                    
                         X    2   
                        
                    
                   Size   
                   +X to damage   
                   Uses 5x power.   
                  
                
              Mecha and Power Armor Melee Weapons  
            
\subsection{Sensors and Controls}
    
\chapter{Monsters}
    
\section{Building a Monster}
     Because of the wide variety of monstrous creatures
             available in literature and fantasy, it is a somewhat futile
             task to attempt to construct an exhaustive list. In
             addition, most GameMasters find that while any particular
             monster is okay, the specific monster they are envisioning
             is not available to them. The Monster section is thus
             divided into three parts:   
               : the fundamental
               attributes, abilities, and `look' of the base
               creature 
               : modifications to the
               base-type that give the `flavor' of the
               monster 
               : that have been
               pre-generated 
          
\subsection{Base Types}
     The base-types are the fundamental
               `type' or `look' of the monster,
               they are:   
                  essentially symmetric,
                 bipedal, a head (and possible tail) etc.; a human-like
                 monster 
                  covers non-anthropoid
                 bipedal monsters, e.g. Duck, Chipmunk, Wyvern (two legs,
                 two wings) 
                  typical of regular
                 animals (mammals), e.g. Dog, Dragon (two legs, two arms,
                 two wings) 
                tubes, e.g. Snail   
                  monsters with many legs
                 or many arms, e.g. Angelus Guhendrian (torus with twelve
                 wings, and an eyeball) 
                  blobs, pools, puddles,
                 whiffs-of-smoke 
             When constructing a monster, the base-type must be
               chosen. Based on the base-type, the size of the creature
               is chosen. The table below lists the fundamental physical
               attributes of the various base-types.   
                
                  
                    
                    
                    
                    
                    
                    
                    
                    
                    
                    
                    
                    
                  
                  
                  
                    The fundamental base-types for monsters
                    
                  
                  
                   Type   
                   1/2   
                   1   
                   2   
                   3   
                   4   
                   5   
                   6   
                   7   
                   8   
                   9   
                   10   
                  
                  
                   Anthropoid   
                    Height:     
                         Size  2 
                             \ensuremath{\times}    40 
                           cm     
                    
                  
                  
                    
                      Weight:     
                     Height  3    kilograms
                         
                    
                  
                  
                   Str   
                  
                    
                    
                   4   
                   9   
                   16   
                   25   
                   36   
                   49   
                   64   
                   81   
                   100   
                   121   
                  
                  
                   Bld   
                  
                    
                    
                   4   
                   9   
                   16   
                   25   
                   36   
                   49   
                   64   
                   81   
                   100   
                   121   
                  
                  
                   Dex   
                   8   
                   8   
                   8   
                   8   
                   8   
                   8   
                   8   
                   8   
                   8   
                   8   
                   8   
                  
                  
                   Bipedal   
                    Height:     
                         Size  2 
                             \ensuremath{\times}  35  cm
                               
                    
                  
                  
                    
                     Weight:     
                     Height  3    kilograms
                         
                    
                  
                  
                   Str   
                  
                    
                    
                   4   
                   9   
                   16   
                   25   
                   36   
                   49   
                   64   
                   81   
                   100   
                   121   
                  
                  
                   Bld   
                  
                    
                    
                   4   
                   9   
                   16   
                   25   
                   36   
                   49   
                   64   
                   81   
                   100   
                   121   
                  
                  
                   Dex   
                   9   
                   9   
                   9   
                   9   
                   9   
                   9   
                   9   
                   9   
                   9   
                   9   
                   9   
                  
                  
                   Quadruped   
                    Length:   
                         Size  2 
                             \ensuremath{\times}    35 
                           cm    ; Height:  
                            1  3   
                           \ensuremath{\times}    Length   
                    
                  
                  
                    
                      Weight:     
                     Length  2    kilograms
                         
                    
                  
                  
                   Str   
                   4   
                   9   
                   16   
                   25   
                   36   
                   49   
                   64   
                   81   
                   100   
                   121   
                   144   
                  
                  
                   Bld   
                   4   
                   9   
                   16   
                   25   
                   36   
                   49   
                   64   
                   81   
                   100   
                   121   
                   144   
                  
                  
                   Dex   
                   6   
                   6   
                   6   
                   6   
                   6   
                   6   
                   6   
                   6   
                   6   
                   6   
                   6   
                  
                  
                   Serpent   
                    Length:   
                         Size  2 
                             \ensuremath{\times}  40  cm
                              ; Height:  
                            1  8   
                           \ensuremath{\times}  Length   
                    
                  
                  
                    
                     Weight:    Height
                      2    kilograms  
                     
                    
                  
                  
                   Str   
                   4   
                   9   
                   16   
                   25   
                   36   
                   49   
                   64   
                   81   
                   100   
                   121   
                   144   
                  
                  
                   Bld   
                  
                    
                    
                  
                    
                    
                   4   
                   9   
                   16   
                   25   
                   36   
                   49   
                   64   
                   81   
                   100   
                  
                  
                   Dex   
                   8   
                   8   
                   8   
                   8   
                   8   
                   8   
                   8   
                   8   
                   8   
                   8   
                   8   
                  
                  
                   Polymelic   
                    Size:   
                         Size  2 
                            \ensuremath{\times}  30  cm
                             , in some dimension
                    
                  
                  
                    
                     Weight:    Size
                      3    kilograms  
                     
                    
                  
                  
                   Str   
                  
                    
                    
                   4   
                   9   
                   16   
                   25   
                   36   
                   49   
                   64   
                   81   
                   100   
                   121   
                  
                  
                   Bld   
                  
                    
                    
                   4   
                   9   
                   16   
                   25   
                   36   
                   49   
                   64   
                   81   
                   100   
                   121   
                  
                  
                   Dex   
                   7   
                   7   
                   7   
                   7   
                   7   
                   7   
                   7   
                   7   
                   7   
                   7   
                   7   
                  
                  
                   Amorphous   
                    Diameter:   
                         Size  2 
                             \ensuremath{\times}  50  cm
                             
                    
                  
                  
                    
                   Weight:   
                         Diameter  3 
                           kilograms     
                    
                  
                  
                   Str   
                   9   
                   16   
                   25   
                   36   
                   49   
                   64   
                   81   
                   100   
                   121   
                   144   
                   169   
                  
                  
                   Bld   
                   9   
                   16   
                   25   
                   36   
                   49   
                   64   
                   81   
                   100   
                   121   
                   144   
                   169   
                  
                  
                   Dex   
                   1   
                   1   
                   1   
                   1   
                   1   
                   1   
                   1   
                   1   
                   1   
                   1   
                   1   
                  
                
              Monster Construction Table  
            
\subsection{Adding a Template}
     There are several important items missing from the
               base-type. These are covered by Monstrous Templates which
               add things such as mental attributes (and Cha!), special
               abilities, and a description of the monster, including any
               relavent interaction statistics, such as combat, elocution
               etc. 
\section{Monsters}
    
\chapter{Traits}
    
\section{Traits}
    

\end{document}
  