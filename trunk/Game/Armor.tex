\subsection{Basics for Armor and Damage}

\begin{itemize}
\item Armor has 3 main values: Piercing, slashing, and crushing.  
  Armor values are represented by 3 numbers separated by slashes 
  (represented by:  P/S/C; piercing/slashing/crushing).  Example: Armor 
  with values of 4/3/7 would have 4 points of protection vs. Piercing, 
  3 vs. Slashing, and 7 vs. crushing damage.
\item Armor does not protect from Undefined (U) or Direct (D).
\item A character's Base Armor is the only thing that can protect vs. 
  U damage.  Nothing can protect against D damage.
\item All non D/U damage has an associated damage type to is; either P, 
  S, or C.  Subtract all matching armor values of the character from 
  the damage being done.  If the damage value is still above 0, then 
  the character takes that amount of damage.
\item A weapon with an armor piercing (AP) attack will list the amount 
  of AP it has.  Subtract this AP number from all of the armor's 
  damage values when determining its protection against the attack.  
  This cannot drop the armor's value below 0.
\item GMs determine when armor is damaged beyond use/protection.
\item Hard Armor is any armor that uses hard 
  ceramic/plastic/metallic/wooden plates or coverage to protect 
  the character.  Hard armor is MUCH more durable than non-hard armor.  
  Hard armor is denoted by ``hard'' on the armor sheet.  Note that 
  the components of hard armor are$\ldots$hard.  They can't be bent 
  without damaging the armor.  Hard armor is generally made specifically 
  for the wearer, it will only fit others with nearly identical 
  body proportions.
\item Soft armor (armor that isn't considered hard) is at least 
  somewhat flexible.  It will generally fit anyone of similar size/stature. 
\item Armor with the descriptor of ``fire'' has fire protection.  
  This means the armor has the ``natural'' ability to deflect 
  and protect against fire attacks.  The armor will protect from all 
  fire damage (assuming the fire hits the armor, so full suits are 
  better for this).
\item Armor can be layered, but only one hard suit can be worn at a 
  time.  A person can have different hard armors on different parts of 
  the body.  Non-hard armors can be layered indefinitely.  Add all 
  armor values of the armors together to give the full armor protection.  
  If any armor has fire protection, fire will be stopped, but any armor 
  on top of it is still susceptible to the fire.
\item Encumbrance is determined by: BLD/3
\item A character can wear (without penalty) as much armor in size 
  levels as his Encumbrance score will allow.  Add up all layers of 
  armor to get the complete Size level of the armor, if this number 
  is equal or less than Encumbrance, then no negatives are given to the player.
\item When a character has more armor than his Encumbrance (ENC) 
  allows, then the player has the following penalties against him:
  \begin{itemize}
  \item -2 DEX per pt over ENC
  \item -1 action per 2 full points over ENC
  \end{itemize}
\item An armor's Class is a general idea of what period of history/future 
  the armor would appear.  A later period can find the earlier armor.
\end{itemize}

\subsection{Shields}

\begin{itemize}
\item When creating a shield, there are 2 factors.  The material the 
  shield is made of, and the type of shield.  Choose both of these 
  values when creating a shield to get the full set of statistics 
  for the shield.
\item On a successful block, a shield will subtract its appropriate 
  armor value from the damage.  Any damage that penetrates the shield 
  will then be applied against the user's armor.
\item Shields can be used to block U attacks that are ``thrown'' 
  at them.  These include fireballs, lightning strikes, breath 
  weapons, chemical splashes, etc.  To block one of these attacks, 
  the user must make a block roll at -3.  If successful, then damage 
  will be reduced based on the type of shield:
  \begin{itemize}
  \item Buckler: Full damage
  \item Small: $1/4$
  \item Large: $1/2$
  \item Tower: $3/4$
  \end{itemize}
\end{itemize}