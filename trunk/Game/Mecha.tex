\section{Mecha Suits \& Power Armor}
       
A robot in which the pilot is completely protected by armor and other mechanisms.Mecha are controlled by abstract means and requires the Mecha Piloting skill.

\emph{Mecha} denotes armored mechanical vehicles that are built in the form of a human. There are two basic types of mecha: Power Armor and Mecha.  

Power Armor is an armored suit that is ``worn'' like clothing.  It provides armor protection, but it has the function of increasing the wearer's strength, endurance, and weapon payload.  It is easy to use.  The internal computers, etc, in the suit allow it to follow the wearer's movements.

\begin{itemize}
\item Power Armor does NOT require any skill to use
\item Power Armor must be either the same size as the wearer, or exactly one level larger
\end{itemize}


True Mecha is best thought of as a vehicle, like an airplane or tank, which is often bipedal.  Mecha is \emph{piloted}, not worn like power armor.  A mecha has a cockpit, and is controlled with levers, switches, and pedals.  Piloting a mecha is something like flying an airplane or driving a car.

\begin{itemize}
\item The \emph{Mecha Piloting} skill is required to operate a mecha
\item When piloting a mech, any skill you use is capped at your level of mecha piloting.  So, if you had LVL 5 \emph{Dodge}, but have a \emph{Mecha Piloting} skill of 4, you would roll 4 dice to dodge in your mech.
\item Mecha must be AT LEAST one size level larger than the pilot.  Beyond that, they can be ANY size.
\end{itemize}

As a general summary: Power Armor tends to be a lighter and smaller than mecha.  Power Armor can carry less weaponry (and other hardware), but is usually more manueverable.  It is often possible to perform major feats of dexterity while wearing power armor.   This is because the power armor does not hinder the wearer's natural skills.  As well, power armor is easier to use because the Mecha Piloting skill is not required.

Mecha tend to be larger and more powerful than power armor.  Mecha generally have more and heavier weapons, better armor, and superior auxiliary systems compared to powered armor.  However, the pilot needs to be skilled to maximize the potential of a mecha.

\newcounter{MechaSteps}
\setcounter{MechaSteps}{1}

\subsection{Size and Starting Costs Step \arabic{MechaSteps}}

\begin{table}[htb]
\begin{center}
\begin{tabular}{lc|c|c|c|c|c|c|c|c}
\hline
\textbf{Size} & 1 & 2 & 3 & 4 & 5 & 6 & 7 & 8 & 9 \\
\hline
\hline
Height & 4' & 7' & 15' & 30' & 60' & 125' & 250' & 500' & 1000' \\
\hline
Mecha Weight (tons) & 0.5 & 1 & 2 & 16 & 125 & 1000 & 8000 & 65000 & 500000 \\
\hline
Mecha CP Cost & 5 & 10 & 20 & 40 & 80 & 160 & 320 & 640 & 1280 \\
\hline
Power Armor CP Cost & 1 & 4 & 9 & 16 & 25 & 50 & 100 & 200 & 400 \\
\hline
\end{tabular}
\caption{Step \arabic{MechaSteps}: Size Level and Starting CP Cost}
\end{center}
\end{table}
\addtocounter{MechaSteps}{1}


\begin{itemize}
\item Mecha can be no smaller than one size level greater than the wearer's size
\item Humans are size level 2, weight is given in tons
\item Height and weight indicates the maximum size for a humanoid shaped robot for a given size level
\end{itemize}

\subsection{Start Values \arabic{MechaSteps}}

\begin{table}[htb]
\begin{center}
\begin{tabular}{lc|c|c|c|c|c|c|c|c}
\hline
\textbf{Size} & 1 & 2 & 3 & 4 & 5 & 6 & 7 & 8 & 9 \\
\hline
\hline
Mecha Slots & 10 & 20 & 40 & 80 & 160 & 320 & 640 & 1280 & 2560 \\ \hline
Power Armor Slots & 8 & 20 & 35 & 60 & 100 & 190 & 330 & 575 & 1000 \\ \hline
Mecha Armor & 5 & 10 & 20 & 30 & 40 & 50 & 60 & 70 & 80 \\ \hline
Power Armor Armor & 2 & 6 & 12 & 20 & 30 & 40 & 50 & 60 & 70 \\ \hline
Dex Minus & -1 & -2 & -3 & -4 & -5 & -6 & -7 & -8 & -9 \\ \hline
A. Penalty & 3 & 5 & 10 & 15 & 20 & 25 & 30 & 35 & 40 \\ \hline
Hit Points & 75 & 150 & 300 & 600 & 1200 & 2500 & 5000 & 10000 & 20000 \\ \hline
Strength & 5 & 10 & 20 & 40 & 80 & 160 & 320 & 640 & 1280 \\ \hline
Power & 2 & 5 & 10 & 20 & 40 & 80 & 160 & 320 & 640 \\
\hline
\end{tabular}
\caption{Step \arabic{MechaSteps}: Starting Values for Statistics}
\end{center}
\end{table}
\addtocounter{MechaSteps}{1}

\begin{description}
\item[Slots] The amount of equipment that can be installed on the suit
\item[Armor] Starting armor value of the suit (All equal for P/S/C)
\item[Dex Minus] Robots have some control-lag, this is represented by a DEX minus to the pilots DEX attribute
\item[A. Penalty] Armor Penalty.  For every full amount of this value, the DEX of the pilot is reduced by 1.
\item[Power] is the number of PC per turn consumed by the robot, just to run basic needed systems
\item[Hit Points] The starting number of HP that a mecha has
\item[Strength] The starting strength value for the mecha
\end{description}

\subsection{Frame Style \arabic{MechaSteps}}

\emph{This section is for Mecha only!}

\begin{table}[htb]
\begin{center}
\begin{tabular}{cc||c|c|c|c|c}
\hline
\multicolumn{2}{l}{\textbf{Frame Table}} &
\multicolumn{5}{c}{\textbf{Frame Style (Added Slots)}} \\
\multicolumn{2}{c}{} & V. Light & Light & Medium & Heavy & V. Heavy \\
\multicolumn{2}{c}{Frame Material (Added HP)} &
-20\% & -10\% & +0\% & +10\% & +20\% \\
\hline
\hline
Carbon Composite & -20\% & XXXX & XXXX & XXXX & XXXX & 0 \\ \hline
Aluminum Alloy & -10\% & XXXX & XXXX & XXXX & 0 & 2 \\ \hline
Standard Steel & +0\% & XXXX & XXXX & 0 & 2 & 4 \\ \hline
Heavy Steel & +20\% & XXXX & 0 & 2 & 4 & 6 \\ \hline
Duralimin Alloy & +40\% & 0 & 2 & 4 & 6 & 8 \\ \hline
Matrix Composite & +60\% & 2 & 4 & 6 & 8 & 10 \\ \hline
Adamantite Alloy & +80\% & 4 & 6 & 8 & 10 & 12 \\ \hline
Gundanium Alloy & +100\% & 6 & 8 & 10 & 12 & 14 \\ \hline
\end{tabular}
\caption{Step \arabic{MechaSteps}: Frame Style}
\end{center}
\end{table}
\addtocounter{MechaSteps}{1}

The number of HP and Slots that a mecha has can be further altered by the internal frame's style and material makup. Frame Style determines the efficiency of how bulky the internal frame of the mecha is, this adjusts the number of slots that are within the mecha. Frame Material is the makeup of the frame, what it is made of.  Different materials offer decreased or added hit points to the mecha. Choose the type of Frame Style, as well as the Material.  Cross reference them on the chart below, and multiply the number by the size of the mecha to get the CP cost for that configuration.

For example, if you chose a "Light Matrix Composite" frame for a size 4 mecha, it will receive -10\% to it's slot total, would gain +60\% to it's hit points, and cost 16 CP (size of 4*4).

If a mecha suit runs out of HP, it is rendered useless until repaired.  If it goes -HP then it is destroyed. Any time the mecha takes 10\% of it's total HP in one blow, the pilot takes 1D6+1 U damage.

\subsection{Armor \arabic{MechaSteps}}

Starting armor is already listed above, the mecha suit starts with this value in P/S/C. Additional armor is purchased by spending 1 CP per +1/+1/+1 armor value. Different types of armors can be used on the mecha suit.  These different types modify the ratio of the three armor values.  Chose one, but they cannot be mixed.

\begin{table}[htb]
\begin{center}
\begin{tabular}{ccp{3.5in}}
\hline
Type & Ratio of P/S/C & Description \\
\hline
\hline
Normal & 100\% / 100\% / 100\%  & Standard protective armor \\ \hline
Reactive & 80\% / 80\% / 140\%  & Armor will detonate against all attacks, blunting the blow of C damage \\ \hline
Ultrahard & 130\% / 90\% / 80\%  & Extra hard armor fragments piercing weapons \\ \hline
Slick-steel & 90\% / 120\% / 90\%  & Slashing attacks tend to slide off the armor \\ \hline
Reflective & 70\% / 70\% / 70\% &  Armor will protect against energy based attacks as if the attack was not U \\ \hline
\end{tabular}
\caption{Step \arabic{MechaSteps}: Armor}
\end{center}
\end{table}
\addtocounter{MechaSteps}{1}

\subsection{Engines etc. \arabic{MechaSteps}}

Various parts of a mecha or powered suit require power to operate.  This can come from any number of different sources.  A mecha will generally have an ENGINE of some sort that supplies power for moving around, and operating equipment \& weapons.  Some mecha will have supplimentary power sources used for especially power-hungry equipment.  All power sources take up slots within the suit based on the amount of power that it produces.  The more expensive (in CP) a power system is, the more power can be gotten from one slot's worth of the power source. Power is measured in PCs (or Power Cells).  All power sources produce the number of listed PC at the beginning of each of the character's turns.  All systems on a suit will draw power from this, and if there isn't enough power available, that piece of equipment will not function.  So it is vital that the suit has adequate power.

\begin{table}[htb]
\begin{center}
\begin{tabular}{c|c|c|p{3.5in}}
\multicolumn{4}{c}{\textbf{Engines (provide constant power while the mech is operating)}} \\
System & PC / slot & CP / slot & Description \\
\hline
\hline
Solar & 2 & .5 & Solar panels on the suit provide power.  No light = no power \\ \hline
Fuel & 4 & 1 & A standard fuel powered engine.  Easily accessible fuel (gasoline). \\ \hline
Turbine & 6 & 2 & A high efficiency fuel engine.  Fuel is hard to come by and expensive. \\ \hline
Fission & 8 & 4 & A small nuclear reactor provides power. \\ \hline
Fusion & 10 & 6 & Most efficient, and most expensive. \\ \hline
\end{tabular}
\caption{Step \arabic{MechaSteps}: Power Sources}
\end{center}
\end{table}
\addtocounter{MechaSteps}{1}

Engines provide constant power every turn while the mecha is operating.

\begin{table}[htb]
\begin{center}
\begin{tabular}{c|c|c|p{3.5in}}
\multicolumn{4}{c}{\textbf{Booster Engines (provide temporary power for 3 turns only before refueling)}} \\
\hline
\hline
System & PC / slot & CP / slot & Description \\ \hline
LPG & 7 & 1 & Liquified Petroleum Gas micro-turbine \\ \hline
Alcohol & 10 & 2 & Alcohol-driven high-speed rotary generator \\ \hline
Hydrazine & 15 & 3 & Hydrazine-Gasoline fueled generator. \\ \hline
LOX & 22 & 4 & Liquid-Oxygen / Aluminum fueled generator. \\ \hline
\end{tabular}
\caption{Step \arabic{MechaSteps}: Booster Engines}
\end{center}
\end{table}

When turned on, booster engines provide power every turn for three turns in a row.  They cannot be shut offonce they are turned on. They must be refueled before they can be used again.  These are useful for emergency applications or powering large engergy weapons in combat.

\begin{table}[htb]
\begin{center}
\begin{tabular}{c|c|c|p{3.5in}}
\multicolumn{4}{c}{\textbf{Power Cells (store power like a battery.  Can be recharged.)}} \\
\hline
\hline
System & PC / Slot & CP / Slot & Description \\ \hline
Standard & 50 & 1 & Simple capacitors store power \\ \hline
HG & 80 & 2 & special high grade power cell \\ \hline
HD & 120 & 3 & High-density power cell \\ \hline
Super & 170 & 4 & Most efficent power cell \\ \hline
\end{tabular}
\caption{Step \arabic{MechaSteps}: Booster Engines}
\end{center}
\end{table}

Power cells store energy like a rechargeable battery.  They contain a certain amount of power, which can be drawn from them at will.  Once the power is used up, they must be recharged before they can be used again.  Better power cells can store more energy in a smaller amount of space.

\subsection{Movement \arabic{MechaSteps}}

\subsubsection{Mecha}
DEX can only be used to reduce the DEX minus to a +0.
\textbf{DEX Bonus:} $2*Size CP per +1$

\textbf{STR Bonus:} $(Size-1) per +1$

\subsubsection{Power Armor}
Power armor STR bonus increases the wearer's own STR while wearing the armor
\textbf{STR:} $3 CP per +(Size-1)$

\textbf{DEX:} $(2*Size) CP per +1$

\subsection{Movement Systems \arabic{MechaSteps}}

A mecha suit or power armor needs to move.  All movement systems use power.  Unless otherwise stated, the systems use their power only when activated (being used).

\begin{table}[htb]
\begin{center}
\begin{tabular}{c|c|c|c|c|c|p{1.5in}}
\textbf{Type} & \textbf{Speed} & \textbf{Altitude} & \textbf{Power} & \textbf{Slots} & \textbf{CP} & \textbf{Notes} \\
\hline
\hline
Legs & 20 mph & 5*Size & $(Size-1)*6$ & 0 & 4 \\ \hline
Wheels & 50 mph & 0 & $(Size-1)*2$ & 1 & 3 \\ \hline
Treads & 30 mph & 0 & $(Size-1)*4$ & 2 & 4 \\ \hline
Micro-hover & 20 mph & 15' & $(Size-1)*10$ & 4 & 8 \\ \hline
True-hover & 30 mph & 200' & $(Size-1)*15$ & 6 & 10 \\ \hline
Jump jets & 0 mph & 100' & $Size*10$ & 1 & $Size^2$ \\ \hline
Mech. Jump & 0 mph & 50' & 0 & 2 & $Size^2$ \\ \hline
Manuever Jets & 40 mph & 5' & $(Size-1)*20$ & 1 & ${3*(X*Size-1)}\over{4}$ & Gives [+2X] to dodge \\ \hline
Jet System & 200 mph & Unlimited & $(Size-1)*20$ & 10 & 15 \\ \hline
Added Speed & +15 mph & n/a & n/a & n/a & 1 & Not for jet system \\ \hline
Added Speed & +50 mph & n/a & n/a & n/a & 1 & For jet system only \\ \hline
\end{tabular}
\caption{Step \arabic{MechaSteps}: Booster Engines}
\end{center}
\end{table}
\addtocounter{MechaSteps}{1}

\subsection{Equipment \arabic{MechaSteps}}
Purchase equipment from the generic equipment sheet.

\addtocounter{MechaSteps}{1}

\subsection{Weapons \arabic{MechaSteps}}
Purchase weapons from the generic weapons sheet. Note: a mecha suit can weild any hand-held weapon that is no bigger than STR/2 slots.
