\section{Introduction}

Augmentations are modifications to the body made through technology. 
This can be anything from using a metal implant to repair a broken bone, to 
replacing a person's brain with a computerized electronic brain.

In general, augmentations fall into two categories.  They are either 
ADDITIONS to the body or REPLACEMENTS of something already there.  

An addition is something that is added to the body that normally isn't there.
A good example is a cardiac pacemaker.  Additions add new functionality to the 
body---either as a ``fix'' for an illness, or as an upgrade to the body.  Some 
additions are very small devices that perform simple functions.  Others are 
large and complex systems that might require significant body modification!

A replacement is when a part of the body is removed and replaced by something 
which performs a similar task.  Replacements are usually upgrades of some kind, 
but might also be used as a repair for a body part that was injured or for 
utility purposes.

\subsection{Additions}

There are two basic types of addition augmentations.  The first are small 
devices that don't take up a lot of room.  These are easy to add to the body 
because they don't require any power or a lot of space.  As a general rule, 
this type of augmentation is very small and its function is often limited.

The second type are larger, complex systems, which take up a lot of space in 
the body.  Generally, the only way to add something like this to your body is 
to remove, move, replace, or otherwise modify some \emph{other}
part of your body in 
order to accommodate it.  This works on a system of slots (see below) which is 
similar to the Mecha system.

\subsection{Replacements}

Augmentation replacements, in their most basic forms, are exactly the same 
function-wise as normal body parts.  Of course, many augmentation parts are 
upgrades of some sort.  They might be stronger, faster, or more durable than 
normal.  Some augmentation replacements have new functions or different 
abilities than normal.


\subsection{Augmentation Limbs and Bodies}

Entire limbs or even entire bodies may be replaced with augmented versions.  
There are two basic types of replacement limbs and bodies: Robotic or Cybernetic.

Robotic limbs are built entirely from non-organic materials.  They are machines
made of steel, high-tensile alloys, wires and high pressure hoses.  Movement 
comes from hydraulics, electric motors, gears and levers.  Robotic body parts 
contain no flesh or blood.

Cybernetic limbs are basically organic.  They consist of bone, tendons, and 
muscles just like ``natural'' limbs.  However, cybernetic limbs are enhanced 
with technology including electric implants, genetically modified tissues, 
and some synthetic material reinforcements.

These rules apply to cybernetic and robotic components:

\subsubsection{Robotics:}
\begin{itemize}
\item Do not bleed when injured (though they might leak oil or make sparks)
\item Have unnatural heat signatures
\item Do not look or feel realistic in the slightest; rather they look 
like machinery.
\item Take double damage from electrical sources.
\item Are immune to disease, poison, and magical effects that work on 
living targets (a robotic body part is not alive).  Note that this 
means that healing magic or typical healing/medical items have no effect on robotics.
\item Have double the HP of non-robotic parts.
\item Robotic limbs cannot be upgraded directly.  They can, however, be 
easily removed and replaced with upgraded parts.
\item Used robotic parts generally have high resale value and demand.
Used parts are regularly available.  Many robotic parts are interchangeable.
\item Robotic parts are considered fairly easy to replace and repair.  
Some parts can be swapped in a matter of minutes.
\item Robotic parts do not heal automatically.  If damaged, repairs must be
made by a mechanic or by replacing the damaged parts with new ones.
\end{itemize}

\subsubsection{Cybernetics:}
\begin{itemize}
\item Bleed when injured
\item Appear normal when viewed casually, or on thermal sensors.  Many 
cybernetic limbs will appear 100\% normal even under close visual inspection.  
However, inspection by a knowledgeable person (Doctor, Nurse, Cyber Doctor, 
etc.) will reveal the true nature of the limb.
\item Usually feels realistic to the touch, but not necessarily.
\item Are fully susceptible to disease, poison, and life-affecting magic 
just like any other living tissue (cybernetic body parts are living tissue)
\item Cybernetic parts can be upgraded at a later date without total 
replacement.  However, this usually requires surgery.
\item Cybernetic parts are usually built around the user's body.  Used 
cybernetic parts are valuable, but the potential market is limited to those 
people who are very similar to the original owner.  Therefore the used 
cybernetic parts market tends to be flooded with supply for which it is 
difficult to find a buyer.  It is often time consuming to find a donor or 
source for a particular part.  Cybernetics parts prices for both buyers and 
sellers will vary wildly depending on market conditions, which are nearly 
impossible to predict.
\item Repair to cybernetic limbs is a combination of electro-mechanical work 
and traditional medicine.  Repairing significant injuries or adding/removing 
cybernetics often entails surgery, a hospital stay, and may well require 
lengthy recovery time.
\item Minor injuries to cybernetic limbs heal just like injuries to a normal 
human.  Magical healing and regeneration works for cybernetics in most cases, 
though healing may be incomplete if the damage is severe.
\end{itemize}

\subsection{General Augmentations Rules}

\begin{itemize}
\item All humanoid characters have, by default, (size) free slots in the body.
These are assumed to be located in the torso.  (Humans, which are size 2, 
start with +2 slots) If you want more slots to store things in, you have to 
purchase augmentations that give you back additional slots.
\item After all augmentations are chosen for a character, add up the slot 
total.  This number must be positive.  (In other words, augmentations cannot 
be installed unless there are slots available to house those parts)
\item A robotic limb may be attached to any kind of torso.  However, a 
cybernetic limb can only be attached to a natural (non-robotic) torso.  
This is because all cybernetic limbs depend on the support organs and 
blood supply found in the torso.
\end{itemize}

\subsection{Determining the Cost of an Augmentation body part}