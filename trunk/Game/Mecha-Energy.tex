Energy weapons draw energy straight from the engines or batteries. There
are multiple different methods for producing the blast that comes from
an energy weapon, but all of them share some things in common. First, the
base damage-table is used to calculate the damage an energy weapon
does based on it's class:

\begin{table}[htb]
\begin{center}
\begin{tabular}{c|c|c|c}
\textbf{Class} & \textbf{Energy Units} & \textbf{Damage} & \textbf{C.P.} \\ \hline \hline
0 & 5 & 5D6 P & 2 \\ \hline
1 & 10 & 1D6 x10 P & 4 \\ \hline
2 & 15 & 2D6 x10 P & 6 \\ \hline
3 & 20 & 4D6 x10 P & 8 \\ \hline
4 & 25 & 8D6 x10 P & 10 \\ \hline
\end{tabular}
\caption{Sizes, number of die with multiplier, and energy usage for energy weapons.}
\end{center}
\end{table}

Energy weapons use Class+2 slots.

All Energy weapons can be ``tuned'' for various effects (one
or the other), which takes (Class+1) hours to do in a professional shop. Unless
otherwise stated all weapons have a range of (Class+3)*10000 feet.

\begin{description}
\item[OverFire] spends twice as much energy but deals 50\% more damage
\item[Flame] spends twice as much energy but also causes the target to suffer
the damage as U (Undefined) Fire
\item[Drawing] spends three times as much energy, deals S (Slashing) damage instead of P
\item[Pulse] spends 1/2 as much energy, use four-sided dice (D4) instead of 6.
\item[Higgs Particle] weapon deals C (Crushing) damage, but the maximum range is
(Class+1)*100 feet
\item[Area Effect] uses twice the energy, but has (Class+1)*5' area effect
\end{description}

When buying an energy weapon you may pay it's cost again to add an effect. Certain
combinations of effects are ``named'' below.

After every twenty shots an Energy weapon \emph{must} be retuned or else
it will suffer damage. For every five shots after the twenty-limit, roll
a 1D20 with a minus equal to the number of shots, plus (Class+1)*2. If the
die-roll fails, the weapon is \emph{slagged} and requires (Class+1)*10 hours
of repair in a professional shop.
